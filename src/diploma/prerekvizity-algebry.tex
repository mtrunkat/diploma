\section{Algebry a moduly}\label{algebry-moduly}

  \paragraph{ }V této části budeme pracovat s pevně zvoleným
  asociativním okruhem s jednotkou $R$, který navíc bude 
  komutativní, lokální a artinovský. Nejprve si tyto definice připomeneme:
    
  \begin{dfn}
    Komutativní okruh $R$ je lokální, pokud má právě jeden maximální pravý ideál a ten je nenulový.
  \end{dfn}
  
  Pokud je $R$ je lokální s maximálním pravým ideálem $\underline 
  m$, pak $\underline m$ je oboustranný ideál a je zároveň maximálním 
  levým ideálem dle \cite{1} Lemma I.4.6.
  
  My ale budeme pracovat s komutativním okruhem $R$, ten je 
  lokální jednoduše právě tehdy, 
  když má právě jeden maximální nenulový ideál.
  
  \begin{lem}\label{faktor-lokalniho-lokalni}
    Pokud je $R$ lokální okruh, pak $R/J$ je lokální okruh pro každý vlastní 
    ideál $J$ okruhu $R$.
  \end{lem}
  \begin{proof}
    Buď $R$ lokální okruh, $J$ jeho vlastní ideál a nechť $\underline{m}$ 
    značí maximální ideál okruhu $R$. Dokážeme, že $J/\underline{m}$ je 
    maximální jednoznačně určený ideál okruhu $R/J$.
    
    Mějme libovolný ideál $Y$ okruhu $R/J$, pak $Y$ je tvaru $Y=X/J$ pro nějaký 
    ideál $X$ okruhu $R$ takový, že $X\subsetneq Y\subsetneq R$. 
    Jelikož ideál $\underline{m}$ 
    je maximální, musí být $X\subseteq \underline{m}$ a tedy \\\\
    \centerline{$Y=X/J \subseteq \underline{m}/J$.}
  \end{proof}
  
  \begin{dfn}
    Okruh $R$ je zleva (resp. zprava) artinovský, pokud se každý klesající 
    řetězec jeho levých (resp. pravých) ideálů zastaví. Neboli máme-li klesající řetězec 
    levých (resp. pravých) ideálů okruhu $R$ \\\\
    \centerline{$R=I_0\supseteq I_1\supseteq I_2\supseteq\ldots$,} \\\\
    pak existuje $n\in \mathbb{N}$ takové, že $I_j=I_i$ pro každé $i,j\geq n$.
  \end{dfn}
  
  \begin{dfn}
     Jakobsonův radikál $rad(R)$ okruhu R definujeme jako průnik všech 
     maximálních ideálů $R$.    
  \end{dfn}
  
    Dále tedy nechť $R$ značí pevně zvolený komutativní, lokální a artinovský okruh.
    Nyní zavedeme klíčový pojem $R$-algebry.
        
    \begin{dfn}
      $R$-algebra $A$ je okruh, který je zároveň $R$-modulem takovým, že pro 
      $\alpha,\beta,\lambda\in A$ a $r,s\in R$ platí:
      \begin{description}
        \item[(a)] $(r\alpha+s\beta)\lambda=r(\alpha\lambda)+s(\beta\lambda)$ 
        \item[(b)] $\alpha(r\beta+s\lambda)=r(\alpha\beta)+s(\alpha\lambda)$ 
      \end{description}      
      Artinovská $R$-algebra je $R$-algebra, která je konečně generovaná jako 
      $R$-modul.  
      Opačná algebra $A^{op}$ k algebře $A$ je algebra se stejnou 
      modulovou strukturou,  která má ale okruhové násobení definováno opačně: \\\\
      \centerline{$a\cdot_{A^{op}}b:=b\cdot_A a$}  \\\\
      Kategorii levých $A$-modulů značíme $Mod(A)$. O pravých $A$-modulech
      budeme referovat jako o $A^{op}$-modulech. 
      Poznamenejme, že v části \hyperref[teorie-reprezentaci]{části \ref*{teorie-reprezentaci}}  
      zabývající se teorií reprezentací a poté v 
      \hyperref[teorie-reprezentaci]{Kapitole \ref*{implementace}}  zabývájící se implementací algoritmu budeme pracovat s pravýmu 
      moduly a $Mod(A)$ bude tedy značit kategorii pravých modulů. Na vše později upozorníme.
      
      $R$-podmodul $I$ 
      $R$-algebry $A$ je pravým (resp. levým) ideálem $A$, pokud $xa\in I$ (resp. $ax\in I$)
      pro každý $x\in I$ a $a\in A$. Oboustranný ideál je ideál, který je 
      zároveň levým i pravým ideálem.  
    \end{dfn}
    
    Dále bude $A$ značit pevně zvolenou artinovskou $R$-algebru a pojďme se 
    podívat na vztah $R$-modulů a $A$-modulů.
        
    \begin{dfn}
      Nechť $S$ je okruh. Označme $mod(S)$ úplnou podkategorii $Mod(S)$ konečně 
      generovaných modulů.
    \end{dfn}
    
    \begin{lem}\label{lem-mod-Mod} Platí:
      \begin{description}
        \item[(a)] $Mod(A)\subseteq Mod(R)$
        \item[(b)] $Hom_A(M,N)\subseteq Hom_R(M,N)$ pro $M,N\in Mod(A)$ 
        \item[(c)] $mod(A)\subseteq mod(R)$   
        \item[(d)] Pokud $M\in Mod(A)\cap mod(R)$, pak $M\in mod(A)$.  
      \end{description}
      Poznamenejme, že v případech (a) i (c) se jedná o podmnožinu, ne o úplnou 
      podkategorii.
    \end{lem}
    \begin{proof}
      \begin{description}
        \item
        \item[(a)] Pro $M\in Mod(A)$ definujeme násobení $R\times M\rightarrow M$ následovně: 
          \\\\
          \centerline{$(r,m)\mapsto (r\cdot 1_A)m$}\\\\
          Protože $r\cdot 1_A\in A$, je násobení dobře definované. 
        \item[(b)] Nechť $M,N\in Mod(A)$ a $f\in Hom_A(M,N)$. Pak pro $m_1,m_2\in M$ 
          $a\in A$ máme \\
          \centerline{$f(am_1+m_2)=af(m_1)+f(m_2)$.}\\\\
          A tedy pro každé $r\in R$ dle bodu (a) platí, že
          $f(rm_1+m_2)=f((r1_A)m_1+m_2)=(r1_A)f(m_1)+f(m_2)=rf(m_1)+f(m_2)$.
        \item[(c)] Nechť $M\in mod(A)$, pak pro nějaké $n\in\mathbb{N}$ existuje
          $A$-modulový epimorfismus \\
          \centerline{$g:A^n\rightarrow M$.}\\\\
          Ten je dle bodu (b) také $R$-modulovým epomorfismem. Algebra $A$ je 
          artinovská, tedy konečně generovaná jako $R$-modul. Pak existuje  
          $R$-modulový epimorfismus \\
          \centerline{$h:R^m\rightarrow A$}\\\\
          a také $R$-modulový epimorfismus \\\\
          \centerline{$h^n:R^{mn}\rightarrow A^n$.}\\\\
          Složením $g$ a $h^n$ dostaneme $R$-modulový epimorfismus \\\\
          \centerline{$gh^n:R^{mn}\rightarrow M$,}\\\\
          z jehož existence plyne, že $M\in mod(R)$
        \item[(d)] Struktura $R$-modulu je na $M$ definována následovně:\\\\
          \centerline{$rm=(r1_A)m$} \\\\
          Nechť $\{m_1,m_2,\cdots,m_n\}\subseteq M$  generuje $M$ jako $R$-modul. 
          Pak pro každé $m\in M$ platí: 
          \begin{eqnarray}
            m &=& \sum_{i=1}^n r_i m_i=\sum_{i=1}^n \underbrace{(r_i1_A)}_{\in A} m_i\subseteq\sum_{i=1}^n Am_i 
            \nonumber
          \end{eqnarray} 
          A tedy množina $\{m_1,m_2,\cdost,m_n\}\subseteq M$  generuje $M$ i jako 
          $A$-modul.
      \end{description}    
    \end{proof}
    
    \begin{thm}\label{dir-sum-hom} Nechť $S$ je okruh, $\bigoplus_{i=1}^nX_i$ je direktní součet 
       $S$-modulů a pro $i=1,\ldots,n$ jsou $\nu_i:X_i\rightarrow \bigoplus_{i=1}^nX_i$
       kanonické inkluze a $\rho_i:\bigoplus_{i=1}^nX_i\rightarrow X_i$ 
       kakonické projekce. Pak máme pro $Y\in Mod(S)$ následující izomorfismus: \\\\
       \centerline{$\xi:Hom_S(\bigoplus_{i=1}^nX_i,Y)\rightarrow \bigoplus_{i=1}^nHom_S(X_i,Y)$}\\\\
       daný předpisem $f\mapsto\{f\nu_i\}_{i=1}^n$ a jeho inverz\\\\
       \centerline{$\xi^{-1}:\bigoplus_{i=1}^nHom_S(X_i,Y)\rightarrow Hom_S(\bigoplus_{i=1}^nX_i,Y)$}\\\\
       daný předpisem $\{f_i\}_{i=1}^n\mapsto \sum_{i=1}^nf_i\rho_i$.    
       
       Pokud buď $Y=S$, nebo pro všechny $n=1,2,\ldots,n$ je $X_i=S$, pak $\xi$ 
       a $\xi^{-1}$ jsou izomorfismy $S$-modulů.
     \end{thm}
     
     \begin{proof}
       Dokážeme pouze první část, druhá část je přímočaré ověření modulových axiomů. 
       
       Nechť tedy $f\in Hom_S(\bigoplus_{i=1}^nX_i,Y)$. Pak 
       \begin{eqnarray}
         \xi^{-1}\xi(f) 
         &=& \xi^{-1}(\{f\nu_i\}_{i=1}^n) \nonumber \\
         &=& \sum_{i=1}^n f\nu_i\rho_i \nonumber \\
         &=& f \sum_{i=1}^n \nu_i\rho_i \nonumber \\
         &=& f 1_{\otimes_{i=1}^n X_i} \nonumber \\
         &=& f.  \nonumber
       \end{eqnarray}
       
       Opačně nechť $\{f_i\}_{i=1}^n\in {\bigoplus_{i=1}^n Hom_S(X_i,Y)$, pak
       \begin{eqnarray}
         \xi\xi^{-1}(\{f_i\}_{i=1}^n)
         &=& \xi \left( \sum_{i=1}^n f_i\rho_i \nonumber \right)\\
         &=& \left\{  \left( \sum_{i=1}^n f_i\rho_i \right) \nu_j \right\}_{j=1}^n \nonumber \\
         &=& \left\{ \sum_{i=1}^n f_i(\rho_i \nu_j) \right\}_{j=1}^n  \nonumber \\
         &=& \left\{ \sum_{i=1}^n f_i \delta_{i,j} \right\}_{j=1}^n  \nonumber \\
         &=& \left\{ f_j \}_{j=1}^n  \nonumber
       \end{eqnarray}
     \end{proof}
    
    Dále budeme pracovat s pojmem tenzorového součinu, ten zde nebudeme zavádět. 
    Definici a všechny potřebné vlastnosti je možné nalézt například v \cite{5} 
    Kapitole 2.2, nebo v \cite{3} Kapitole 2.8. 
    
    Připomeňme pouze, že pro $M\in Mod(A^{op})$, $N\in Mod(A)$
    tenzorový součin $M\otimes_A N$ vždy existuje, má strukturu $R$-modulu a 
    máme následující dva funktory
    \\\\ \centerline{$M\otimes_A -: Mod(A)\to Mod(R)$} \\\\ 
    \centerline{$-\otimes_A N: Mod(A^{op})\to Mod(R)$} \\\\ 
     z nichž první je zleva exaktní. 
          
     \begin{thm}\label{thm-adjunkce}
       Nechť $M\in Mod(A^{op})$, $N\in Mod(A)$ a $L\in Mod(R)$. Pak máme 
       následující izomorfismus abelovských grup 
       \\\\
       \centerline{$\theta_{M,N,L}:Hom_R(M\otimes_A N, L)\rightarrow Hom_A(N,Hom_R(M,L))$} 
       \\\\
       daný předpisem \\
       \centerline{$\theta_{M,N,L}(f):=[n\mapsto f(-\otimes n)]$}
       \\\\
       pro $f\in Hom_R(M\otimes_A N,L)$. Navíc $\theta_{M,N,L}$ je přirozený v 
       $M$, $N$ i $L$.
     \end{thm}
     \begin{proof}
       \cite{5} Theorem 2.75, 2.76.
     \end{proof}     
     
     \begin{lem}[Lemma five]\label{lemma-five} Uvažujme následující diagram 
       v $Mod(A)$ s exaktními řádky:\\
       \centerline{\xymatrix{
       A_1 \ar@{->}[r]_{f_1} \ar@{->}[d]_{t_1} 
         & A_2 \ar@{->}[r]_{f_2} \ar@{->}[d]_{t_2} 
         & A_3 \ar@{->}[r]_{f_3} \ar@{->}[d]_{t_3} 
         & A_4 \ar@{->}[r]_{f_4} \ar@{->}[d]_{t_4} 
         & A_5 \ar@{->}[d]_{t_5}  \\
       B_1 \ar@{->}[r]_{h_1}
         & B_2 \ar@{->}[r]_{h_2} 
         & B_3 \ar@{->}[r]_{h_3}  
         & B_4 \ar@{->}[r]_{h_4} 
         & B_5  \\
      }}\\\\
      Pak platí:
      \begin{description}
        \item[(a)] Jsou-li $t_2$ a $t_4$ epimorfismy a $t_5$ monomorfismus, pak 
        je $t_3$ epimorfismus.
        \item[(b)] Jsou-li $t_2$ a $t_4$ monomorfismy a $t_1$ epimorfismus, pak 
        je $t_3$ monomorfismus.
        \item[(c)] Jsou-li $t_1$, $t_3$, $t_4$ a $t_5$ izomorfismy, pak je i $t_3$ 
        izomorfismus.
      \end{description}
     \end{lem}
     
     \begin{proof}
       Tvrzení (c) jasně plyne z (a) a (b). Dokážeme zde pouze (a), jelikož 
       (b) se dokáže analogicky.
       
       Nechť $b_3\in B_3$. Protože $t_4$ je epimorfismus, tak existuje $a_4\in A_4$ 
       takové, že $t_4(a_4)=h_3(b_3)$. Z komutativity posledního čtverce a 
       exaktnosti spodního řádku v $B_4$ dostaneme \\
       \centerline{$t_5h_4(a_4)=h_4t_4(a_4)=h_4h_3(b_3)=0$.} \\\\
       Protože $f_5$ je monomorfismus musí být $f_4(a_4)=0$ a $a_4\in 
       Ker(f_4)=Im(f_3)$. Tedy existuje $a_3\in A_3$ takové, že $a_4=f_3(a_4)$ a 
       tedy platí, že \\\\
       \centerline{$h_3t_3(a_3)=t_4f_3(a_3)=t_4(a_4)=h_3(b_3)$.} \\\\
       Pak $b_3-t_3(a_3)\in Ker(h_3)=Im(h_2)$ a tedy existuje $b_2\in B_2$ takové, že 
       $h_2(b_2)=b_3-t_3(a_3)$. Protože $t_2$ je epimorfismus, musí existovat $a_2\in A_2$, 
       pro které $t_2(a_2)=b_2$. Z komutativity diagramu plyne \\\\
       \centerline{$t_3f_2(a_2)=h_2t_2(a_2)=h_2(b_2)=b_3-t_3(a_3)$} \\
       a \\
       \centerline{$t_3(f_2(a_2)+a_3)=b_3$.} \\\\     
       Homomorfismus $t_3$ je tedy epimorfismem.  
     \end{proof}
     
     \begin{lem}\label{lem-komut-schod}
       Nechť $S$ je okruh a nechť \\\\
       \centerline{\xymatrix{
         0 \ar@{->}[r] 
           & M \ar@{->}[r]^f
           & N \ar@{->}[r]^g \ar@{->}[d]_\alpha^\simeq
           & L \ar@{->}[d]_\beta^\simeq \\
        & & N' \ar@{->}[r]^u
           & L' \ar@{->}[r]^v
           & K' \ar@{->}[r]
           & 0   
      }}\\\\\\
      je komutativní diagram s exaktními řádky v $Mod(S)$. Pak\\\\
       \centerline{\xymatrix{
         0 \ar@{->}[r] 
           & M \ar@{->}[r]^f
           & N \ar@{->}[r]^g 
           & L \ar@{->}[r]^{v\beta}
           & K' \ar@{->}[r]
           & 0   
      }}\\\\      
      je exaktní posloupnost v $Mod(S)$.
     \end{lem}
     
     \begin{proof}
       Exaktnost v $M$ a $N$ plyne z exaktnosti původního diagramu. 
       Protože $v$ a $\beta$ jsou epimorfismy, pak je $v\beta$ také epimorfismem, takže 
       posloupnost je exaktní i v $K'$. 
       
       Protože $(v\beta)g=(vu)\alpha=0$ máme $Im(g)\subseteq Ker(v\beta)$. Zbývá dokázat 
       opačnou inkluzi.
       
       Nechť $c\in Ker(v\beta)$, pak $v\beta(c)=0$ a tedy $\beta(c)\in 
       Ker(v)=Im(u)$. Potom existuje $b'\in N'$ takové, že \\\\
       \centerline{$u(b')=\beta(c)$.}\\\\
       Pak $\alpha^{-1}(b')\in N$ a
       \\\\
       \centerline{$\beta g \alpha^{-1}(b')=u(b')=\beta(c)$.}\\\\
       Protože $\beta$ je epimorfismus, tak nám tato úvaha dává
       \\\\
       \centerline{$g(\alpha^{-1}(b'))=c$}\\\\
       a tedy $c\in Im(g)$.
     \end{proof}
     
   \subsection{Projektivní moduly}
   
     \begin{dfn}
       $A$-modul $P$ nazveme projektivním, pokud pro každé $A$-moduly $M,N$, 
       epimorfismus $h\in Hom_A(M,N)$ a homomorfismus $f\in Hom_A(P,N)$ existuje 
       $f'\in Hom(P,M)$ takový, že následující diagram komutuje:\\\\
       \centerline{\xymatrix{
         & P \ar@{.>}[ld]_{f'} \ar@{->}[d]^f \\
         M \ar@{->}[r]_h & N \ar@{->}[r] & 0     
      }}
     \end{dfn}
   
     \begin{lem}\label{lem-faktorizuje-skrze-proj}
       Nechť $M,N\in mod(A)$, $P\in Mod(A)$, $f\in Hom_A(M,N)$ se faktorizuje skrze projektivní modul 
       a $g\in Hom_A(P,N)$ je \textbf{na}. Pak $f$ se faktorizuje skrze $P$.
     \end{lem}
     
     \begin{proof}
       Uvažujme následující diagram v $Mod(A)$:\\\\
       \centerline{\xymatrix{
         & 
           & M \ar@{->}[dd]^f \ar@{->}[dl] \\
         & P' \ar@{->}[rd] \ar@{.>}[ld]_{\exists h} \\
         P \ar@{->}[rr]_g
           & 
           & N \ar@{->}[r]
           & 0      
      }}\\\\
      Homomorfismus $f$ se faktorizuje skrze nějaký projektivní modul $P'$. 
      Protože $P'$ je projektivní a $g$ je \textbf{na}, existuje $h:P'\rightarrow P$ 
      takové, že diagram výše komutuje a $f$ se tedy faktorizuje skrze $P$.
     \end{proof}
        
     \begin{dfn}
       Prvek $e\in A$ nazveme idempotentem, pokud $e^2=e$. 
       Množinu idempotentů $\{e_1,e_2,\ldots,e_n\}$
       nazveme ortogonální, pokud pro každé $i\neq j$ je $e_ie_j=e_je_i=0$. Primitivní 
       idempotent je takový, který není možné zapsat jako součet alespoň 
       dvou nenulových ortogonálních idempotentů. 
       Idempotent je centrální, pokud $ea=ae$  pro každý prvek $a\in A$.
     \end{dfn}
     
     \begin{thm}\label{rozklad-A-na-proj}
       Nechť $A$ je Artinovská algebra. Pak $A=Ae_1\oplus Ae_2\oplus\ldots\oplus 
       Ae_n$ právě tehdy, když $e_1,$ $e_2,\ldots,e_n$ je množina po dvou ortogonálních 
       primitivních idempotentů $A$ a navíc $1=e_1+e_2+\ldots+e_n$.
     \end{thm}
     
     \begin{proof}
       Každá algebra má alespoň dva idempotenty a to $0$ a $1$. Nemá-li žádné 
       jiné, pak máme množinu nenulových primitivních ortogonálních idempotentů 
       $\{1\}$ a rozklad $A=A1$.
       
       Existuje-li netriviální idempotent $e$ algebry $A$, pak je netriviálním 
       idempotentem také $1-e$, protože 
       $(1-e)^2=1-2e+e^2=1-2e+e=1-e$. Ty jsou 
       navíc ortogonální, protože $e(1-e)=e-e^2=e-e=0$. 
       Algebru $A$ tedy můžeme 
       rozložit na $A=Ae\oplus A(1-e)$ a $1=e + (1-e)$.
       
       Protože algebra $A$ je konečně generovaná, můžeme ji rozložit na direktní součet 
       nerozložitelných levých ideálů $A=P_1\oplus P_2\oplus\ldots\oplus P_n$. 
       Ty musejí být z diskuze výše tvaru $P_i=Ae_i$ pro $e_1,e_2,\ldots,e_n$ po 
       dvou ortogonální primitivní idempotenty takové, že $1=e_1+e_2+\ldots+e_n$.
       
       Naopak máme-li $1=e_1+e_2+\ldots+e_n$ rozklad $1$ na součet po dvou 
       ortogonálních primitivních idempotentů je existence rozkladu $A=Ae_1\oplus Ae_2\oplus\ldots\oplus 
       Ae_n$ zřejmá.
     \end{proof}
       
     \begin{dfn}
       Množinu po dvou ortogonálních primitivních idempotentů $e_1,$ $e_2,\ldots,e_n$ 
       takových, že $A=Ae_1\oplus Ae_2\oplus\ldots\oplus Ae_n$ nazveme úplnou 
       množinou primitivních ortogonálních idempotentů $A$.
     \end{dfn}
                       
     \begin{thm}\label{rozklad-proj}
       Nechť $e_1,e_2,\ldots,e_n$ je úplná množina primitivních ortogonálních idempotentů $A$ 
       a $P$ projektivní $A-modul$. Pak existují $m_1,m_2,\ldots,m_n\in\mathbb N$ taková, že\\
       \centerline{$P=(Ae_1)^{m_1}\plus (Ae_2)^{m_2}\oplus\ldots\oplus (Ae_n)^{m_n}$.}       
     \end{thm}
     \begin{proof}
       Dle \cite{5} Theorem 3.5 je modul $P$ projektivní právě tehdy, když existuje $n\in\mathbb 
       N$ takové, že $P$ je direktním sčítancem $A^n$. Výsledný tvar dostáváme 
       aplikací \hyperref[rozklad-A-na-proj]{Věty \ref*{rozklad-A-na-proj}} na $A^n$.
     \end{proof}
     
    \begin{dsl}\label{Ae-projektivni}
       Prvek $e\in A$ je primitivní idempotent právě tehdy, když $Ae$ je 
       nerozložitelný projektivní $A$-modul.
     \end{dsl} 
     
     \begin{thm}\label{izo-hom-aei}
       Buď $e\in A$ je idempotent, pak zobrazení \\\\
       \centerline{$Hom_A(Ae,A)\simeq eA$}\\
       dané předpisem\\
       \centerline{$f\mapsto f(e)$} \\\\
       je izomorfismem $A$-modulů.
     \end{thm}
     \begin{proof}
       Že se jedná o homomorfismus je zřejmé.
       Označme si tento homomorfismus $Hom_A(Ae,A)\simeq eA$ jako $\psi_1$.  
       Definujme zobrazení
       \begin{eqnarray}
         \psi_2:eA &\simeq& Hom_A(Ae,A) \nonumber \\
         ea &\mapsto& [a'e\mapsto a'ea].  \nonumber
       \end{eqnarray}
       Pak pro $f\in Hom_A(Ae,A)$ a $a\in A$ máme
       \begin{eqnarray}
         \psi_1\psi_2(ea)&=&\psi_1([a'e\mapsto a'ea]) \nonumber \\
         &=&ea \nonumber
       \end{eqnarray}
       a
       \begin{eqnarray}
         \psi_2\psi_1(f)(ae)&=&\psi_2(f(e))(ae) \nonumber\\
         &=&\psi_2(ef(e))(ae) \nonumber\\
         &=&[xe\mapsto xe(fe)] \nonumber\\
         &=&aef(e) \nonumber \\
         &=&f(ae). \nonumber
       \end{eqnarray}
       Jde tedy o izomorfismus.
     \end{proof}
     
     \begin{thm}\label{rozklad-A-na-P}
       Buď $P$ projektivní $A$-modul. Pak existuje druhý projektivní $A$-modul $P'$ 
       a $n\in \mathbb N$ takové, že $P\oplus P'\simeq A$.
     \end{thm}
     \begin{proof}
       Dle \hyperref[rozklad-A-na-proj]{Věty \ref*{rozklad-A-na-proj}} můžeme algebru $A$ vyjádřit jako \\\\
        \centerline{$A=A e_1\oplus A e_2\oplus\ldots\oplus A e_m$,} \\\\
        kde $A e_i$ jsou 
        nerozložitelné projektivní $A$-moduly takové, že $Ae_i\not\simeq Ae_j$ 
        pro $i,j\in{1\ldots m}$ a $i\neq j$. Pak lze projektivní modul $P$  dle
        \hyperref[rozklad-proj]{Věty \ref*{rozklad-proj}} vyjádřit jako 
        \\\\
        \centerline{$P\simeq(A e_1)^{n_1}\oplus(A e_2)^{n_2}\oplus\ldots\oplus(A 
        e_m)^{n_m}$.} \\\\       
       Položme $n=max_{1\leq i \leq m}(n_i)$, pak \\\\
       \centerline{$
         A^n\simeq P_0
           \oplus(A e_1)^{n-n_1}
           \oplus(A e_2)^{n-n_2}
           \oplus\ldots
           \oplus(A e_m)^{n-n_m}
       $.} \\\\
       Definujme projektivní $A$-modul \\\\ 
       \centerline{$
         P_0':= 
           (A e_1)^{n-n_1}
           \oplus(A e_2)^{n-n_2}
           \oplus\ldots 
           \oplus(A e_m)^{n-n_m}
       $,} \\\\ 
       pak $P_0\oplus P_0'\simeq A^n$.
     \end{proof}
     
   \subsection{Funktor Ext}
   
     \begin{dfn}
       Injektivní rezolventa E modulu $M\in Mod(R)$ je exaktní posloupnost  \\
       \centerline{\xymatrix{
       \textbf E: 0 \ar@{->}[r] 
         & M \ar@{->}[r]^{\eta}
         &  E^0 \ar@{->}[r]^{d^0}
         &  E^1 \ar@{->}[r]^{d^1}
         &  E^2 \ar@{->}[r]^{d^2}
         &  \cdots,
      }}\\\\
      kde každé $E^n$ je injektivní. 
    \end{dfn}
     
     \begin{lem}
       Pro každý modul $M\in mod(R)$ existuje injektivní rezolventa.
     \end{lem}
     \begin{proof}
       Dle \cite{5} Theorem 3.38 může být každý $R$-modul vnořený jako 
       podmodul do injektivního $R$-modulu. Tedy existuje injektivní $R$-modul $E^0$ a 
       kanonické vnoření $\nu^0:M\rightarrow E^0$. Dostáváme exaktní posloupnost\\\\
       \centerline{\xymatrix{
         0 \ar@{->}[r]
         & M \ar@{->}[r]^{\nu^0}
         & E^0 \ar@{->}[r]^{\pi^0}
         & Cok(\nu^0) \ar@{->}[r]
         & 0  
      },}\\\\
      kde $\pi^0$ je kanonická projekce. Tento postup můžeme iterativně opakovat 
      pro $Cok(\nu^0)$ namísto M a dále s 
      $Cok(\nu^1),$ $Cok(\nu^2),$ $\ldots$:\\\\
       \centerline{\xymatrix{
         0 \ar@{->}[r]
         & M \ar@{->}[r]^{\nu^0}
         & E^0 \ar@{->}[rr]^{d^0:=\nu^1\pi^0} \ar@{->}[rd]_{\pi^0}
         & & E^1 \ar@{->}[rr]^{d^0:=\nu^2\pi^1} \ar@{->}[rd]_{\pi^1}
         & & \ldots \\
         &
         &
         & Cok(\nu^0) \ar@{->}[ru]_{\nu^1}
         &
         & Cok(\nu^1) \ar@{->}[ru]_{\nu^2}
      }}     
    \end{proof}     
    
    \begin{dfn}
      Buď $\mathcal A$ libovolná abelovská katergorie. Komplex $C$ v kagorii $\mathcal A$ je 
      (konečná či nekonečná) posloupnost morfismů a objektů\\\\
       \centerline{\xymatrix{
       \textbf C:  \cdots \ar@{->}[r]^{d_{n+2}}
         &  C_{n+1} \ar@{->}[r]^{d_{n+1}}
         &  C_{n} \ar@{->}[r]^{d_{n}}
         &  C_{n-1} \ar@{->}[r]^{d_{n-1}}
         &  \cdots
      },}\\\\
      kde $d_nd_{n+1}=0$ pro všechna $n\in\mathbb Z$. Dále položme:
      \begin{eqnarray}
        Z_n(C) &:=& Ker(d^n) \nonumber \\
        B_n(C) &:=& Im(d^{n+1}) \nonumber\\
        H_n(C) &:=& Z_n(E)/B_n(E) = Ker(d^n)/Im(d^{n+1}) \nonumber
      \end{eqnarray} 
     \end{dfn}
     
     \begin{dfn}
       Nechť $M,N\in Mod(R)$ a zvolme libovolně injektivní rezolventu modulu $N$\\
       \centerline{\xymatrix{
       \textbf E: 0 \ar@{->}[r] 
         & N \ar@{->}[r]^{\eta}
         &  E^0 \ar@{->}[r]^{d^0}
         &  E^1 \ar@{->}[r]^{d^1}
         &  E^2 \ar@{->}[r]^{d^2}
         &  \cdots
      }.}\\\\
      Aplikujeme-li kovariantní funktor $Hom(M,-)$ na injektivní rezolventu $\textbf 
      E$, dostaneme komplex $Hom_R(M,\textbf E)$:\\\\
       \centerline{\xymatrix{
         0\ar@{->}[r]
         &  Hom_R(M,E^0) \ar@{->}[r]^{d^0_*}
         &  Hom_R(M,E^1) \ar@{->}[r]^{d^1_*}
         &  Hom_R(M,E^2) \ar@{->}[r]^{d^2_*}
         &  \cdots
      }.}\\\\
      Definujme nyní \\\\
      \centerline{$Ext^n_R(M,N):=H^n(Hom_R(M,\textbf E))=\frac{Ker(d_*^n)}{Im(d_*^{n-1})}$,}\\
      kde \\
      \centerline{$d_*^n:Hom_R(M,E^n)\rightarrow Hom_R(M,E^{n+1})$}\\\\
      je dáno předpisem $f\mapsto d^nf$.
     \end{dfn}
     
     Nebudeme zde dokazovat podrobně vlastnosti funktoru $Ext$, vše je obsaženo 
     například v \cite{5} kapitole 6.2.3.
     Pouze bez důkazu uvedeme následující tvrzení: 
     
     \begin{thm}
       \begin{description}
         \item 
         \item[(a)] $Ext^n_R(M,N)$ je nezávislý na volbě injektivní rezolventy 
           modulu $N$.
         \item[(b)] Pro $R$ komutativní je $Ext^n_R(M,-)$ funktor $Mod(R)\rightarrow 
           Mod(R)$.
       \end{description}
     \end{thm}

   \subsection{Funktor $D$}
     
     \begin{dfn}
       Nechť $S$ je okruh.
       \begin{description}
         \item[(a)] Nechť  navíc $M,N\in Mod(S)$ a $M\subseteq N$. Řekneme, že $N$ je esenciálním rozšířením 
         $M$, pokud $X\cap M\neq0$ pro každý podmodul $0\neq X\subseteq N$. 
         \item[(b)] Injektivní obal $I$ modulu $N\in Mod(S)$ je injektivní $S$-modul $I$ společně s 
         monomorfismem $i:M\rightarrow I$, kde $I$ je esenciální rozšíření $Im(i)$. 
       \end{description}
     \end{dfn}
     
     Náš okruh $R$ je komutativní a lokální, má tedy jednoznačně určený maximální ideál, který 
     budeme značit $\underline{m}$. 
     Jako $I$ budeme dále značit injektivní obal 
     jednoduchého $R$-modulu $R/\underline{m}$ (jednoduchost plyne z maximality $\underline{m}$).
     
     \begin{dfn}
       Definujme funktor $D:=Hom_R(-,I)$. Funktor D se nazývá duál.
     \end{dfn}
     
     \begin{thm}\label{lem-dual}
       Funktor $D$ je exaktní a kontravariantní funktor 
       \begin{description}
         \item[(a)] $D:mod(R)\rightarrow mod(R)$ 
         \item[(b)] $D:mod(A)\rightarrow mod(A^{op})$ 
         \item[(c)] V obou předchozích případech je funktor $D$ navíc dualita kategorií.
       \end{description}
     \end{thm}
     \begin{proof}
         Víme, že v obou případech $D$ je zleva exaktní kontravariantní funktor 
         v kategorii abelovských grup. Navíc $D$ je exaktní, protože
       modul $I$  je injektivní dle \cite{5} Proposition 3.25. Dále dokážeme, že 
       $D$ je zároveň funktorem v obou výše uvedených kategoriích:
        
       \begin{description}
         
         \item[(a)] Nechť $M\in Mod(R)$, pak $Hom_M(R,I)\in mod(R)$. Dokážeme,
         že máme-li navíc libovolný modul $M'\in Mod(R)$ a $h\in Hom_R(M,M')$, pak $Dh\in 
         Hom_R(DM,DM')$.
         Nechť tedy $f\in DM'$ a $r\in R$. Pak
         \begin{eqnarray}
           Dh(rf')(m) &=& (-\circ h)_I(rf')(m)\nonumber \\
           &=& (rf'h)(m') \nonumber \\ 
           &=& r(f'h)(m')\nonumber \\
           &=& r(-\circ h)_I(f')(m)\nonumber \\
           &=& rDh(f')(m)\nonumber
         \end{eqnarray}
         pro každé $m\in M$ a tedy \\\\
         \centerline{$Dh(rf')=rDh(f')$.}\\\\
         Platnost rovnosti \\\\
         \centerline{$Dh(f_1'+f_2')=Dh(f_1')+Dh(f_2')$}\\\\
         pro každé $f_1',f_2'\in DM'$ dokazovat nebudeme a přenecháme ji 
         čtenáři.
                  
         \item[(b)] Nechť $M\in mod(A)$, pak $Hom_R(M,I)\in Mod(A^{op})$ a navíc 
         je $Hom_R(M,I)$ konečně generovaný jako $R$-modul a tedy $Hom_R(M,I)\in 
         mod(A^{op})$. 
         Ukážeme pouze, že $D$ zobrazí homomorfismus $A$-modulů na homomorfismus 
         $A^{op}$-modulů, zbytek důkazu opět ponecháme čtenáři. 
         Nechť $M,M'\in mod(A)$ a $h\in Hom_A(M,M')$. Pak pro $f'\in DM'$ 
         a $\lambda\in A$ máme
         \begin{eqnarray}
           Dh(f'\lambda)(m)&=& (-\circ)_I(f'\lambda)(m)  \nonumber   \\    
           &=& (f'\lambda)(\underbrace{h(m)}_{\in M'}) \nonumber \\      
           &=& f'(\lambda h (m)) \nonumber       \\
           &=& f'(h(\lambda m)) \nonumber       \\
           &=& (-\circ h)_I (f')(\lambda m) \nonumber       \\
           &=& \underbrace{Dh(f')}_{\in Hom_R(M,I)}(\lambda m) \nonumber       \\
           &=& (Dh(f')\lambda)(m) \nonumber         
         \end{eqnarray}
         pro každé $m\in M$ a tedy \\\\
         \centerline{$Dh(f'\lambda)=Dh(f')\lambda$.}
         
         \item[(c)] Dokážeme, že  zobrazení $\alpha: 1_{mod(R)}\to D^2$,
         definované pro $M\in mod(R)$ 
         \begin{eqnarray}
           \alpha_M: M &\to& Hom_R(Hom_R(M,I),I) \nonumber \\
           m&\mapsto&[f\mapsto f(m)] \nonumber
         \end{eqnarray}
         je přirozenou ekvivalencí funktorů. Zvolme pevně $M\in mod(R)$. Zřejmě máme 
         $\alpha_M(m)\in  Hom_R(Hom_R(M,I),I)$ 
         pro každé $m\in M$. Mějme 
         $f_1,f_2\in Hom_R(M,I)$ a $r\in R$ , pak         
         \begin{eqnarray}
           \alpha_M(m)(rf_1+f_2) &=& (rf_1+f_2)(m) \nonumber \\
           &=& rf_1(m)+f_2(m) \nonumber \\
           &=& r\alpha_M(f_1) + \alpha_M(f_2).\nonumber 
         \end{eqnarray}                 
         Navíc pro $m_1,m_2\in M$ a $r\in R$ je 
         \begin{eqnarray}
           \alpha_M(rm_1+m_2)(f)&=& f(rm_1+m_2)\nonumber \\
           &=& rf(m_1)+f(m_2)\nonumber \\
           &=& r\alpha_M(m_1)(f_1)+\alpha_M(m_2))f\nonumber \\
           &=& (r\alpha_M(m_1)+\alpha_M(m_2))(f)\nonumber 
         \end{eqnarray}
         pro každé $f\in Hom_R(M,I)$, tedy platí, že         
         \begin{eqnarray}
           \alpha_m(rm_1+m_2) &=& r\alpha_M(m_1) + \alpha_M(m_2) \nonumber
         \end{eqnarray}
         a $\alpha_M$ je tedy $R$-modulovým homomorfismem. Dále dokážeme, že jde 
         o přirozenou transformaci. Nechť $M,M'\in Mod(R)$ a $h\in Hom_R(M,M')$. 
         Pak 
         \begin{eqnarray}
           D^2(h) &=& Hom_R(-,I)Hom_R(-,I)(h)\nonumber \\
           &=& Hom_R(-,I)((-\circ h)_I)\nonumber\\
           &=& (-\circ(-\circ h)_I)_I.\nonumber 
         \end{eqnarray}
         Musíme dokázat, že následující diagram komutuje:
         \\\\
         \centerline{$\xymatrix{
           M \ar[rrr]^{\alpha_M} \ar[d]^h
          &&& Hom_R(Hom_R(M,I),I) 
           \ar[d]^{(-\circ(-\circ h)_I)_I}\\
           M' \ar[rrr]^{\alpha_{M'}}
          &&& Hom_R(Hom_R(M',I),I) \\
         }$}\\\\
         Nechť $m\in M$ a uvažujme následující diagram:
         \\\\
         \centerline{$\xymatrix{
           m \ar@{|->}[rrr] \ar@{|->}[d]
          &&& [f\mapsto f(m)] 
           \ar@{|.>}[d]^?\\
           h(m) \ar@{|->}[rrr]
          &&& [f'\mapsto f'h(m)] \\
         }$}\\\\
         Musíme dokázat, že \\\\
         \centerline{$
         (-\circ(-\circ h)_I)_I([f\mapsto f(m)])= [f'\mapsto f'h(m)].
         $}\\\\
         Pro $f'\in Hom_R(Hom_R(M',I),I)$ platí, že 
         \begin{eqnarray}
           (-\circ(-\circ h)_I)_I([f\mapsto f(m)])(f')
           &=& [f\mapsto f(m)]\circ (-\circ h)_I (f')\nonumber \\
           &=& [f\mapsto f(m)](f'h)\nonumber \\
           &=& f'h(m)\nonumber 
         \end{eqnarray}
         a $\alpha$ je přirozenou transformací. Zbývá dokázat, že pro každé $M\in mod(R)$ 
         je $\alpha_M$ izomorfismem $R$-modulů. Začneme s důkazem, že jde o 
         monomorfismus, neboli že prokaždé $0\neq m\in M$ existuje $f\in Hom_R(M,I)$, že $\alpha_M(m)(f)=f(m)\neq 
         0$.
         Uvažujme homomorfismus $R$-modulů 
         \begin{eqnarray}
           \hat f: Rm&\to& I \nonumber \\
           rm &\mapsto& r+\underline m. \nonumber
         \end{eqnarray}
         Připomeňme, že $\underline m$ značí maximální ideál $R$. Protože $Rm\subseteq M$ 
         je podmodul, tak inkluze $i: Rm\to M$ je monomorfismem $R$-modulů. Z 
         injektivity $I$ existuje $R$-modulový homomorfismus $f:M\to I$ takový, 
         že následující diagram komutuje:  \\\\
         \centerline{$\xymatrix{
           Rm \ar[r]^i  \ar[d]_{\hat f} & M \ar@{.>}[ld]^f \\
           I
         }$}\\\\
         Pak \\\\
         \centerline{$f(m)=fi(m)=\hat f(m)=\hat f(1_Rm)=1_R+ \underline m\neq 0$} 
         \\\\
         a $\alpha_M$ je tedy monomorfismem. Že jde zároveň o epimorfismus a tedy i o izomorfismus se 
         dokáže na základě tvrzení \cite{2} Proposition 1.4 (Kapitola 1). 
         Zároveň vynecháme i důkaz, že $\alpha_M$ s výše uvedenou definicí 
         je pro $M\in mod(A)$ izomorfismem $A$-modulů. 
         Ten se provede přímočaře a přenecháme ho tedy čtenáři.
       \end{description}
     \end{proof}
     
     \begin{dsl}\label{lem-dual-teleso}
       Pro $R=K$ těleso je funktor $D$ daný vztahem $D=Hom_K(-,K)$
       jako funktor $mod(K)\rightarrow mod(K)$.
     \end{dsl}
     \begin{proof}
       Pokud $R$ je těleso, pak  je jeho maximální ideál $\underline{m}=0$, tedy 
       \\\\
       \centerline{$R=R/\underline{m}=K$.}\\\\
       Navíc $R$ jakožto $R$-modul je injektivní a jelikož každý nenulový 
       podmodul $R$ má s $R$ nenulový průnik, je $R$ svým vlastním injektivním 
       obalem a tedy \\\\
       \centerline{$I=R=K$.}
     \end{proof}
     
     \begin{dfn}
       Buď $V\in mod(K)$, kde $K$ je libovolné těleso, a nechť \\\\
       \centerline{$B_V:=\{v_1,v_2,\ldots,v_l\}$}\\\\
       je K-báze V. Definujme zobrazení \\\\
       \centerline{$d_{B_V}:B_V\rightarrow D(V)$} \\\\
       předpisem $d_{B_V}(v_i)(v_j):=\delta_{i,j}$, kde $\delta$ je Kroneckerova 
       delta.
     \end{dfn}
     
     \begin{lem}
       Buď $V\in mod(K)$, kde $K$ je libovolné těleso, a nechť \\\\
       \centerline{$B_V:=\{v_1,v_2,\ldots,v_n\}$}\\\\
       je $K$-báze $V$. Položme \\\\
       \centerline{$dB_V:=\{d_{B_V}(v_1),d_{B_V}(v_2),\ldots,d_{B_V}(v_l)\},$}  \\\\
       lak $dB_V$ je $K$-báze D(V).  Nazveme ji duální bází $D(V)$ 
       vzhledem k $B_V$.
     \end{lem}
     
     \begin{proof}
       Nejprve dokážeme, že jsou prvky $dB_V$ lineárně nezávislé. 
       Nechť $\alpha_1,\ldots,\alpha_l \in K$ jsou takové, že 
       $\sum_{i=1}^l \alpha_i d_{B_V}(v_i)=0$. To znamená, že 
       $\sum_{i=1}^l \alpha_i d_{B_V}(v_i)(v)=0$ pro každé $v\in V$, neboli,  že 
       pro každé $j=1,2,\ldots,l$ máme \\\\
       \centerline{$0
         =\sum_{i=1}^l \alpha_i d_{B_V}(v_i)(v_j)
         =\sum_{i=1}^l \alpha_i \delta_{i,j}
         =\alpha_j$.} \\\\
       A tedy $dB_V$ je lineárně nezávislá množina.
       
       Nyní dokážeme, že libovolné $f\in D(V)$ vyjádříme jako lineární kombinaci 
       prvků $dB_V$. Máme vztah 
       \begin{eqnarray}
         d_{B_V}(v_i)\left(\sum_{j=1}^l\alpha_jv_j\right)
         =\sum_{j=1}^l\alpha_jd_{B_V}(v_i)(v_j)       
         =\alpha_i.  \nonumber          
       \end{eqnarray}
       Pro každé $v=\sum_{j=1}^l\alpha_jv_j$ pak platí
       \begin{eqnarray}
         f(v)
         &=& f\left(\sum_{j=1}^l\alpha_jv_j\right)  \nonumber \\        
         &=& \sum_{j=1}^l\alpha_j f(v_j)  \nonumber \\        
         &=& \sum_{j=1}^l d_{B_V}(v_j)(v) f(v_j)  \nonumber \\     
         &=& \sum_{j=1}^l f(v_j) d_{B_V}(v_j)(v)  \nonumber \\              
         &=& \left( \sum_{j=1}^l f(v_j) d_{B_V}(v_j) \right)(v) \nonumber         
       \end{eqnarray}
       Z toho plyne, že $f=\sum_{j=1}^l f(v_j) d_{B_V}(v_j)$ a tím jsme hotovi.
     \end{proof}
     
     \begin{dsl}
       Nechť $B_V$ je báze vektorového prostoru $V$.
       Koeficienty $f\in D(V)$ vzhledem k bázi $dB_V$ pak spočteme 
       jako obrazy korespondujících prvků báze $B_V$ při zobrazení $f$.
     \end{dsl}
         
     \begin{lem}\label{lem-baze-dual-xi}
       Nechť $V,W\in mod(K)$, kde $K$ je\, těleso.\,\, Dále \, nechť $\xi \in Hom_K(V,W)$ 
       je izomorfismus,  \\\\
       \centerline{$B_V:=\{v_1,v_2,\ldots,v_l\}$}\\\\
       je K-báze V a  \\\\
       \centerline{$\xi(B_V):=\{\xi(v_1),\xi(v_2),\ldots,\xi(v_l)\}$}\\\\
       je korespondující K-báze W. Pak pro každé $1\leq j \leq l$ máme: \\\\
       \centerline{$(D\xi^{-1})(d_{B_V}(v_j))=d_{\xi(B_V)}(\xi(v_j))$}\\
     \end{lem}
     \begin{proof}
       Nechť $1\leq j \leq l$ a $\sum_{i=1}^l\alpha_i\xi(v_i)\in W$. Pak
       \begin{eqnarray}
         (D\xi^{-1})(d_{B_V}(v_j)) \left( \sum_{i=1}^l\alpha_i\xi(v_i) \right)
         &=& d_{B_V}(v_j) \circ \xi^{-1} \left( \sum_{i=1}^l\alpha_i\xi(v_i) \right) \nonumber \\      
         &=& d_{B_V}(v_j) \left( \sum_{i=1}^l\alpha_i v_i \right) \nonumber \\      
         &=& \alpha_j \nonumber \\      
         &=& d_{\xi(B_V)}(\xi(v_j)) \left( \sum_{i=1}^l\alpha_i\xi(v_i) \right) \nonumber        
       \end{eqnarray}
       Z toho již přímo plyne hledaná rovnost.
     \end{proof}
     
   \subsection{Funktor $()^*$}
     
     \begin{dfn}       
       \begin{description} \item
         \item[(a)] Nechť P(A) značí kategorii, jejíž objekty jsou konečně generované 
            projektivní $A$-moduly a $Hom_{P(A)}(P,P')=Hom_A(P,P')$ pro $P,P'\in P(A)$. 
         \item[(b)] Pro $A,B\in mod(A)$ definujeme esenciální epimorfismus $t\in Hom_A(A,B)$
           jako epimorfismus $A$-modulů takový, že platí: Pro každé $M\in mod(A)$ a 
           $u\in Hom_A(M,A)$ takové, že složený 
           homomorfismus $tu$ je epimorfismus, je $u$ epimorfismus.  
           \\\\      
           \centerline{\xymatrix{           
              A
                \ar@{->}[r]^t
              & B  \\
              M
                \ar[ru]_{tu}
                \ar[u]^u
           }}\\
         \item[(c)] Projektivní pokrytí modulu $X$ je projektivní $A$-modul $P$ 
           spolu s esenciálním epimorfismem $t:P\rightarrow X$. Poznamenejme, že 
           někdy budeme o projektivním pokrytí referovat jako o dvojici 
           $(P,t)$, jindy pouze jako o modulu či epimorfismu, pokud nebude druhé 
           v našich úvahach potřeba.
       \end{description}       
     \end{dfn}
     
     \begin{lem}\label{lem-proj-pokryti}
       Nechť $X\in mod(A)$, pak
       \begin{description}
         \item[(a)] je-li $P$ projektivní pokrytí $X$, pak $P\in P(A)$ a $P$ je tedy konečně generovaný. 
         \item[(b)] projektivní pokrytí modulu $X$ je dáno jednoznačně až na izomorfismus. 
         \item[(c)] projektivní pokrytí modulu $X$ existuje.
       \end{description} 
     \end{lem}
     \begin{proof}
       \begin{description}
         \item
         \item[(a)] Buď $P$ projektivní pokrytí $X\in mod(A)$. Pak máme 
         esenciální epimorfismus $t\in Hom_A(P,X)$ a navíc jelikož je $X$ konečně generovaný,  
         existuje $n\in \mathbb{N}$ a epimorfismus $u\in Hom_A(A^n,X)$.\\\\      
           \centerline{\xymatrix{           
               & A^n \ar[d]^u \ar@{.>}[ld]_{\exists v} \\
             P \ar[r]_t
               & X \ar[r] \ar[d] 
               & 0 \\
             & 0  
           }} \\\\ 
         Pak existuje $v\in Hom_A(A^n,P)$ takové, že diagram výše komutuje. Protože $tv=u$ 
         je epimorfismus a $t$ je esenciální epimorfismus, musí být také $v$ 
         epimorfismem a modul $P$ konečně generovaný.
         \item[(b)] Buďte $P$ a $P'$ dvě projektivní pokrytí modulu $X$ a $t,t'$ 
         jejich esenciální epimorfismy. Pak z vlastností projektivního pokrytí 
         existují homomorfismy $h\in Hom_A(P,P')$ a $h'\in Hom_A(P',P)$ takové, 
         že \\\\
         \centerline{$t'h=t$ a $th'=t'$.}\\\\      
         \centerline{\xymatrix{           
               & P \ar[d]^t \ar@/_0.3pc/[ld]_{h} \\
             P' \ar[r]_{t'} \ar@/_0.3pc/[ru]_{h'}
               & X \ar[r] \ar[d] 
               & 0 \\
             & 0  
         }} \\\\ 
         Protože $t$ a $t'$ jsou esenciální epimorfismy, $th'$ a $t'h$ 
         epimorfismy, tak jsou i zobrazení $h$ a $h'$ epimorfismy. Dále máme \\\\
         \centerline{$t'(hh')=(t'h)h'=th'=t'$} 
         \\\\
         a $hh'$ je také epimorfismus. Modul $P'$ je konečně generovaný a $hh'$ 
         je identita. To nám implikuje, že $h$ je monomorfismus a tedy 
         i izomorfimus modulů $P$ a $P'$.
         \item[(c)] Plyne z \cite{2} Theorem 4.2, jelikož $A$ je artinovská $R$-algebra.
       \end{description} 
     \end{proof}
     
     \begin{pzn}
       Připomeňme, že homomorfismus konečně generovaných $A$-modulů má jádro v 
       $mod(A)$, které je určeno jednoznačně až na izomorfismus.
     \end{pzn}
     
     \begin{dfn}
       Minimální projektivní prezentace modulu $X\in mod(A)$ je exaktní 
       posloupnost 
      \\\\      
      \centerline{\xymatrix{     
         P_1 \ar@{->}[r]^s 
           & P_0 \ar@{->}[r]^t 
           & X \ar@{->}[r]
         & 0
      },}\\\\ 
      kde $P_0$ je projektivní pokrytí modulu $X$ a $P_1$ je projektivní 
      pokrytí modulu $Ker(t)\subseteq P_0$. Přesněji $s=iw$, kde $i:Ker(t)\rightarrow P_0$ 
      je kanonická inkluze jádra homomorfismu a $w:P_1\rightarrow Ker(t)$ je 
      projektivní pokrytí modulu $Ker(t)$.\\\\
       \centerline{\xymatrix{
         P_1 \ar@{->}[rr]^{s=iw} \ar@{->}[rd]_w
           & & P_0 \ar@{->}[r]^t 
           & X \ar@{->}[r]
           & 0 \\
         & Ker(t) \ar@{->}[ru]_i
      },}
     \end{dfn}
     
     \begin{lem} 
       Minimální projektivní prezentace modulu $X\in mod(A)$ je dána jednoznačně 
       až na izomorfismus.
     \end{lem}
     \begin{proof}
      Mějme následující dvě projektivní prezentace modulu $X$:\\\\
       \centerline{\xymatrix{
         P_1 \ar@{->}[r]^{s} 
         & P_0 \ar@{->}[r]^t 
         & X \ar@{->}[r]
         & 0 \\
      }}\\\\
       \centerline{\xymatrix{
         P_1' \ar@{->}[r]^{s'} 
         & P_0' \ar@{->}[r]^{t'} 
         & X \ar@{->}[r]
         & 0 \\
      }}\\\\
      Uvažujme inkluze
      \begin{eqnarray}
          i:Ker(t) &\to& P_0 \nonumber \\
         i':Ker(t') &\to& P_0' \nonumber 
      \end{eqnarray}
      a projekce
      \begin{eqnarray}
          \pi:P_1 &\to& Im(s)=Ker(t) \nonumber \\
          \pi':P_1' &\to& Im(s')=Ker(t') \nonumber 
      \end{eqnarray}
       Máme následující diagram:
      \\\\      
      \centerline{\xymatrix{     
         P_1 \ar@{->}[rr]^s \ar@{->}[rd]^{\pi} \ar@{.>}[ddd]^{h_1}
           & & P_0 \ar@{->}[r]^t \ar@{->}[ddd]^{h_0}_{\simeq}
           & X \ar@{->}[r] \ar@{=}[ddd]
         & 0 \\
         & Ker(t) \ar@{->}[ru]_{i} \ar@{.>}[ddd]^{u}_{\simeq} \\
         \\
         P_1' \ar@{->}[rr]^{s'} \ar@{->}[rd]^{\pi'} \ar@{.>}[ddd]^{h_1'} 
           & & P_0' \ar@{->}[r]^{t'} \ar@{->}[ddd]^{h_0'}_{\simeq} 
           & X' \ar@{->}[r] \ar@{=}[ddd]
         & 0 \\
         & Ker(t') \ar@{->}[ru]_{i'} \ar@{.>}[ddd]^{u'}_{\simeq}\\
         \\
         P_1 \ar@{->}[rr]^s \ar@{->}[rd]^{\pi}
           & & P_0 \ar@{->}[r]^t 
           & X \ar@{->}[r]
         & 0 \\
         & Ker(t) \ar@{->}[ru]_{i}\\
      }}\\\\ 
      Izomorfismy $h_0$ a $h_0'$ existují dle 
      \hyperref[lem-proj-pokryti]{Lemma \ref*{lem-proj-pokryti}}. Existence 
      izomorfismů $u$ a $u'$ takových, že 
      \begin{eqnarray}
        h_0i &=& i'u \nonumber \\
        h_0'i'&=&iu'.  \nonumber
      \end{eqnarray}
      plyne z \hyperref[lemma-five]{Lemma \ref*{lemma-five}}.
      
      Protože $\pi'$ je epimorfismus a $P_1$ projektivní, existuje
      $h_1\in Hom_A(P_1,P_1')$ takové, že\\
      \centerline{$u\pi=\pi'h_1$.}\\\\
      Navíc protože je $u\pi$ epimorfismus a $\pi'$ esenciální epimorfismus,
      musí být i $h_1$ epimorfismus. Ekvivalentně existuje epimorfosmus 
      $h_1\in Hom_A(P_1,P_1')$ takový, že\\\\
      \centerline{$u'\pi'=\pi h_1'$.}\\\\
      Složené zobrazení $h_1'h_1$ je epimorfismus a jelikož je modul $P_1$ 
      dle \hyperref[lem-proj-pokryti]{Lemma \ref*{lem-proj-pokryti}} 
      konečně generovaný, je $h_1'h_1$ izomorfismus. To implikuje, že 
      homomorfismus $h_1$ je monomorfismem a tedy i izomorfismem.
     \end{proof}
     
     \begin{dfn}
       Definujme hom funktor $()^* := Hom_A(-,A)$. 
     \end{dfn}
     
     \begin{thm}\label{lem-star}
       Funktor $()^*$ je
       \begin{description}
         \item[(a)] kontravariantním zleva exaktním funktorem $mod(A)\rightarrow mod(A^{op})$. 
         \item[(b)] kontravariantním funktorem $P(A)\rightarrow P(A^{op})$.
       \end{description}
     \end{thm}
     \begin{proof}
       Víme, že $Hom_A(-,A)$ je kontravariantním zleva exaktním funktorem 
       abelovských grup.
       \begin{description}
         \item[(a)] Buď $M\in mod(A)$, pak $Hom_A(M,A)\in mod(R)$. Definujme na 
         $Hom_A(M,A)$ násobení prvky $A$ zprava předpisem 
         \begin{eqnarray} 
           Hom_A(M,A)\times A&\to& Hom_A(M,A)\nonumber \\
            f\lambda &\mapsto& [f\lambda(m) \mapsto f(m)\lambda]. \nonumber 
         \end{eqnarray}          
         Že je násobení dobře definované a asociativní ke sčítání homomorfismů je zřejmé.
         Ověříme, že $()^*$ přenaší homomorfismus $A$-modulů na homomorfismus 
         $A^{op}$-modulů. Nechť $M,M'\in mod(A)$, $h\in Hom_A(M,M')$, $f'\in M^*'$ 
         a $\lambda\in A$. Pak
         \begin{eqnarray}
           (h^*(f'\lambda))(m)&=& ((f'\lambda)h)(m)\nonumber \\
           &=& (f'\lambda)(h(m)) \nonumber \\
           &=& f'(h(m))\lambda \nonumber \\
           &=& (f'h)(m)\lambda\nonumber \\
           &=& ((f'h)\lambda)(m)\nonumber \\
           &=& (h^*(f')\lambda)(m)\nonumber 
         \end{eqnarray}
         pro všechna $m\in M$ a tedy \\\\
         \centerline{$h^*(f'\lambda)=h^*(f')\lambda$.}\\\\
         Obdobně se dokáže rovnost \\\\
         \centerline{$h^*(f'_1 + f'_2)=h^*(f'_1 + f'_2)$.}\\\\
         Pak $Hom_A(M,A)\in Mod(A^{op})\cap mod(R)$ a tedy dle 
           \hyperref[lem-mod-Mod]{Lemma \ref*{lem-mod-Mod}} máme $Hom_A(M,A)\in 
           mod(A)$.
           
         \item[(b)] Buď $P$ projektivní $A$-modul. Pak dle 
         \hyperref[rozklad-A-na-P]{Věty \ref*{rozklad-A-na-P}} 
         existuje projektivní $A$-modul  $P'$ a 
         $n\in \mathbb N$ takové, že $A^n=P\oplus P'$. Potom máme následující izomorfismus 
         $A^{op}$-modulů: \\\\
         \centerline{$ Hom_A(A^n,A)\simeq Hom_A(P\oplus P', A)$.} \\\\
         Navíc dle 
         \hyperref[dir-sum-hom]{Věty \ref*{dir-sum-hom}} následující izomorfismy 
         $A^{op}$-modulů:\\\\
         \centerline{$ Hom_A(P, A) \oplus Hom_A(P', A) \simeq Hom_A(A, A)$.} \\\\
         \centerline{$ Hom_A(A^n, A) \simeq Hom_A(A, A)^n$.} \\\\
         Pak s využitím \hyperref[izo-hom-aei]{Věty \ref*{izo-hom-aei}} 
         dostáváme izomorfismus  $A^{op}$-modulů\\\\
         \centerline{$ Hom_A(P, A) \oplus Hom_A(P', A) \simeq Hom_A(A^n, A) \simeq (A^{op})^n$.} \\\\
         Z toho již plyne, že $P^*=Hom_A(P,A)$ je direktním sčítancem $(A^{op})^n$ 
         a tedy projektivní.
       \end{description}
     \end{proof}
     
   \subsection{Funktor $Tr$}
    
     \begin{lem}\label{lemma-cok*}
       Nechť $X\in mod(A)$ a posloupnost  \\\\
       \centerline{\xymatrix{
       P_1 \ar@{->}[r]^s 
         & P_0 \ar@{->}[r]^t 
         & X \ar@{->}[r]
         & 0
      },}\\\\ je minimální projektivní prezentace modulu $X$. Pak
       \begin{description}
         \item[(a)] Následující posloupnost je exaktní v $mod(A^{op})$: \\\\
         \centerline{\xymatrix{
           0 \ar@{->}[r] 
             & X^* \ar@{->}[r]^{t^*} 
             & P_0^* \ar@{->}[r]^{s^*} 
             & P_1^* \ar@{->}[r] 
             & Cok(s^*) \ar@{->}[r]
             & 0
          }}       
         \item[(b)] Je-li navíc  \\
         \centerline{\xymatrix{
           P_1' \ar@{->}[r]^{s'} 
             & P_0' \ar@{->}[r]^{t'} 
             & X \ar@{->}[r]
             & 0
          }} \\\\
          druhá projektivní prezentace modulu $X$, pak máme izomorfismus 
          $A^{op}$-modulů: \\\\
          \centerline{$Cok(s^*)\simeq Cok(s'^*)$}
       \end{description}
     \end{lem}
     \begin{proof}
       \begin{description}
         \item
         \item[(a)] Plyne z \hyperref[lem-star]{Lemma \ref*{lem-star}} a z toho, 
           že kojádro morfismu konečně generovaných modulů je konečně 
           generované.
         \item[(b)] Dle \hyperref[lem-proj-pokryti]{Lemma \ref*{lem-proj-pokryti}}
           existují izomorfismy $h_0\in Hom_A(P_0,P_0')$ a $h_1\in Hom_A(P_1,P_1')$ 
           takové, že následující diagram komutuje: \\\\
           \centerline{\xymatrix{
             0 \ar@{->}[r] 
               & P_1 \ar@{->}[r]^s \ar@{.>}[d]_{h_1} 
               & P_0 \ar@{->}[r]^t \ar@{.>}[d]_{h_0} 
               & X \ar@{->}[r] \ar@{=}[d] 
               & 0 \\
             0 \ar@{->}[r] 
               & P_1' \ar@{->}[r]^{s'} 
               & P_0' \ar@{->}[r]^{t'} 
               & X \ar@{->}[r]
               & 0
           }}\\\\\\       
           Aplikujeme-li funktor $()^*$ dostaneme následující komutativní 
           diagram:\\\\
           \centerline{\xymatrix{
             0 \ar@{->}[r]
               & X^{*} \ar@{->}[r]^{t^{'*}} \ar@{=}[d]
               & P_0^{'*} \ar@{->}[r]^{s^{'*}} \ar@{->}[d]^{h_0^*}
               & P_1^{'*} \ar@{->}[r]  \ar@{->}[d]^{h_1^*}
               & Cok(s^{'*}) \ar@{->}[r]  \ar@{.>}[d]^{\simeq}
               & 0 \\
             0 \ar@{->}[r]
               & X^{*} \ar@{->}[r]^{t^{*}} 
               & P_0^{*} \ar@{->}[r]^{s^{*}} 
               & P_1^{*} \ar@{->}[r] 
               & Cok(s^{*}) \ar@{->}[r]
               & 0
           }}\\\\\\       
           Z \hyperref[lemma-five]{Lemma \ref*{lemma-five}} pak plyne hledaný $S^{op}$-modulový 
           izomorfismus $Cok(s^{*})\simeq Cok(s^{*'})$.
         \end{description} 
     \end{proof}
     
     \begin{dfn}
       Na základě předchozího \hyperref[lemma-cok*]{Lemma \ref*{lemma-cok*}} 
       definujme zobrazení \\\\  
       \centerline{$Tr:mod(A)\rightarrow mod(A^{op})$} \\\\
       předpisem $Tr(X):=Cok(s^*)$, kde \\\\
       \centerline{\xymatrix{
         P_1 \ar@{->}[r]^s 
           & P_0 \ar@{->}[r]^t 
           & X \ar@{->}[r]
           & 0
       },}\\\\
       je libovolná projektivní prezentace modulu $X$.       
     \end{dfn}
     
     \begin{dfn}
       \begin{description} \item
         \item[(a)] Pro $M,N\in mod(A)$ položme: 
           \\\\
           $P_A(M,N):=\{f\in Hom_A(M,N)|f$ se faktorizuje skrze projektivní modul$\}$ \\
           \\
          \centerline{$\underline{Hom}_A(M,N):=Hom_A(M,N)/P_A(M,N)$} \\
         \item[(b)] Definujme kategorii $\underline{mod}(A)$: \\\\
           \centerline{$Ob(\underline{mod}(A)):=Ob(mod(A))$}\\\\
           \centerline{$Hom_{\underline{mod}(A)}(M,N):=\underline{Hom}_A(M,N)$}\\ 
       \end{description}       
     \end{dfn}
     
     \begin{thm}
       Pro dva moduly $X,X'\in mod(A)$ a jejich dvě projektivní prezentace 
       \\\\      
       \centerline{\xymatrix{
         P_1 \ar@{->}[r]^s 
           & P_0 \ar@{->}[r]^t 
           & X \ar@{->}[r]
           & 0
       }}\\\\       
       \centerline{\xymatrix{
         P_1' \ar@{->}[r]^{s'} 
           & P_0' \ar@{->}[r]^{t'} 
           & X' \ar@{->}[r]
           & 0
       },}\\\\\\
       dodefinujme $Tr:\underline{Hom}_A(X,X')\rightarrow 
       \underline{Hom}_{A^{op}}(Tr(X'),Tr(X))$ předpisem \\\\
       \centerline{$h+P_A(X,X')\mapsto (h_1^*)_{Cok}+P_{A^{op}}(Tr(X'),Tr(X))$,}
       \\\\kde $h_0\in Hom_A(P_0,P_0')$ a $h_1\in Hom_A(P_1,P_1')$ jsou 
       libovolně zvolené homomorfismy takové, že následující diagram komutuje: 
       \\\\
           \centerline{\xymatrix{
             X^{'*} \ar@{->}[r]^{t^{'*}} \ar@{->}[d]^{h^*}
               & P_0^{'*} \ar@{->}[r]^{s^{'*}} \ar@{->}[d]^{h_0^*}
               & P_1^{'*} \ar@{->}[r]  \ar@{->}[d]^{h_1^*}
               &Tr(X') \ar@{->}[r]  \ar@{->}[d]^{(h_1^*)_{Cok}}
               & 0 \\
             X^{*} \ar@{->}[r]^{t^{*}} 
               & P_0^{*} \ar@{->}[r]^{s^{*}} 
               & P_1^{*} \ar@{->}[r] 
               & Tr(X)\ar@{->}[r]
               & 0
           },}\\\\\\     
       Pak je $Tr$ kontravariantním funktorem $Tr: \underline{mod}(A)\rightarrow 
       \underline{mod}(A^{op})$. 
     \end{thm}
     \begin{proof}
       Důkaz rozdělíme na 3 části:
       \begin{description}
         \item[(1)] Pro morfismy $g_0,h_0\in Hom_A(P_1,P_1')$ a $g_0,h_0\in Hom_A(P_0,P_0')$ 
         takové, že následující diagramy komutují, \\\\
         \centerline{\xymatrix{
         P_1 \ar@{->}[r]^s  \ar[d]^{h_1}
           & P_0 \ar@{->}[r]^t   \ar[d]^{h_0}
           & X \ar@{->}[r] \ar[d]^{h}
           & 0 \\
         P_1' \ar@{->}[r]^{s'} 
           & P_0' \ar@{->}[r]^{t'} 
           & X' \ar@{->}[r]
           & 0
       } \\ \xymatrix{
         P_1 \ar@{->}[r]^s  \ar[d]^{g_1}
           & P_0 \ar@{->}[r]^t   \ar[d]^{g_0}
           & X \ar@{->}[r] \ar[d]^{h}
           & 0 \\
         P_1' \ar@{->}[r]^{s'} 
           & P_0' \ar@{->}[r]^{t'} 
           & X' \ar@{->}[r]
           & 0
       } \\}\\\\\\
       platí, že \\\\
       \centerline{$(h_1^*)_{Cok}-(g_1^*)_{Cok}\in P_{A^{op}}(Tr(X'),Tr(X))$.} 
       \\\\
       Neboli $(h_1^*)_{Cok}$ a $(g_1^*)_{Cok}$ reprezentují stejný prvek z 
       $\underline{Hom}_{A^{op}}(Tr'(X),Tr(X))$.
       
         \item[(2)] Pro $h\in P_A(X,X')$ je \\\\
         \centerline{$(h_1^*)_{Cok}\in P_{A^{op}}(Tr(X'),Tr(X))$.}\\\\
         Neboli prvek $\underline{Hom}_{A}(X',X)$ se zobrazuje na prvek 
         $\underline{Hom}_{A^{op}}(Tr'(X),Tr(X))$
         nezávisle na zvoleném reprezentantu.

         \item[(3)] Funktor $Tr$ kontravariantním funktorem $\underline{mod}(A)\rightarrow 
       \underline{mod}(A^{op})$.
       \end{description}
       
       
       \begin{description}
         \item[(1)] Uvažujme následující diagram: \\\\
         \centerline{\xymatrix{
         P_1 \ar@{->}[rr]^s  \ar@/^1pc/[dd]^{h_1} \ar@/_1pc/[dd]_{g_1}
           && P_0 \ar@{->}[rr]^t   \ar@/_1pc/[dd]_{g_0}  \ar@/^1pc/[dd]^{h_0}  \ar@{.>}[lldd]_u
           && X \ar@{->}[rr] \ar[dd]^{h}
           && 0 \\\\
         P_1' \ar@{->}[rr]^{s'} \ar[rd]
           && P_0' \ar@{->}[rr]^{t'} 
           && X' \ar@{->}[rr]
           && 0 \\
           & Ket(t') \ar[ru]
         }\\}\\\\\\
         Protože $ht=t'h_0=t'g_0$, pak $t'(h_0-g_0)=0$ a tedy $h_0-g_0$ se 
         faktorizuje skrze $Ker(t')$. Pak z projektivity $P_0$ existuje 
         kanonická projekce $P_1'$ na $Im(s')=Ker(t')$ a $(g_0-h_0)$ se 
         faktorizuje skrze $P_1'$. Proto existuje $u\in Hom_A(P_0,P_1')$ takové, 
         že \\\\
         \centerline{$s'u=g_0-h_0$.}\\\\
         Aplikujeme-li funktor $()^*$, dostaneme následující diagram v 
         $mod(A^{op})$:\\\\
         \centerline{\xymatrix{
         P_0'^* \ar@{->}[rr]^{s'^*}  \ar@/^1pc/[dd]^{h_0'^*} \ar@/_1pc/[dd]_{g_0'^*}
           && P_1'^* \ar@{->}[rr]^{\hat t'}   \ar@/_1pc/[dd]_{g_1'^*}  \ar@/^1pc/[dd]^{h_1'^*}  
           \ar[lldd]_{u^*}
           && Tr(X') \ar@{->}[rr] \ar@/^1pc/[dd]^{(h_1^*)_{Cok}} \ar@/_1pc/[dd]_{(g_1^*)_{Cok}}
           \ar@{.>}[lldd]_{v}
           && 0 \\\\
         P_0^* \ar@{->}[rr]^{s^*}
           && P_0^* \ar@{->}[rr]^{\hat t} 
           && X' \ar@{->}[rr]
           && 0 \\
           & \quad\quad &
           & \quad\quad &
         }\\}\\
         Protože platí 
         \\\\\centerline{$u^*s'^*=g_0^*-h_0^*$,} \\\\
         pak 
         \\\\\centerline{$s^*u^*s'^*=s^*(g_0^*-h_0^*)=(g_1^*-h_1^*)s'^*$}\\\\
          a tedy
         \\\\\centerline{$(g_1^*-h_1^*-s^*u^*)s'^*=0$.}\\\\
         Pak se $(g_1^*-h_1^*-s^*u^*)$ faktorizuje skrze kojádro $s'^*$, 
         konkrétně $\hat t'$. To znamená, že existuje $u\in Hom_{mod(A^{op})}(Tr(X'),P_1^*)$ 
         takové, že 
         \\\\\centerline{$v \hat t'=g_1^*-h_1^*-s^*u^*$.}\\\\
         Přenásobíme-li poslední řádek $\hat t$, dostaneme
         \\\\\centerline{$\hat t v \hat t'=\hat t (g_1^*-h_1^*-s^*u^*)=((g_1^*)_{Cok} - (h_1^*)_{Cok})\hat t '$,}\\\\
         a protože $\hat t'$ je epimorfismus, pak 
         \\\\\centerline{$\hat t v=(g_1^*)_{Cok}-(h_1^*)_{Cok}$.}\\\\
         A tedy $(g_1^*)_{Cok}-(h_1^*)_{Cok}$ se faktorizuje skrze $P_1^*$ a 
         \\\\\centerline{$(g_1^*)_{Cok}-(h_1^*)_{Cok} \in P_{A^{op}}(Tr(X'),Tr(X))$.}
                  
         \item[(2)] Uvažujme diagram:\\\\
         \centerline{\xymatrix{
         P_1 \ar@{->}[rr]^s   \ar[dd]_{h_1}
           && P_0 \ar@{->}[rr]^t  \ar[dd]^{h_0} \ar@{.>}[lldd]_v  \ar@{.>}[lddd]_v
           && X \ar@{->}[rr] \ar[dd]^{h}  \ar@{.>}[lldd]_u
           && 0 \\\\
         P_1' \ar@{->}[rr]^{s'} \ar[rd]
           && P_0' \ar@{->}[rr]^{t'} 
           && X' \ar@{->}[rr]
           && 0 \\
           & Ket(t') \ar[ru]
         }\\}\\\\\\
         Předpokládejme, že se $h$ faktorizuje skrze projektivní $A$-modul $P$. 
         Pak protože $t'$ je epimorfismus, tak se $h$ faktorizuje i skrze $t'$. 
         Neboli existuje $u'\in Hom_A(X,P_0)$ takové, že 
         \\\\\centerline{$t'u=h$.}\\\\
         Uvažujme $A$-homomorfismus $(h_0-ut)$, ten se faktorizuje skrze 
         $Ket(t')$. Přenásobením $t'$ zleva 
         dostaneme rovnost
         \\\\\centerline{$t'(h_0-ut)=t'h_0-t'ut=t'h_0-ht=0$.}\\\\
         Pak protože $P_0$ je projektivní modul a z existence kanonické 
         projekce $P_1'$ na $Im(s')=Ket(t')$, se musí $h_0-ut$ faktorizovat 
         skrze $P_1'$. Tedy existuje $v\in Hom_A(P_0,P_1')$ takové, že
         \\\\\centerline{$s'v=h_0-ut$.}\\\\
         Aplikací funktoru $()^*$ získáme rovnost
         \\\\\centerline{$v^*s'^*=h_0^*-t^*u^*$}\\\\
         a následující diagram komutuje:\\\\
         \centerline{\xymatrix{
          0 \ar[r]
           & X'^* \ar@{->}[rr]^{t'^*}  \ar[dd]^{h^*}
           && P_0'^* \ar@{->}[rr]^{s'^*}  \ar[dd]^{h_0^*} \ar@{->}[lldd]_{u^*}
           && P_1'^* \ar@{->}[rr]^{\hat t'} \ar[dd]^{h_1^*}  \ar@{->}[lldd]_{v^*}
           && Tr(X')  \ar@{->}[r] \ar[dd]^{(h_1^*)_{Cok}}  \ar@{.>}[lldd]_{w}
           & 0 \\\\
         0 \ar[r]
           & X^* \ar@{->}[rr]^{t^*}
           && P_0^* \ar@{->}[rr]^{s^*} 
           && P_1 \ar@{->}[rr]^{\hat t}           
           && Tr(X) \ar@{->}[r]
           & 0
         }\\}\\\\\\
         Ukážeme, že se $(h_1^*-s^*v^*)$ faktorizuje skrze $Cok(s'^*)=Tr(X'):$
         \begin{eqnarray}
           (h_1^*-s^*v^*)s'^* &=& h^*s'^*-s^*v^*s'^*   \nonumber \\
           &=&  h^*s'^*-s^*(h_0^*-t^*u^*)  \nonumber \\
           &=&  h^*s'^*-s^* h_0^*-s^*t^*u^*    \nonumber \\
           &=& 0.   \nonumber 
         \end{eqnarray}
         Pak existuje $w\in Hom_A(Tr(X'),P_1^*)$ takové, že
         \\\\\centerline{$w\hat t'=h_1^*-s^*v^*$,}\\\\
         a my konečně vidíme, že
         \\\\\centerline{$\hat w\hat t'=\hat t(h_1^*-s^*v^*)=\hat t h_1^*=(h_1^*)_{Cok}\hat t'$.}\\\\
         A protože $\hat t'$ je epimorfismus, musí být
         \\\\\centerline{$(h_1^*)_{Cok}=\hat t w$.}\\\\
         Tedy $(h_1^*)_{Cok}\in P_{A_{op}}(Tr(X'),Tr(X))$.
         \item[(3)] Již víme, že zobrazení
           \\\\\centerline{$Tr:Ob(\underline{mod}(A))\to Ob(\underline{mod}(A^{op}))$}\\\\
           je dobře definované a z (1) a (2) navíc, že pro $X,X'\in \underline{mod}(A)$ 
           je dobře definované i zobrazení
           \\\\\centerline{$Tr: \underline{Hom}_A(X,X') \to \underline{Hom}_{A^{op}}(Tr(X'),Tr(X))$.}\\\\
           Zbývá tedy dokázat, že $Tr$ je kompatibilní se skládáním morfismů a 
           že zachovává identitu. Nechť tedy $X,Y,Z\in \underline{mod}(A)$, $f\in Hom_A(X,Y)$ 
           a $g\in Hom_A(Y,Z)$. Chceme ukázat, že 
           \\\\\centerline{$Tr(gf)=Tr(f)Tr(g)$.}\\\\
           Máme následující  komutativní diagram v $mod(A)$, kde jednotlivé 
           řádky jsou minimální  projektivní prezentace modulů $X$, $Y$ a $Z$:
           \\\\\centerline{$\xymatrix{
                P_{X,1} \ar[r]^{s_X} \ar[d]^{f_1}
             & P_{X,0} \ar[r]^{t_X} \ar[d]^{f_0}
             & X \ar[r]  \ar[d]^{f}
             & 0 \\
                P_{Y,1} \ar[r]^{s_Y} \ar[d]^{g_1}
             & P_{Y,0} \ar[r]^{t_Y} \ar[d]^{g_0}
             & Y \ar[r]  \ar[d]^{g}
             & 0 \\
                P_{Z,1} \ar[r]^{s_Z} 
             & P_{Z,0} \ar[r]^{t_Z}
             & Z \ar[r]
             & 0
           }$} \\\\\\
           Dle bodu (1) si můžeme při hledání $Tr(gf)$ zvolit libovolné dva 
           homomorfismy $(gf)_0$ a $(gf)_1$ takové, že následující diagram 
           komutuje:
           \\\\\centerline{$\xymatrix{
                P_{X,1} \ar[r]^{s_X} \ar[d]^{(gf)_1}
             & P_{X,0} \ar[r]^{t_X} \ar[d]^{(gf)_0}
             & X \ar[r]  \ar[d]^{gf}
             & 0 \\
                P_{Z,1} \ar[r]^{s_Z} 
             & P_{Z,0} \ar[r]^{t_Z}
             & Z \ar[r]
             & 0
           }$} \\\\\\
           Tuto podmínku jasně splňuje i volba:
           \begin{eqnarray}
            (gf)_0 &:=& g_0f_0 \nonumber \\
            (gf)_1 &:=& g_1f_1.\nonumber
           \end{eqnarray}
           Aplikací funktoru $()^*$ dostaneme následující diagram v $mod(A^{op})$, 
           na který je možné zároveň nahlížet jako na diagram v 
           $\underline{mod}(A^{op})$, kde jednotlivé homomorfismy jsou zástupci 
           svých tříd ekvivalence:
           \\\\\centerline{$\xymatrix{
                P_{Z,0}^* \ar[r]^{s_Z^*} \ar[d]^{g_0^*}
             & P_{Z,1}^* \ar[r]^{\hat t_Z} \ar[d]^{g_1^*}
             & Tr(Z) \ar[r]  \ar[d]^{Tr(g)} \ar@/^6pc/[dd]^{Tr(gf)}
             & 0 \\
                P_{Y,0}^* \ar[r]^{s_Y^*} \ar[d]^{f_0^*}
             & P_{Y,1}^* \ar[r]^{\hat t_Y} \ar[d]^{f_1^*}
             & Tr(Y) \ar[r]  \ar[d]^{Tr(f)}
             & 0 \\
                P_{X,0}^* \ar[r]^{s_X^*} 
             & P_{X,1}^* \ar[r]^{\hat t_X}
             & Tr(X) \ar[r]
             & 0
           }$} \\\\\\
          V tomto diagramu vidíme následující rovnost
           \\\\\centerline{$
             Tr(gf)\hat t_Z
             =\hat t_X f_1^*g_1^*
             =\hat t_X f_1^*g_1^*
             = Tr(f) \hat t_Y g_1^*
             = Tr(f)Tr(g)\hat t_Z$}\\\\
           a tedy, protože $\hat t_Z$ je epimorfismus, máme
           \\\\\centerline{$Tr(gf)=Tr(f)Tr(g)$.}\\\\
           Nyní zbývá ukázat, že
           \\\\\centerline{$Tr(1_X)=1_{Tr(X)}$,}\\\\
           pro každé $X\in\underline{mod}(A)$. Uvažme diagram
           \\\\\centerline{$\xymatrix{
                P_{1} \ar[r]^{s} \ar[d]^{1_{P_0}}
             & P_{0} \ar[r]^{t} \ar[d]^{1_{P_1}}
             & X \ar[r]  \ar[d]^{1_X}
             & 0 \\
                P_{1} \ar[r]^{s} 
             & P_{0} \ar[r]^{t}
             & X \ar[r]
             & 0
           }$} \\\\\\
           Aplikujeme-li funktor $()^*$ dostaneme
           \\\\\centerline{$Tr(1_X)=(1_{P_1}^*)_{Cok}=1_{Tr(X).}$}\\\\
       \end{description}
     \end{proof}
     
     \begin{pzn}
       Zmiňme bez důkazu ještě několik užitečných vlastností funktoru $Tr$, 
       jejichž důkaz a věškeré podrobnosti lze nalézt hned v několika zdrojích, 
       například \cite{2}:
       \begin{description}
         \item[(a)] $Tr^2=1_{\underline{mod}(A)}$.
         \item[(b)] $Tr(\bigoplus_{i=1}^n M_i)=\bigoplus_{i=1}^n Tr(M_i)$. 
         \item[(c)] $Tr(M)=0$ $\Leftrightarrow$ $M$ je projektivní. 
         \item[(d)] $Tr(M)\simeq$ neprojektivní části $M$.
       \end{description}
     \end{pzn}
     
   \subsection{Funktory $\delta^*$ a $\delta_*$} 
     
     \begin{dfn}\label{def-delta-*}
       Nechť $X\in mod(A)$ a $\delta$ značí následující exaktní posloupnost \\\\
       \centerline{\xymatrix{
         0 \ar@{->}[r] 
           & M \ar@{->}[r]^f 
           & N \ar@{->}[r]^g 
           & L \ar@{->}[r]
           & 0
       }.}\\
       \begin{description}
         \item[(a)] Definujme $\delta_*(X)$ exaktností následující posloupnosti $R$-modulů: \\\\
           \centerline{\xymatrix{
             0 \ar@{->}[r] 
               & Hom_A(L,X) \ar@{->}[r]^{(-\circ g)_X}
               & Hom_A(N,X) \ar@{->}[r]^{(-\circ f)_X}
               & Hom_A(M,X) \ar@{->}[r]
               & \delta_*(X) \ar@{->}[r]
               & 0 &\,&             
             }}\\ 
         \item[(b)] Definujme $\delta^*(X)$ exaktností následující posloupnosti $R$-modulů: \\\\
           \centerline{\xymatrix{
             0 \ar@{->}[r] 
               & Hom_A(X,M) \ar@{->}[r]^{(f\circ -)_X}
               & Hom_A(X,N) \ar@{->}[r]^{(g\circ -)_X}
               & Hom_A(X,L) \ar@{->}[r]
               & \delta^*(X) \ar@{->}[r]
               & 0 &\,&
           }}  
       \end{description}
     \end{dfn}
     
     \begin{thm}\label{thm-delta}
       Nechť $X,X'\in mod(A)$ a $h\in Hom_A(X,X')$ a $\delta$ je jako v definici 
       \hyperref[def-delta-*]{Definici \ref*{def-delta-*}}. Pak platí:
       \begin{description}
         \item[(a)] Položme $\delta_*(h):=((h \circ -)_M)_{Cok}$ jako na následujícím diagramu:  \\\\
           \centerline{\xymatrix{
             0 \ar@{->}[r] 
               & Hom_A(L,X) \ar@{->}[r]^{(-\circ g)_X} \ar@{->}[d]^{(h\circ -)_L}
               & Hom_A(N,X) \ar@{->}[r]^{(-\circ f)_X} \ar@{->}[d]^{(h\circ -)_N}
               & Hom_A(M,X) \ar@{->}[r] \ar@{->}[d]^{(h\circ -)_M}
               & \delta_*(X) \ar@{->}[r] \ar@{->}[d]^{((h \circ -)_M)_{Cok}}
               & 0\\
             0 \ar@{->}[r] 
               & Hom_A(L,X') \ar@{->}[r]^{(-\circ g)_X'}
               & Hom_A(N,X') \ar@{->}[r]^{(-\circ f)_X'}
               & Hom_A(M,X') \ar@{->}[r]
               & \delta_*(X') \ar@{->}[r]
               & 0 &\,&\\
           }}\\\\\\
           Spolu s tímto zobrazením je $\delta_*$ kovariantní funktor $mod(A)\rightarrow mod(R)$.

         \item[(b)] Položme $\delta^*(h):=((-\circ h)_L)_{Cok}$ jako na následujícím diagramu: \\\\
          \centerline{\xymatrix{
             0 \ar@{->}[r] 
               & Hom_A(X',M) \ar@{->}[r]^{(f\circ -)_X'} \ar@{->}[d]^{(-\circ h)_M}
               & Hom_A(X',N) \ar@{->}[r]^{(g\circ -)_X'} \ar@{->}[d]^{(-\circ h)_N}
               & Hom_A(X',L) \ar@{->}[r] \ar@{->}[d]^{(-\circ h)_L}
               & \delta^*(X') \ar@{->}[r] \ar@{->}[d]^{((-\circ h)_L)_{Cok}}
               & 0 \\
             0 \ar@{->}[r] 
               & Hom_A(X,M) \ar@{->}[r]^{(f\circ -)_X}
               & Hom_A(X,N) \ar@{->}[r]^{(g\circ -)_X}
               & Hom_A(X,L) \ar@{->}[r]
               & \delta^*(X) \ar@{->}[r]
               & 0 &\,&
           }}\\\\\\
           Spolu s tímto zobrazením je $\delta^*$ kontravariantní funktor $mod(A)\rightarrow mod(R)$.
       \end{description}
     \end{thm}
     
     \begin{proof}
       Dokážeme pouze (a), část (b) se dokáže analogicky. Máme 
       $Hom_A(N,X)$, $Hom_A(M,X)\in mod(R)$ a tedy i $\delta_*(X)\in mod(R)$ 
       jakožto kojádro homomorfismu konečně generovaných $R$-modulů. 
       
       Hom 
       funktory nám zobrazují $A$-homomorfismy na $R$-homomorfismy a stejně tak 
       funktor $\delta_*$, jelikož je definovaný jako kojádro zobrazení 
       $R$-homomorfismů. Dále je zřejmé, že $\delta_*(1_X)=1_{\delta_*(X)}$.
     
       Zvolme si pevně $X,X',X''\in mod(A)$, $h\in Hom_A(X,X')$ a $h'\in 
       Hom_A(X',X'')$. Dokážeme, že $\delta_*(h'h)=\delta_*(h')\delta_*(h)$. 
       Máme následující komutativní diagram: \\\\
       \centerline{
           \xymatrix{
             Hom_A(M,X) \ar@{->}[rr]^{\pi} \ar@{.>}[d]^{(h\circ -)_M} \ar@/_5pc/[dd]_{(h'h\circ-)_M}
               & & \delta_*(X) \ar@{.>}[d]_{\delta_*(h)} \ar@/^5pc/[dd]^{\delta_*(h'h)} \\
             Hom_A(M,X') \ar@{->}[rr]^{\pi'} \ar@{.>}[d]^{(h'\circ -)_M}
               & & \delta_*(X') \ar@{.>}[d]_{\delta_*(h')} \\   
             Hom_A(M,X') \ar@{->}[rr]^{\pi''}
               & & \delta_*(X') \\           
           }}\\\\\\
       Pro každé $u\in Hom_A(M,X)$ platí
       \begin{eqnarray}
         (h'h\circ-)_M(u)
         &=& h'hu \nonumber \\
         &=& (h'\circ-)_M(hu) \nonumber \\
         &=& (h'\circ-)_M(h\circ-)_M(u), \nonumber
       \end{eqnarray}
       pak $(h'h\circ-)_M=(h'\circ-)_M(h\circ-)_M$. Z komutativity diagramu 
       plyne 
       \begin{eqnarray}
         \delta_*(h'h)\pi
         &=& \pi''(h'h\circ-)_M \nonumber \\
         &=& \pi''(h'\circ-)_M(h\circ-)_M  \nonumber \\
         &=& \delta_*(h')\pi'(h\circ-)_M \nonumber \\
         &=& \delta_*(h')\delta_*(h)\pi,  \nonumber 
       \end{eqnarray}
       a protože $\pi$ je epimorfismus, tak nám tato rovnost implikuje \\\\
       \centerline{ $\delta_*(h'h)=\delta_*(h')\delta_*(h)$.}
     \end{proof}
     
     \begin{dfn}
       Funktor \\
       \centerline{$\delta_*:mod(A)\rightarrow mod(R)$}\\\\
       se nazývá kovariantní defekt funktor a funktor  \\\\
       \centerline{$\delta^*:mod(A)\rightarrow mod(R)$} \\\\ 
       se nazývá kontravariantní defekt funktor.
     \end{dfn}
     
  \subsection{Skoro štěpitelné posloupnosti}
     
     \begin{lem}\label{lem-almost-split-def}
       Pro následující exaktní posloupnost\\\\
       \centerline{\xymatrix{
         0 \ar@{->}[r] 
           & M \ar@{->}[r]^f 
           & N \ar@{->}[r]^g 
           & L \ar@{->}[r]
           & 0
       }}\\\\ 
       v $mod-A$ jsou následující tvrzení ekvivalentní:
       \begin{description}
         \item[(a)] Existuje $f'\in Hom_A(N,M)$ takové, že $f'f=1_M$.
         \item[(b)] Existuje $g'\in Hom_A(L,N)$ takové, že $gg'=1_L$.  
       \end{description}        
     \end{lem}
     \begin{proof}
       Dokážeme, že z (a) plyne (b). Opačná implikace je analogická. Nechť tedy $f'\in Hom_A(N,M)$ 
       je takový, že $f'f=1_M$. Uvažujme následující komutativní diagram v 
       $mod(A)$:\\\\
       \centerline{\xymatrix{
         0 \ar@{->}[r] 
           & M \ar@{->}[r]^f \ar@{=}[d] 
           & N \ar@{->}[r]^g \ar@{=}[d] \ar@{->}[ld]_{f'}
           & L \ar@{->}[r]      \ar@{.>}[d]^{1_C} \ar@{.>}[ld]_{g'}
           & 0 \\
         0 \ar@{->}[r] 
           & M \ar@{->}[r]^f 
           & N \ar@{->}[r]^g 
           & L \ar@{->}[r]
           & 0
       }}\\\\\\
       Položme $h:=1_N-ff'$, pak \\\\
       \centerline{$hf=f-(ff')f=f-f(f'f)=0$,} \\\\
       neboli $h$ se faktorizuje skrze $Cok(f)=L$. Což znamená, že existuje homomorfismus $g'\in Hom_A(L,N)$ 
       takový, že \\\\
       \centerline{$g'g=h$.} \\\\
       Pak $(gg')g=g(g'g)=gh=g1_B-g(ff')=g-(gf)f'=1_Lg$, a protože $g$ je 
       epimorfismus, musí být \\\\
       \centerline{$gg'=1_C$.}
     \end{proof}
     
     \begin{dfn}
       \begin{description} \item
         \item[(a)] Nechť $M,N\in mod(A)$ a $f\in Hom_A(M,N)$ je epimorfismus. 
         Řekneme, že $f$ je štěpitelný epimorfismus, pokud existuje $f'\in mod_A(N,M)$
         takový, že \\\\ 
         \centerline{$ff'=1_N$.}
         \item[(b)] Nechť $\delta$ je následující exaktní posloupnost v $mod(A)$\\\\
           \centerline{\xymatrix{
             0 \ar@{->}[r] 
               & M \ar@{->}[r]^f 
               & N \ar@{->}[r]^g 
               & L \ar@{->}[r]
               & 0,
               &
               &
           }}\\\\
           pak řekneme, že $\delta$ je štěpitelná posloupnost, pokud 
             splňuje jednu z ekvivalentních podmínek z 
             \hyperref[lem-almost-split-def]{Lemma \ref*{lem-almost-split-def}}.      
       \end{description}         
     \end{dfn}
     
     \begin{dfn}
         Nechť $\delta$ je následující exaktní posloupnost v $mod(A)$\\\\
           \centerline{\xymatrix{
             0 \ar@{->}[r] 
               & M \ar@{->}[r]^f 
               & N \ar@{->}[r]^g 
               & L \ar@{->}[r]
               & 0,
               &
               &
           }}\\\\
           pak řekneme, že $\delta$ je skoro štěpitelná posloupnost, pokud 
         splňuje následující dvě podmínky:                
         \begin{description}  
           \item[($g$ je zprava minimální)] Pokud $h\in End_A(M)$ a $gh=g$, pak $g$ je izomorfismus.
           \item[($g$ je zprava skoro štěpitelný)]  Homomorfismus $g$ není štěpitelný epimorfismus a pro 
             každé $Y\in mod(A)$ a $h\in 
             Hom_A(Y,L)$, které není štěpitelný epimorfismus, existuje $u\in Hom_A(Y,N)$ 
             takové, že $h=gu$.\\\\
             \centerline{\xymatrix{
               & & & Y \ar@{->}[d]^h \ar@{-->}[ld]_u \\             
               0 \ar@{->}[r] 
                 & M \ar@{->}[r]^f 
                 & N \ar@{->}[r]^g 
                 & L \ar@{->}[r]
                 & 0
                 &
                 &
                 &
             }}\\
       \end{description}         
     \end{dfn}     
     
     Poznamenejme, že \cite{2} Proposition 1.14 (Kapitola V) popisuje řadu 
     ekvivalentních definic skoro štěpitelné posloupnosti.
     
     \begin{thm}
       \begin{description} \item
         \item[(a)] Všechny skoro štěpitelné posloupnosti v $mod(A)$ jsou tvaru 
           \\\\
           \centerline{\xymatrix{
             0 \ar@{->}[r] 
               & DTr(X) \ar@{->}[r] 
               & E \ar@{->}[r] 
               & X \ar@{->}[r]
               & 0
               &
           }}\\\\
           kde $E\in mod(A)$ a $X\in mod(A)$ je nerozložitelý a neprojektivní.
         \item[(b)] Pro každý $X\in mod(A)$ nerozložitelný a neprojektivní modul
           existuje skoro štěpitelná posloupnost 
           \\\\
           \centerline{\xymatrix{
             0 \ar@{->}[r] 
               & DTr(X) \ar@{->}[r] 
               & E \ar@{->}[r] 
               & X \ar@{->}[r]
               & 0
               & 
           }} \\\\
           v $mod(A)$.
       \end{description}      
     \end{thm}
     \begin{proof}
        \begin{description}
           \item
           \item[(a)]
           Pokud je \\\\
             \centerline{\xymatrix{
               0 \ar@{->}[r] 
                 & Y \ar@{->}[r]^f 
                 & E \ar@{->}[r]^g
                 & X \ar@{->}[r]
                 & 0
                 &
             }}\\\\
         skoro štěpitelná posloupnost, pak dle \cite{2} Proposition. 1.14 (Kapitola 5) 
         je $X$ nerozložitelný a $Y\simeq DTr(X)$. Pokud by $X$ bylo projektivní, 
         pak by existovalo $g'\in Hom_A(X,E)$ takové, že $gg'=1_E$ a posloupnost 
         by byla štěpitelná.         
         \item[(b)] Tvrzení plyne z \cite{2} Theorem 1.15 (Kapitola 5).
       \end{description}      
     \end{proof}
     
     \begin{dfn}
       Nechť $U,V\in mod(A)$. Označme následující dvě množiny:
       \begin{description}
         \item[(a)] $\Upsilon_{U,V}:=\{$Krátké exaktní posloupnosti vedoucí z $U$ do $V\}$
         \item[(b)] $\hat{\Upsilon}_{U,V}:=\{$Skoro štěpitelné posloupnosti vedoucí z $U$ do $V\}\subseteq \Upsilon_{U,V}$  
       \end{description}      
       Definujme relaci ekvivalence $\sim$ na $\Upsilon_{U,V}$ respektive na $\hat{\Upsilon}_{U,V}$ 
       tak, že dvě posloupnosti \\
           \centerline{\xymatrix{
             0 \ar@{->}[r] 
               & U \ar@{->}[r]
               & E \ar@{->}[r]
               & V \ar@{->}[r]  
               & 0 \\
             0 \ar@{->}[r] 
               & U \ar@{->}[r]
               & E' \ar@{->}[r]
               & V \ar@{->}[r]
               & 0 
           }} \\\\\\
       jsou ekvivalentní, pokud existuje $e\in Hom_A(E,E')$ takové, že 
       následující diagram komutuje:  \\
           \centerline{\xymatrix{
             0 \ar@{->}[r] 
               & U \ar@{->}[r] \ar@{=}[d] 
               & E \ar@{->}[r] \ar@{->}[d]_e 
               & V \ar@{->}[r] \ar@{=}[d] 
               & 0 \\
             0 \ar@{->}[r] 
               & U \ar@{->}[r] 
               & E' \ar@{->}[r] 
               & V \ar@{->}[r]
               & 0 
           }} \\\\
           
          Poznamenejme, že symetrie této relace plyne z 
          \hyperref[lemma-five]{Lemma \ref*{lemma-five}}, 
          zatímco tranzitivita a reflexivita jsou zřejmé. Zároveň bychom 
          měli relaci $\sym$ indexovat koncovými moduly  posloupností 
          $U,V$, z kontextu je ale vždy zřejmé, k jakým modulům se vztahuje.
          
         Na třídách ekvivalence $\Upsilon_{U,V}/\sim$ nyní zavedeme sčítání 
         (nazýváné Baerova suma), s nímž bude mít $\Upsilon_{U,V}/\sim$ strukturu  
         abelovské grupy.
         
         Mějme tedy dvě krátké exaktní posloupnosti vedoucí z $U$ do $V$\\\\
           \centerline{\xymatrix{
             \xi_1 :0 \ar@{->}[r] 
               & U \ar@{->}[r]^{\beta_1}
               & E \ar@{->}[r]^{\alpha_1}
               & V \ar@{->}[r]  
               & 0 \\
             \xi_2 :0 \ar@{->}[r] 
               & U \ar@{->}[r]^{\beta_2} 
               & E' \ar@{->}[r]^{\alpha_2} 
               & V \ar@{->}[r]
               & 0 
           }} \\\\\\
         Nechť $[\xi_1]$ resp. $[\xi_2]$ značí jejich třídy ekvivalence. Definujme 
         $[\xi_1]+[\xi_2]\in \Upsilon_{U,V}/\sim$
         následovně: Nechť $f:V\to V\oplus V$ je dáno předpisem $f(v)=(v,v)$ pro všechna 
         $v\in V$ a $g:U \oplus U\to U$ je dáno předpisem $g(u_1,u_2)=u_1+u_2$ 
         pro všechna $u_1,u_2\in U$. Nechť $\xi_1\oplus\xi_2$ je suma \\\\
           \centerline{\xymatrix{
             \xi_1\oplus\xi_2 :0 \ar@{->}[r] 
               & U\oplus U \ar@{->}[r]^{(\beta_1,\beta_2)}
               & E\oplus E \ar@{->}[r]^{(\alpha_1,\alpha_2)}
               & V\oplus V \ar@{->}[r]  
               & 0
           }} \\\\
        a položme $[\xi_1]+[\xi_2]$ rovno třídě ekvivalence exaktní posloupnosti\\\\
           \centerline{\xymatrix{
             0 \ar@{->}[r] 
               & U \ar@{->}[r]
               & \tilde E \ar@{->}[r]
               & V \ar@{->}[r]  
               & 0
           },} \\\\
           kterou spočteme následovně jedním ze dvou možných postupů: 
           \begin{description}
             \item[(a)]
               Modul $E_1$ dostaneme jako pushout $g$ a $(\beta_1,\beta_2)$. 
               Modul $\tilde E$ následně položíme rovno pullbacku $h_1$ a $f$. \\\\             
               \centerline{\xymatrix{
                 0 \ar@{->}[r] 
                   & U\oplus U \ar@{->}[rr]^{(\beta_1,\beta_2)} \ar@{->}[d]_g
                   & & E\oplus E \ar@{->}[rr]^{(\alpha_1,\alpha_2)} \ar@{->}[d]
                   & & V\oplus V \ar@{->}[r] \ar@{=}[d]
                   & 0 \\
                 0 \ar@{->}[r] 
                   & U \ar@{->}[rr] \ar@{=}[d]
                   & &  E_1 \ar@{->}[rr]^{h_1} 
                   & & V\oplus V \ar@{->}[r]   
                   & 0 \\
                 0 \ar@{->}[r] 
                   & U \ar@{->}[rr]
                   & & \tilde E \ar@{->}[rr] \ar@{->}[u]
                   & & V \ar@{->}[r]  \ar@{->}[u]_f
                   & 0 & &
                 }}\\     
             \item[(b)]
               Modul $E_2$ dostaneme jako pullback $f$ a $(\alpha_1,\alpha_2)$. 
               Modul $\tilde E$ následně položíme rovno pushoutu $h_2$ a $g$. \\\\             
               \centerline{\xymatrix{
                 0 \ar@{->}[r] 
                   & U\oplus U \ar@{->}[rr]^{(\beta_1,\beta_2)} \ar@{=}[d]
                   & & E\oplus E \ar@{->}[rr]^{(\alpha_1,\alpha_2)} 
                   & & V\oplus V \ar@{->}[r] 
                   & 0  \\
                 0 \ar@{->}[r] 
                   & U\oplus U \ar@{->}[rr]^{h_2} \ar@{->}[d]_g
                   & &  E_2 \ar@{->}[rr] \ar@{->}[d] \ar@{->}[u]
                   & & V \ar@{->}[r] \ar@{->}[u]_f
                   & 0 \\
                 0 \ar@{->}[r] 
                   & U \ar@{->}[rr]
                   & & \tilde E \ar@{->}[rr] 
                   & & V \ar@{->}[r]  \ar@{=}[u]
                   & 0 & &
                 }}\\
           \end{description}
           
           Oběma postupy dospějeme ke stejnému výsledku, jak je znázorněno na 
           následujícím komutativním diagramu:
           \\\\
           \centerline{\xymatrix{
             0 \ar@{.>}[rr] 
               & & U\oplus U \ar@{->}[rr]^{(\beta_1,\beta_2)} \ar@{=}[dd] \ar@{->}[ld]_g
               & & E\oplus E \ar@{->}[rr]^{(\alpha_1,\alpha_2)} \ar@{->}[ld]
               & & V\oplus V \ar@{.>}[r]  \ar@{=}[ld]
               & 0 \\
             0 \ar@{.>}[r] 
               & U \ar@{->}[rr] \ar@{=}[dd]
               & & E_1 \ar@{->}[rr] \ar@{->}[dd]
               & & V\oplus V \ar@{.>}[rr]  
               & & 0 \\
             0 \ar@{.>}[rr] 
               & & U\oplus U \ar@{->}[rr] \ar@{->}[ld]_g
               & & E_2 \ar@{->}[rr] \ar@{->}[uu] \ar@{->}[ld]
               & & V \ar@{.>}[r]  \ar@{->}[uu]_f \ar@{=}[dd] \ar@{=}[ld]
               & 0 \\
             0 \ar@{.>}[r] 
               & U \ar@{->}[rr]
               & & \tilde E \ar@{->}[rr]
               & & V \ar@{.>}[rr]  \ar@{->}[uu]_f
               & & 0 \\
           }} \\\\\\
         Nulový prvek $\Upsilon_{U,V}/\sim$ bude exaktní posloupnost \\\\
           \centerline{\xymatrix{
             0 \ar@{->}[r] 
               & U \ar@{->}[r]
               & U\oplus V \ar@{->}[r]
               & V \ar@{->}[r]  
               & 0
           },} \\\\
         a inverzní prvek k posloupnosti \\\\
           \centerline{\xymatrix{
             0 \ar@{->}[r] 
               & U \ar@{->}[r]^{\beta_1}
               & E \ar@{->}[r]^{\alpha_1}
               & V \ar@{->}[r]  
               & 0
           }} \\\\ je \\\\
           \centerline{\xymatrix{
             0 \ar@{->}[r] 
               & U \ar@{->}[r]^{\beta_1}
               & E \ar@{->}[r]^{-\alpha_1}
               & V \ar@{->}[r]  
               & 0
           }.} \\
                    
         To, že oba postupy výpočtu Baerovy sumy zaručují stejný výsledek, 
         že je Baerova suma 
         správně definovaná a splňuje všechny axiomy, aby
         $\Upsilon_{U,V}/\sim$ spolu s ní tvořilo abelovskou grupu, zde 
         dokazovat nebudeme. Prodrobnou konstrukci a důkaz je možné nalézt v 
         \cite{2} Kapitole I Sekci 5.
     \end{dfn}
          
     Zvolme si nyní pevně libovolný nerozložitelný a neprojektivní $A$-modul 
     $X$. Budeme s ním pracovat po zbytek této kapitoly.
  
     \begin{dfn}
       Něchť $\Gamma:=\underline{End}_A(X)$.
     \end{dfn}
     
     \begin{lem}\label{lem-D-na-jednoduchych}
     \begin{description}
       \item
       \item[(a)] Artinovská $R$-algebra $\Gamma$ je lokální okruh. 
       \item[(b)] Pokud $M\in mod(\Gamma)$ je nenulový, pak $D(M)\in mod(\Gamma^{op})$ je nenulový. 
       \item[(c)] Pokud $M\in mod(\Gamma)$ je jednoduchý, pak $D(M)\in mod(\Gamma^{op})$ je jednoduchý. 
      \end{description} 
     \end{lem}
     \begin{proof}
       \begin{description}
         \item
         \item[(a)] Dle \cite{2} Theorem 2.2 (Kapitola 2) je $End_A(X)$ 
           lokální okruh. Pak je dle \hyperref[faktor-lokalniho-lokalni]{Lemma \ref*{faktor-lokalniho-lokalni}}
           lokálním okruhem i $\Gamma$. 
           
         \item[(b)] Nechť $M\in mod(\Gamma)$ a $m\in M$ je nenulový prvek. 
           Připomeňme, že $M$ je zároveň $R$-modulem a $I$ je injektivní obal $R/\underline{m}$, 
           kde $\underline{m}$ je maximální ideál $R$. Nechť $\hat f\in Hom_R(Rm,I)$ 
           je daný předpisem \\\\
           \centerline{$\hat f (rm):=r+\underline m$} \\\\
           a $i$ je kanonická inkluze \\\\
           \centerline{$i:Rm\rightarrow M$.} \\\\
           Protože $I$ je injektivní, existuje $f\in Hom_R(M,I)$  takové, že 
           \\\\
           \centerline{$fi=\hat f$} \\\\
           a \\\\
           \centerline{$f(m)=fi(m)=\hat f (m)=1_R+ \underline m\neq 0$.} \\\\
           A tedy $f\in D(M)$ je nenulový prvek.           
         
         \item[(c)] Nechť $M\in mod(\Gamma)$ je jednoduchý. Pokud $M=0$, pak $D(M)=0$. 
           Nechť tedy je $M$ nenulový modul.  Předpokládejme pro spor, že $D(M)$ není jednoduchý 
           $\Gamma^{op}$-modul. Pak $D(M)$ obsahuje netriviální podmodul $U$. 
           To nám dává následující exaktní posloupnost $mod(\Gamma^{op})$:
           \\\\
           \centerline{\xymatrix{
             0 \ar@{->}[r] 
               & D(D(M)/U) \ar@{->}[r] 
               & M \ar@{->}[r] 
               & D(U) \ar@{->}[r] 
               & 0 \\
           }} \\\\
           Připomeňme, že funktor $D=Hom_R(-,I):mod(\Gamma)\rightarrow mod(\Gamma^{op})$ 
           je dualita a tedy máme $\Gamma$-modulový izomorfismus $D^2(N)\simeq N$ 
           pro každý $N\in mod(\Gamma)$. Aplikujeme-li tedy funktor $D$ na naši exaktní posloupnost, 
           dostaneme následující posloupnost v $mod(\Gamma)$:          
           \\\\
           \centerline{\xymatrix{
             0 \ar@{->}[r] 
               & U \ar@{->}[r] 
               & D(M) \ar@{->}[r] 
               & D(M)/U \ar@{->}[r] 
               & 0 \\
           }} \\\\
           Protože $U\neq D(M)$, tak $D(M)/U\neq 0$. Pak jelikož je $M$  
           jednoduchý, tak $D(D(M)/U)\simeq D^2(M)$ a z exaktnosti naší 
           posloupnosti vidíme, že  $D(U)=0$. Pak ale i $D=0$ dle bodu (a), což 
           je spor s naším předpokladem, že $U$ je netriviální podmodul $D(M)$.
       \end{description}
     \end{proof}
     
     \begin{dfn}
       Nechť $S$ je artinovský okruh, $M\in mod(S)$.  Definujme
       \begin{description}
         \item[(a)] $Top_S(M):=M/rad(S)M$.
         \item[(b)] $Soc_S(M):=\sum\{U|U$ je jednoduchý $S$-podmodul  $M\}$.
       \end{description} 
     \end{dfn}
     
     \begin{lem}\label{lem-soc-top}
       Platí:
       \begin{description}
         \item[(a)] $Top_\Gamma(\Gamma)$ je jednoduchý $\Gamma$-modul. 
         \item[(b)] $Soc_\Gamma(D\Gamma)\simeq DTop_{\Gamma^{op}}(\Gamma)$ jako 
         $\Gamma$-moduly.
         \item[(c)] $Soc_\Gamma(D\Gamma)$ je jednoduchý $\Gamma$-modul.
       \end{description} 
     \end{lem}
     
     \begin{proof}
       \begin{description}
         \item
         \item[(a)] Máme izomorfismus $End_\Gamma(\Gamma)=Hom_\Gamma(\Gamma,\Gamma)\simeq 
           \Gamma$. Dle \hyperref[lem-D-na-jednoduchych]{Lemma \ref*{lem-D-na-jednoduchych}} 
           je $End_\Gamma(\Gamma)$ lokální okruh. Protože $\Gamma$ je artinovská 
           $R$-algebra, je také artinovským okruhem. Navíc $\Gamma$ je 
           projektivní $\Gamma$-modul. Z toho plyne, že $rad(\Gamma)\Gamma$ je 
           jediný maximální podmodul $\Gamma$.
           
           Ukážeme nyní, že $Top_\Gamma(\Gamma)=\Gamma/rad(\Gamma)\Gamma$ je 
           jednoduchý $\Gamma$-modul. Nechť $M$ je nenulový podmodul 
           $\Gamma/rad(\Gamma)\Gamma$. Pak $M$ je tvaru \\\\
           \centerline{$M=N/rad(\Gamma)\Gamma$} \\\\
           pro nějaký $\Gamma$-modul $N$ takový, že \\\\
           \centerline{$rad(\Gamma)\Gamma\subseteq N\subseteq \Gamma$.}\\\\
           Protože $M$ je nenulový je $rad(\Gamma)\neq N$. Protože $rad(\Gamma)\Gamma$ 
           je maximální podmodul $\Gamma$, musí být $N=\Gamma$ a tedy \\\\
           \centerline{$M=\Gamma/rad(\Gamma)\Gamma$.}
           
         \item[(b)] Uvažujme následující exaktní posloupnost 
           $\Gamma^{op}$-modulů: \\\\
           \centerline{$0
             \longrightarrow rad(\Gamma)  
             \longrightarrow \Gamma   
             \longrightarrow \Gamma/rad(\Gamma)  
             \longrightarrow 0$} \\\\
           Aplikací funktoru $D$ dostaneme exaktní posloupnost $\Gamma$-modulů:\\\\
           \centerline{$0
             \longrightarrow D(\Gamma/rad(\Gamma))  
             \longrightarrow D(\Gamma)   
             \longrightarrow D(rad(\Gamma))  
             \longrightarrow 0$} \\\\
           Dle \cite{2} Proposition 3.1 (Kapitola 1) je $\Gamma/rad(\Gamma)$ 
           polojednoduchý $\Gamma^{op}$-modul. Tedy z 
           \hyperref[lem-D-na-jednoduchych]{Lemma \ref*{lem-D-na-jednoduchych}} 
           a z komutativity $D$ s konečnými direktními sumami (plyne z 
           \hyperref[dir-sum-hom]{Lemma \ref*{dir-sum-hom}}) je  
           $D(\Gamma/rad(\Gamma))$ polojednoduchý podmodul $D\Gamma$. To 
           znamená, že \\\\
           \centerline{$D(\Gamma/rad(\Gamma))\subseteq Soc_\Gamma(D\Gamma)$.} 
           \\\\
           Ze stejného důvodu je $DSoc_\Gamma(D\Gamma)$ polojednoduchý 
           $\Gamma^{op}$-modul. Navíc \\\\
           \centerline{$Soc_\Gamma(D\Gamma)\subseteq D\Gamma$} 
           \\\\
           a dosáváme následující komutativní diagram v $mod(\Gamma)$:\\\\
           \centerline{\xymatrix{
               & 0 \ar@{->}[d] \\
             0 \ar@{->}[r]
               & D(\Gamma/rad(\Gamma)) \ar@{->}[r] \ar@{->}[d]
               & D\Gamma \ar@{=}[d] \\
             0 \ar@{->}[r]
               & Soc_\Gamma(D\Gamma) \ar@{->}[r]
               & D\Gamma \ar@{->}[r]
               & D\Gamma/Soc_\Gamma(D\Gamma) \ar@{->}[r]
               & 0
           }}\\\\\\
           Aplikací $D$ dostaneme následující komutativní diagram z exaktními 
           řádky v $mod(\Gamma^{op})$:\\\\
           \centerline{\xymatrix{
              0 \ar@{->}[r]
                & D(D\Gamma/Soc_\Gamma(D\Gamma)) \ar@{->}[r]
                & \Gamma \ar@{=}[d] \ar@{->}[rr]^b
                & & D(Soc_\Gamma(D\Gamma))  \ar@{->}[r] \ar@/^2pc/[d]^d
                & 0 \\
              0 \ar@{.>}[r]
                & rad(\Gamma) \ar@{.>}[r]^a
                & \Gamma \ar@{->}[rr]^c
                & & \Gamma/rad(\Gamma) \ar@{->}[r] \ar@{->}[d] \ar@/^2pc/[u]^e
                & 0 \\
              & & & & 0
           }}\\\\           
           Víme, že $rad(\Gamma)$ je jádro $c$, tedy ho doplníme do 
           diagramu a dostaneme exaktní řádek. Protože $rad(\Gamma)$ anihiluje 
           polojednoduché \,$\Gamma^{op}$-moduly, \,je $b1_\Gamma a=0$. Navíc
           $\Gamma/rad(\Gamma)$ je kojádro $a$ a tedy musí existovat 
           homomorfismus
           $e\in Hom_{\Gamma^{op}}(\Gamma/rad(\Gamma), D(Soc_\Gamma(D\Gamma)))$ 
           takový, že \\\\
           \centerline{$ec=b1_\Gamma=b$.}\\\\
           Navíc protože $b$ je epimorfismus, tak je jím také $e$. 
           Složení homomorfismů $de\in End_{\Gamma^{op}}(\Gamma/rad(\Gamma))$ je tedy podle 
           \cite{2} Proposition 1.4 (Kapitola 1) izomorfismus, protože $\Gamma/rad(\Gamma)$ 
           je konečně generovaný modul. Z toho plyne že $e$ je navíc 
           epimorfismem, neboli \\\\
           \centerline{$D(Soc_\Gamma(D\Gamma))\simeq \Gamma/rad(\Gamma)=Top_{\Gamma^{op}}(\Gamma)$} \\\\
           jakožto $\Gamma^{op}$-moduly. A ekvivalentně \\\\
           \centerline{$Soc_\Gamma(D\Gamma)\simeq DTop_{\Gamma^{op}}(\Gamma)$} 
           \\\\
           jako $\Gamma$-moduly.     
                    
         \item[(c)] Plyne z (a), pokud budeme nahlížet na $\Gamma$ jako na 
           $\Gamma^{op}$-modul. Protože $Top_{\Gamma^{op}}(\Gamma)$ je jednoduchý 
           $\Gamma^{op}$-modul, plyne z 
           \hyperref[lem-D-na-jednoduchych]{Lemma \ref*{lem-D-na-jednoduchych}} a ze 
           vztahu $Soc_\Gamma(D\Gamma)\simeq DTop_{\Gamma^{op}}(\Gamma)$, že je 
           jednoduchý $\Gamma$-modul.
       \end{description}
     \end{proof}

     \begin{lem}\label{lem-jednoduchy-modul-gen}
       Každý jednoduchý modul $M$ okruhu $S$ může být vygenerován jakýmkoli svým 
       nenulovým prvkem.
     \end{lem}
     \begin{proof}
       Pokud $m\in M\backslash\{0\}$, pak $m$ generuje nenulový podmodul $M$, 
       který musí být z jednoduchosti $M$ roven celému $M$.
     \end{proof}
     
     \begin{thm}\label{ekvivalence-upsilon-ext}
       \begin{description} \item 
         \item[(a)] Nechť $U,V \in mod(A)$. Pak $\Upsilon_{U,V}/\sim$ je 
           abelovská grupa a máme izomorfismus abelovských grup: \\\\ 
           \centerline{$Ext_A^1(V,U)\simeq (\Upsilon_{U,V}/\sim)$}
         \item[(b)] Nechť $X\in mod(A)$ je nerozložitelný a neprojektivní. Potom 
           můžeme definovat na $\Upsilon_{DTr(X),X}/\sim$ takovou strukturu, že:
           \begin{description}
             \item[(i)] $(\Upsilon_{DTr(X),X}/\sim)\in mod(\Gamma)$,
             \item[(ii)] 
             $Soc_\Gamma(\Upsilon_{DTr(X),X}/\sim)=(\hat{\Upsilon}_{DTr(X),X}/\sim)$.
           \end{description}
       \end{description}      
     \end{thm}
     \begin{proof}
       \begin{description}
         \item
         
         \item[(a)] \cite{5} Theorem. 7.21.
           
         \item[(b)] Nebudeme zde provádět kompletní důkaz. Všechny podrobnosti 
         je možné nalézt v \cite{2} Kapitole 5. Dále pouze naznačíme důkaz části (i).
         
         \begin{description}
         \item[(i)]
         Základem je pozorování, že $(\Upsilon_{DTr(X),X}/\sim)\in Mod(\Gamma)$ 
         Důkaz tohoto pozorování provedeme v části věnované konstrukci algoritmu v 
         \hyperref[upsilon-je-modul]{Lemma \ref*{upsilon-je-modul}}. 
         Potom izomorfismus z (a) přenáší $\Gamma$-modulovou strukturu i na
         $Ext_A^1(X,DTr(X))$ a stává se $\Gamma$-izomorfismem.         
         V důkazu \hyperref[thm-omega-x]{Věty \ref*{thm-omega-x}} ukážeme, že
         $Ext_A^1(X,DTr(X))=\delta_*(DTr(X))$ pro určitou krátkou exaktní posloupnost $\delta$
         a protože dle
         \hyperref[thm-delta]{Věty \ref*{thm-delta}} je
         $\delta_*(DTr(X))\in mod(R)$, pak i $Ext_A^1(X,DTr(X))\in mod(R)$.
         A tedy dle \hyperref[lem-mod-Mod]{Lemma \ref*{lem-mod-Mod}} vidíme, že 
         oba $Ext_A^1(X,DTr(X))$ i $(\Upsilon_{DTr(X),X}/\sim)$ jsou konečně generované jako $\Gamma$-moduly.
         \end{description}         
       \end{description}
     \end{proof}
  \clearpage