\chapter{Závěr}

\paragraph{ } V této práci jsme prošli konstrukcí algoritmu pro výpočet generátoru skoro štěpitelných posloupností 
od základů dané problematiky až po důkaz jeho správnosti. V následujících částech jsme 
provedli implementaci v teorii reprezentací, uvedli několik příkladů a srovnali 
algoritmus s jiným již implementovaným.

Podívejme se nyní na implementaci algoritmu. Samotný kód je poměrně rozsáhlý a konstrukce složitá, 
ale velké množství pomocných funkcí použitých během výpočtu nám dává mnoho možností pro pozdější
optimalizaci za účelem zvýšení rychlosti. Přesto je již nyní tato implementace v mnoha případech složitějšího
toulce rychlejší než aktuální algoritmus v balíku \cite{QPA}. 

Samotná práce by při svojí podrobnosti měla obsahovat vše potřebné pro pochopení algoritmu i jeho implementace 
a může sloužit jako vstupní brána do teorie reprezentací a systému \cite{GAP4} bez 
jakýchkoliv
předchozích znalostí dané problematiky.