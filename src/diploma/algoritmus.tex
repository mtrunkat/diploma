\chapter{Algoritmus}\label{kapitola-algoritmus}

  \praragraph{}\paragraph{ }V této kapitole popíšeme algoritmus pro nalezení generátoru skoro štěpitelných 
   posloupností končících v nerozložitelném a neprojektivním modulu $X$. Konstrukce algoritmu je 
   vychází z diplomové práce \cite{3} Tea Sormbroen Lian: Computing almost split sequences. 
   Co se týče teorie, tak navážeme na \hyperref[algebry-moduly]{části \ref*{algebry-moduly}}  a $A$-modulem budeme rozumět levý modul.
    
   V první části zkonstruujeme čtyři potřebné izomorfismy pro obecný komutativní,
   artinovský, lokální okruh $R$.    
   V druhé části konečně popíšeme algoritmus pro jednodušší případ, kdy $R=K$ bude těleso.
    \clearpage
      
  \section{Konstrukce potřebných izomorfismů$}
    \praragraph{ }\paragraph{ } V celé této části budeme pracovat nad pevně zvoleným komutativním, artinovským, 
    lokálním okruhem 
    $R$ a artinovskou $R$-algebrou $A$. 
    Dále nechť $X$ je pevně zvolený, neprojektivní, nerozložitelný $A$-modul
    a zvolme pevně jeho projektivní prezentaci\\
      \centerline{\xymatrix{
        P_1 \ar@{->}[r]^s 
          & P_0 \ar@{->}[r]^t 
          & X \ar@{->}[r] 
          & 0
      }}\\\\
      a jako $\delta$ označme libovolnou krátkou exaktní posloupnost A-modulů \\\\
      \centerline{\xymatrix{
        0 \ar@{->}[r] 
          & M \ar@{->}[r]^f
          & N \ar@{->}[r]^g 
          & L \ar@{->}[r] 
          & 0
      }.}\\\\
  
    \subsection{Izomorfismus $\varphi_{P,Y}$}
      Naším cílem je zkonstruovat $R$-izomorfismus \\\\ 
      \centerline{$\varphi_{P,Y}:Hom_A(P,Y)\rightarrow Hom_A(P,A)\otimes_A Y$,} 
      \\\\
      kde $P,Y\in mod(A)$ a $P$ je projektivní.
            
      \begin{lem}\label{lemma-alpha-P-Y}
        Nechť $Y\in mod(A)$ a $P\in P(A)$. Definujme zobrazení\\\\
        \centerline{$\alpha_{P,Y}:Hom_A(P,A)\otimes_A Y\rightarrow Hom_A(P,Y)$} 
        \\\\
        předpisem \\
        \centerline{$\alpha_{P,Y}(f\otimes y):=[p\mapsto f(p)y]$.}\\\\
        Zobrazení $\alpha_{P,Y}$ je homomorfismem R-modulů přirozeným ve 
        složkách $P$ i $Y$.
      \end{lem}
      \begin{proof}
        Vidíme, že $[p\mapsto f(p)y]\in Hom_A(P,Y)$ pro libovolný $f\in Hom_A(P,Y)$ 
        a $y\in Y$, protože \\\\
        \centerline{$f(p+\lambda p')y=f(p)y+\labda f(p')y$} \\\\
        pro každé $p,p'\in P$ a $\lambda\in A$. Nechť navíc $r\in R$, $f\in Hom_A(P,A)$ 
        a $y\in Y$. Pak
        \begin{eqnarray}
          r\cdot \alpha_{P,Y}(f\otimes y)
           &=& r[p\mapsto f(p)y]  \nonumber \\
           &=& [p\mapsto r(f(p)y)]  \nonumber \\
           &=& [p\mapsto (rf(p))y]  \nonumber \\
           &=& [p\mapsto (rf)(p)y]  \nonumber \\
           &=& \alpha_{P,Y}(rf\otimes y)  \nonumber \\
           &=& \alpha_{P,Y}(r\cdot f\otimes y)  \nonumber 
        \end{eqnarray}
        a $\alpha_{P,Y}$ je homomorfismem $R$-modulů.
        
        Dokážeme nyní, že $\alpha_{P,Y}$ je přirozené v $P$. Mějme libovolné $P,P'\in P(A)$ 
        a $h\in Hom_A(P,P')$. Musíme dokázat, že následující diagram komutuje:\\\\      
         \centerline{\xymatrix{            
           Hom_A(P', A)\otimes_A Y \ar[rr]^{\alpha_{P',Y}} \ar[d]_{(-\circ h)_A\otimes 1_Y}
           & & Hom_A(P', Y) \ar[d]^{(-\circ h)_Y} \\
           Hom_A(P, A)\otimes_A Y \ar[rr]^{\alpha_{P,Y}}
           & & Hom_A(P, Y)
         }}\\\\\\
        Nechť $f \otimes y \in Hom_A(P', A)\otimes_A Y$, pak \\\\
        \centerline{$
          (-\circ h)_A\otimes \alpha_{P',Y}(f\otimes y)
          = (-\circ h)_A\otimes ([p' \mapsto f(p')y])
          = [p\mapsto f(h(p))y]
        $}\\
        a\\
        \centerline{$
          \alpha_{P,Y}\circ[(-\circ h)_A\otimes 1_Y](f\otimes y)
          = \alpha_{P,Y}(fh\otimes y)
          = [p\mapsto f(h(p))y]
        $.}\\\\
        Diagram tedy komutuje. 
        
        Dále dokážeme přirozenost $\alpha_{P,Y}$ v $Y$. Nechť tedy $Y,Y'\in 
        mod(A)$ a $g\in Hom_A(Y,Y')$. Musíme dokázat, že následující diagram komutuje:\\\\      
         \centerline{\xymatrix{            
           Hom_A(P, A)\otimes_A Y \ar[rr]^{\alpha_{P,Y}} \ar[d]_{1_{Hom_A(P,A)}\otimes g}
           & & Hom_A(P, Y) \ar[d]^{(g\circ -)_P} \\
           Hom_A(P, A)\otimes_A Y' \ar[rr]^{\alpha_{P,Y'}}
           & & Hom_A(P, Y')
         }}\\\\\\
         Nechť $f\otimes y \in Hom_A(P, A)\otimes_A Y$. Pak \\\\
         \centerline{$
           (g\circ-)_P\circ\alpha_{P,Y}(f\otimes y)
           = (g\circ-)_P([p\mapsto f(p)\cdot y])
           =[p\mapsto g(f(p)\cdot y)]
         $} \\
         a \\
         \centerline{$
           \alpha_{P,Y'}\circ[1_{Hom_A(P,A)}\otimes g](f\otimes y)
           = \alpha_{P,Y'}(f\otimes g(y))
           = [p\mapsto f(p)\cdot g(y)]
         $.}\\\\
         Protože $f(p)\in A$ a pro každé $p\in P$ je $g$ $A$-modulovým 
         homomorfismem, tak máme rovnost \\\\
         \centerline{$g(f(p)\cdot y)=f(p)\cdot g(y)$.}\\\\         
         Diagram tedy komutuje i tímto směrem a náš homomorfismus je přirozený v obou proměných $P$ 
         i $Y$.         
      \end{proof}
      
      \begin{lem}\label{lem-varphi-a-y-izo}
        Nechť $Y\in mod(A)$. Definujme zobrazení\\\\
        \centerline{$\varphi_{A,Y}: Hom_A(A,Y) \rightarrow Hom_A(A,A)\otimes_A Y$} 
        \\\\
        předpisem \\
        \centerline{$\varphi_{A,Y}(g):=1_A\otimes g(1_A)$.}\\\\
        Zobrazení $\varphi_{A,Y}$ je homomorfismem R-modulů a je inverzní k $\alpha_{A,Y}$.
      \end{lem}
      \begin{proof}
        Jasně $1_A\otimes g(1_A)\in Hom_A(A,A)\otimes_A Y$. Nejprve ukážeme, že $\varphi_{A,Y}$ 
        je homomorfismus $R$-modulů. Nechť $r\in R$ a $g\in Hom_A(A,Y)$. Pak
        \begin{eqnarray}
          \varphi_{A,Y}(rg)
            &=& 1_A\otimes (rg)(1_A) \nonumber \\
            &=& 1_A\otimes (r1_A)(g(1_A)) \nonumber \\
            &=& r1_A\otimes (g(1_A)) \nonumber \\
            &=& r\cdot 1_A\otimes (g(1_A)) \nonumber \\
            &=& r\varphi_{A,Y}(g). \nonumber
        \end{eqnarray}
        Dále ukážeme, že platí následující dvě rovnosti:
        \begin{eqnarray}
          \varphi_{A,Y} \alpha_{A,Y} &=& 1_{Hom_A(A,A)\otimes_A Y} \nonumber \\
          \alpha_{A,Y} \varphi_{A,Y} &=& 1_{Hom_A(A,Y)} \nonumber 
        \end{eqnarray}
        Nechť $f\in Hom_A(A,A)$,  $g\in Hom_A(A,Y)$ a $y\in Y$, pak 
        \begin{eqnarray}
          \varphi_{A,Y} \alpha_{A,Y}(f\otimes g)
          &=& \varphi_{A,Y}([\lambda\mapsto f(\lambda)y]) \nonumber \\
          &=& 1_A \otimes f(1_A)y5 \nonumber \\
          &=& 1_Af(1_A)\otimes Y \nonumber \\
          &=& f\otimes y \nonumber 
        \end{eqnarray}
        a 
        \begin{eqnarray}
          \alpha_{A,Y} \varphi_{A,Y}(g)
          &=& \alpha_{A,Y} (1_A \otimes g(1_A))  \nonumber \\
          &=& [\lambda\mapsto 1_A(\lambda)g(1_A)]  \nonumber \\
          &=& [\lambda\mapsto \lambda g (1_A)=g(\lambda)]=g.  \nonumber 
        \end{eqnarray}
        Pak $\varphi_{A,Y} $ a $\alpha_{A,Y}$ jsou vzájemně inverzní homomorfismy.
      \end{proof}
      
      \begin{lem}
        Nechť $P,Y\in mod(A)$, pak platí:
        \begin{description}
          \item[(a)] $P^*=Hom_A(P,A)\in mod(A^{op})$ 
          \item[(b)] $P^*\otimes Y =Hom_A(P,A)\otimes_A Y \in mod(R)$, kde násobení prvky $R$ definujeme následovně: 
          \\
            \centerline{$r\cdot f\otimes y:=(rf)\otimes y$}  
          \item[(c)] $Hom_A(P,Y)\in mod(R)$, kde násobení prvky $R$ definujeme následovně: 
          \\\\
            \centerline{$(rf)(p):=r(f(p))$}  
        \end{description}
      \end{lem}
      
      \begin{proof}
        To že je modulová struktura u (b) a (c) dobře definovaná je zřejmé. Ověříme 
        pouze u všech tří případů, že se jedná o konečně generované moduly.
        
        \begin{description}          
          \item[(a)] Funktor $()^*$ je funktorem $P(A)\to P(A^{op})$ a tedy $P^*$ 
            je projektivní modul, který je dle \hyperref[rozklad-proj]{Věty \ref*{rozklad-proj}} 
            direktním součtem konečného počtu $A$-modulů generovaných jedním prvkem. 
            Je tedy sám konečně generovaný.
            
          \item[(b)] Tenzorový součin $P^*\otimes Y$ je faktor $R$-modulu kartézského součinu 
             dvou konečně generovaných $R$-modulů a je tedy sám konečně 
             generovaný.
           
          \item[(c)] Plyne z (b) a z izomorfismu z \hyperref[lem-varphi-a-y-izo]{Lemma 
            \ref*{lem-varphi-a-y-izo}}.          
        \end{description}
      \end{proof}      
      
        Homomorfismy $\varphi_{A,Y}$ a $\alpha_{A,Y}$ jsou vzájemně inverzní, 
        jde tedy o izomorfismy. Označme $\varphi_{A,Y}^n$ diagonální $n\times n$ matici, jejíž všechny 
        nenulové prvky jsou rovny $\varphi_{A,Y}$.  Ta definuje zobrazení \\\\
        \centerline{$\varphi_{A,Y}^n:Hom_A(A,Y)^n\rightarrow(Hom_A(A,A)\otimes_A Y)^n$} 
        \\\\
        předpisem \\
        \centerline{$\{f_i\}_{i=1}^n\mapsto \{\varphi_{A,Y}(f_i)\}_{i=1}^n$.}\\\\
        Víme, že $\varphi_{A,Y}$ je izomorfismus, pak je jím také 
        $\varphi^n_{A,Y}$.
        
        Nyní se již pustíme do konstrukce cílového homomorfismu $\varphi_{P,Y}$.
        Následující diagram ilustruje postup, kterým ve třech krocích nalezneme hledaný izomorfismus 
        $\varphi_{P,Y}:Hom_A(P,Y)\rightarrow Hom_A(P,A)\otimes_A Y$: \\\\      
         \centerline{\xymatrix{            
            (Hom_A(A,Y))^n  
              \ar[r]^{ \varphi_{A, Y}^n}
            & (Hom_A(A,A)\otimes_A Y)^n
              \ar[d]\\
           Hom_A(A^n,Y)
              \ar[u]
              \ar@{.>}[r]_{ \varphi_{A^n, Y}}
            & Hom_A(A^n,A)\otimes_A Y
              \ar[d] \\             
            Hom_A(P\oplus P', Y)
              \ar[u]
              \ar@{.>}[r]_{ \varphi_{P\oplus P', Y}}
            & Hom_A(P\oplus P', A)\otimes_A Y
              \ar[d]\\
            Hom_A(P, Y)
              \ar[u]
              \ar@{.>}[r]_{ \varphi_{P, Y}}
            & Hom_A(P, A)\otimes_A Y 
         }}\\\\
        Budeme tímto diagramem postupovat odshora, od nám již známého 
        izomorfismu $\varphi^n_{A,Y}$ k hledanému izomorfismu 
        $\varphi_{P,Y}$.
        
       \paragraph{Krok 1:} 
       
       Buď $\{\nu_i:A\rightarrow A^n\}_{i=1}^n$ množina kanonických inkluzí 
       a $\{\rho_i:A^n\rightarrow A\}_{i=1}^n$ množina kanonických projekcí. 
       Definujme homomorfismus $A^{op}$-modulů \\\\
       \centerline{$\xi_1:Hom_A(A^n,Y)\rightarrow (Hom_A(A,Y))^n$} \\\\
       předpisem $f\mapsto\{f\nu_i\}_{i=1}^n$
       a homomorfismus $R$-modulů \\\\
       \centerline{$\xi_2:(Hom_A(A,A)\otimes_A Y)^n\rightarrow Hom_A(A^n,A)\otimes_A 
       Y$}\\\\
       předpisem $\{g_i\otimes y_i\}_{i=1}^n\mapsto \sum_{i=1}^n g_i\rho_i\otimes y_i$. Složíme-li nyní 
       $\xi_1$ a $\xi_2$ s izomorfismem $\varphi_{A,Y}^n$ na 
       $\varphi_{A^n,Y}:=\xi_2\circ \varphi_{A,Y}^n\circ \xi_1$, dostaneme diagram:\\\\   
         \centerline{\xymatrix{            
            (Hom_A(A,Y))^n  
              \ar[r]^{ \varphi_{A, Y}^n}
            & (Hom_A(A,A)\otimes_A Y)^n
              \ar[d]^{\xi_2:\{g_i\otimes \omega_i\}_{i=1}^n\mapsto\sum_{i=1}^ng_i\rho_i\otimes \omega_i} \\
           Hom_A(A^n,Y)
              \ar[u]_{\xi_1:f\mapsto \{f\nu_i\}_{i=1}^n}
              \ar@{.>}[r]_{ \varphi_{A^n, Y}}
            & Hom_A(A^n,A)\otimes_A Y              
         }}\\\\
       Pro $h\in Hom_A(A^n,Y)$ platí:        
       \begin{eqnarray}
         \varphi_{A^n, Y}(h) &=& \xi_2\circ \varphi_{A,Y}^n\circ \xi_1(h)  \nonumber \\
         &=& \xi_2\circ \varphi_{A,Y}^n(\{h\nu_i\}_{i=1}^n) \nonumber \\
         &=& \xi_2(\{\varphi_{A,Y}(h\nu_i)\}_{i=1}^n) \nonumber \\
         &=& \xi_2(\{1_A\otimes h\nu_i(1_A)\}_{i=1}^n)  \nonumber \\
         &=& \sum_{i=1}^n \rho_i\otimes h\nu_i(1_A) \nonumber
       \end{eqnarray}
       
       \paragraph{Krok 2:} 
       
       Nyní přejdeme k dalšímu kroku. 
       Dle \hyperref[rozklad-A-na-P]{Věty \ref*{rozklad-A-na-P}} 
       existuje $n\in \mathbb N$ a projektivní $A$-modul $P'_0$ takové, že
       $P_0\oplus P'_0\simeq A^n$. Označme tento $A$-modulový izomorfismus 
       $\psi: P_0\oplus P_0'\simeq A^n$. Izomorfismus $\psi$ nám definuje 
       další dva $R$-modulové homomorfismy:  
       \begin{eqnarray}         
          (-\circ \psi^{-1})_Y&:&
          Hom_A(P\oplus P',Y
          )\rightarrow Hom_A(A^n,Y) \nonumber \\
         (-\circ\psi)_A\otimes 1_Y&:&
         Hom_A(A^n,A)\otimes_A Y
         \rightarrow 
         Hom_A(P\oplus P',A)\otimes_A Y \nonumber 
       \end{eqnarray}
       Položme nyní $
         \varphi_{P\oplus P',Y}:=
         [(-\circ\psi)_A\otimes 1_Y] \circ
         \varphi_{A^n,Y} \circ
         [(-\circ \psi^{-1})_Y]
       $. Celý krok ilustruje následující diagram:\\\\   
         \centerline{\xymatrix{            
           Hom_A(A^n,Y)
              \ar@{.>}[r]_{ \varphi_{A^n, Y}}
            & Hom_A(A^n,A)\otimes_A Y
              \ar[d]^{(-\circ\psi)_A\otimes1_Y} \\             
            Hom_A(P\oplus P', Y)
              \ar[u]_{(-\circ\psi^{-1})_Y}
              \ar@{.>}[r]_{ \varphi_{P\oplus P', Y}}
            & Hom_A(P\oplus P', A)\otimes_A Y
         }}\\\\\\
       Pak pro $h\in Hom_A(P\oplus_A P',Y)$ platí: 
       \begin{eqnarray}
          \varphi_{P\oplus P',Y}(h)&=&[(-\circ\psi)_A\otimes 1_Y]\circ\varphi_{A^n,Y}\circ[(-\circ\psi^{-1})_Y](h) \nonumber \\
          &=& [(-\circ\psi)_A\otimes 1_Y]\circ\varphi_{A^n,Y}(h\psi^{-1})  \nonumber \\
          &=& [(-\circ\psi)_A\otimes 1_Y](\sum_{i=1}^n\rho_i\otimes h\psi^{-1}\nu_i(1_A))  \nonumber \\
          &=& \sum_{i=1}^n\rho_i\psi\otimes h\psi^{-1}\nu_i(1_A)  \nonumber
       \end{eqnarray}
       
       \paragraph{Krok 3:}
       
       Postupme nyní k poslednímu třetímu kroku konstrukce. Uvažujme kanonickou 
       projekci \\\\
       \centerline{$\pi : P\oplus P'\rightarrow P$} \\\\
       modulu $P\oplus P'$ na $P$ a kanonickou inkluzi \\\\
       \centerline{$\mu : P\rightarrow P\oplus P'$} \\\\
       modulu $P$ do $P\oplus P'$. Ty nám definují dva $R$-modulové 
       homomorfismy: 
       \begin{eqnarray}
         (-\circ\pi)_Y &:& Hom_A(P,Y)\rightarrow Hom_A(P\oplus P', Y) \nonumber \\
         (-\circ\mu)\otimes 1_Y &:& Hom_A(P\oplus P', A)\otimes_A Y \to Hom_A(P, A)\otimes_A Y \nonumber
       \end{eqnarray}
       Definujme homomorfismus $\varphi_{P,Y}:Hom_A(P,Y)\rightarrow Hom_A(P,A)\otimes_A Y$ vztahem: \\\\
       \centerline{$\varphi_{P,Y}:= [(-\circ\mu)\otimes 1_Y] \circ \varphi_{P\otimes P',Y} \circ (-\circ\pi)_Y$}      
       \\\\  
         \centerline{\xymatrix{           
            Hom_A(P\oplus P', Y)
              \ar@{.>}[r]_{ \varphi_{P\oplus P', Y}}
            & Hom_A(P\oplus P', A)\otimes_A Y
              \ar[d]^{((-\circ\mu)_A\otimes1_Y} \\
            Hom_A(P, Y)
              \ar[u]_{(-\circ\pi)_Y}
              \ar@{.>}[r]_{ \varphi_{P, Y}}
            & Hom_A(P, A)\otimes_A Y 
         }}\\\\\\
       Pro $h\in Hom_A(P,Y)$ pak platí:
       \begin{eqnarray}
         \varphi_{P,Y}(h) &=& [(-\circ\mu)\otimes 1_Y] \circ \varphi_{P\otimes P',Y} \circ (-\circ\pi)_Y (h)  \nonumber \\
         &=& [(-\circ\mu)\otimes 1_Y]\circ\varphi_{P\otimes P',Y} (h\pi)  \nonumber \\
         &=& [(-\circ\mu)\otimes 1_Y][\sum_{i=1}^n\rho_i\psi\otimes h\pi\psi^{-1}\nu_i(1_A)]  \nonumber \\
         &=& \sum_{i=1}^n\rho_i\psi\mu\otimes h\pi\psi^{-1}\nu_i(1_A) \nonumber
       \end{eqnarray}

       \begin{thm}\label{varphi-izomorfismus}
         Zobrazení $\varphi_{P,Y}:Hom_A(P,Y)\rightarrow Hom_A(P,A)\otimes_A Y$, 
         definované \\\\
         \centerline{$\varphi_{P,Y}(h):=\sum_{i=1}^n\rho_i\psi\mu\otimes h\pi\psi^{-1}\nu_i(1_A)$} 
         \\\\
         pro $h\in Hom_A(P,Y)$, je izomorfismem $R$-modulů, který je přirozený v 
         $P$ a $Y$.
       \end{thm}
       \begin{proof}
         Zobrazení $\varphi_{P,Y}$ je konstruováno pouhým skládáním homomorfismů 
         $R$-modulů, je tedy také $R$-modulovým homomorfismem.
         
         Abychom dokázali, že se jedná o izomorfismus, dokážeme, že je inverzní k 
         homomorfismu $\alpha_{P,Y}$ definovanému v \hyperref[lemma-alpha-P-Y]{Lemma 
         \ref*{lemma-alpha-P-Y}}. Nechť tedy $h\in Hom_A(P,Y)$, pak
         
         \begin{eqnarray}
           \alpha_{P,Y} \varphi_{P,Y}(h) &=& \alpha_{P,Y} \left(  \sum_{i=1}^n\rho_i\psi\mu\otimes h\pi\psi^{-1}\nu_i(1_A) \right) \nonumber \\
           &=& 
             \left[ 
               p \mapsto \left( 
                 \sum_{i=1}^n
                 \underbrace{ \rho_i\psi\mu(p) }_{\in A} 
                 \cdot 
                 \underbrace{ h\pi\psi^{-1}\nu_i }_{A-hom.}
                 (1_A) 
               \right) \right] \nonumber \\
           &=& \left[ p \mapsto \left( \sum_{i=1}^n h\pi\psi^{-1}\nu_i\rho_i\psi\mu(p) \right) \right] \nonumber \\
           &=& 
             \left[ 
               p \mapsto h\pi\psi^{-1} 
               \underbrace{ \left( \sum_{i=1}^n \nu_i\rho_i \right) }_{=1_{A^n}}
               \psi\mu(p) 
             \right]\nonumber \\
           &=& [p\mapsto h(p)] \nonumber \\
           &=& h. \nonumber
         \end{eqnarray}         
         Opačně nechť $f\otimes y\in Hom_A(P,A)\otimes_A Y$, pak
         
         \begin{eqnarray}
           \varphi_{P,Y} \alpha_{P,Y} (f\otimes y) &=& \varphi_{P,Y}([p\mapsto f(p)\cdot y]) \nonumber \\
           &=& \sum_{i=1}^n \rho_i\psi\mu \otimes [p\mapsto f(p)\cdot y](\pi\psi^{-1}\nu_i(1_A)) \nonumber \\
           &=& \sum_{i=1}^n \rho_i\psi\mu \otimes \underbrace{f\pi\psi^{-1}\nu_i(1_A)}_{\in A} \cdot y  \nonumber \\
           &=& \sum_{i=1}^n \rho_i\psi\mu \cdot \underbrace{f\pi\psi^{-1}\nu_i(1_A)}_{\in A} \otimes y \nonumber \\
           &=& \left[  
             p \mapsto \sum_{i=1}^n 
               \underbrace{\rho_i\psi\mu(p)}_{\in A}
               \cdot 
               \underbrace{f\pi\psi^{-1}\nu_i}_{A-hom.}
               (1_A)
           \right] \otimes y \nonumber \\
           &=& \left[  
             p \mapsto \sum_{i=1}^n 
               f\pi\psi^{-1}\nu_i
               ( \rho_i\psi\mu(p) )
           \right] \otimes y \nonumber \\
           &=& 
             f\pi\psi^{-1} 
             \left( 
               \underbrace{\sum_{i=1}^n\nu_i\rho_i}_{=1_{A^n}}
             \right)
             \psi\mu \otimes y
             \nonumber \\
           &=& f\otimes y.  \nonumber
         \end{eqnarray}
        Homomorfismus $\varphi_{P,Y}$ je tedy inverzní k $\alpha_{P,Y}$ a tedy 
        izomorfimus. Přirozenost v $P$ a $Y$ plyne z přirozenosti $\alpha_{P,Y}$ v $P$ a $Y$ 
        dokázané v \hyperref[lemma-alpha-P-Y]{Lemma \ref*{lemma-alpha-P-Y}}. 
       \end{proof}


   \subsection{Izomorfismus $\sigma_{\delta,X}$}\label{alg-sigma}
   
      Naším cílem je zkonstruovat $\underline{End}_A(X)^{op}$-izomorfismus \\\\
      \centerline{$\sigma_{\delta,X}:\delta^*(X)\rightarrow Ker(1_{Tr(X)}\otimes f)$.} \\\\
      Aplikujme funktor $()^*$ na minimální projektivní prezentaci modulu $X$.  
      Dostaneme následující krátkou exaktní posloupnost v $mod(A^{op})$, kde $\hat{t}$ 
      značí kanonickou projekci $Hom_A(P_1,Y)\rightarrow Tr(X)=Cok((-\circ 
      s)_A)$:\\\\
      \centerline{\xymatrix{
        Hom_A(P_0,A) \ar@{->}[r]^{(-\circ s)_A}
          & Hom_A(P_1,A) \ar@{->}[r]^{\hat{t}}
          & Tr(X) \ar@{->}[r] 
          & 0
      }}*\\\\
      Dále připomeňme následující funktory, kde $Y\in mod(A):$\\\\
      \centerline{$Hom_A(-,Y):mod(A)\rightarrow mod(R)$}\\\\
      \centerline{$-\otimes_A Y: Mod(A^{op})\rightarrow Mod(R)$}\\\\
      Aplikujeme-li funktor $Hom_A(-,Y)$ na minimální projektivní prezentaci 
      modulu $X$ a funktor $-\otimes_A Y$ na posloupnost *, dostaneme 
      následující komutativní diagram s exaktními řádky v $mod(R)$ (moduly ve spodním 
      řádku jsou konečně generované jakožto izomorfní obrazy konečně 
      generovaných modulů v horním řádku):\\\\
      \centerline{\xymatrix{
        0 \ar[r]
          & Hom_A(X,Y) \ar@{->}[r]^{(-\circ t)_Y}
          & Hom_A(P_0,Y) \ar@{->}[r]^{(-\circ s)_Y} \ar@{->}[d]^{\varphi_{P_0,Y}}
          & Hom_A(P_1,Y) \ar@{->}[d]^{\varphi_{P_1,Y}} \\
        & & Hom_A(P_0,A)\otimes_A Y \ar@{->}[r]^{(-\circ s)_A\otimes 1_Y}
          & Hom_A(P_1,A)\otimes_A Y \ar@{->}[r]^{\hat{t}\otimes 1_Y}
          & Tr(X)\otimes_A Y \ar@{->}[r] 
          & 0
      }}\\\\
      
      \begin{dfn}\label{def-phi} 
        Nechť $Y\in mod(A)$. Definujme homomorfismus $R$-modulů \\\\
        \centerline{$\phi_Y:Hom_A(P_1,Y)\rightarrow Tr(X)\otimes_A Y$} \\\\
        vztahem \\\\
        \centerline{$\phi_Y:=[\hat{t}\otimes 1_Y]\circ\varphi_{P_1,Y}$.}\\
      \end{dfn}
      
      \begin{pzn}
        Pro $h\in Hom_A(P_1,Y)$ máme        
        \begin{eqnarray}
          \phi_Y(h) &=& [\hat{t}\otimes 1_Y]\circ\varphi_{P_1,Y}(h) \nonumber \\
          &=& 
            [\hat{t}\otimes 1_Y] \left(  
              \sum_{i=1}^n\rho_i\psi\mu \otimes h\pi\psi^{-1}\nu_i(1_A)
            \right) \nonumber \\
          &=& \sum_{i=1}^n\hat{t}(\rho_i\psi\mu) \otimes h\pi\psi^{-1}\nu_i(1_A) \nonumber
        \end{eqnarray}
      \end{pzn}
      
      \begin{lem}\label{lem-exakt-radek-phi-Y}
        Homomorfismus $R$-modulů $\phi_Y$ je přirozený v $Y$ a následující 
        posloupnost $R$-modulů je exaktní: \\\\
      \centerline{\xymatrix{
        0 \ar@{->}[r]
          & Hom_A(X,Y) \ar@{->}[r]^{(-\circ t)_Y}
          & Hom_A(P_0,Y) \ar@{->}[r]^{(-\circ s)_Y} 
          & Hom_A(P_1,Y) \ar@{->}[r]^{\phi_Y}
          & Tr(X)\otimes_A Y \ar@{->}[r] 
          & 0
      }}\\
      \end{lem}
      \begin{proof}
        Diagram komutuje díky přirozenosti $\varphi_{P,Y}$.
        Exaktnost posloupnosti plyne z definice $\phi_Y$ a \hyperref[lem-komut-schod]{Lemma 
        \ref*{lem-komut-schod}}.
        
        Ukážeme nyní, že $\phi_Y$ je přirozené v $Y$. Nejprve ukážeme, že 
        homomorfismus $\hat t\otimes 1_Y$ 
        je v $Y$ přirozený. Nechť $Y,Y'\in mod(A)$ a nechť $h\in Hom_A(Y,Y')$. 
        Pak \\\\
        \centerline{\xymatrix{
          Hom_A(P_1,Y)\otimes_A Y \ar@{->}[rr]^{\hat t\otimes 1_Y} \ar@[->][d]_{1_{Hom_A(P_1,Y)}\otimes h}
            & & Tr(X)\otimes_A Y  \ar@[->][d]^{1_{Tr(X)}\otimes h} \\
          Hom_A(P_1,Y)\otimes_A Y' \ar@{->}[rr]^{\hat t\otimes 1_Y'} 
            & & Tr(X)\otimes_A Y' 
        }}\\\\\\
        zřejmě komutuje, protože \\\\
        \centerline{$
          [1_{Tr(X)}\otimes h]\circ [\hat t \otimes 1_Y]
          = \hat t \otimes h
          = [\hat t\otimes 1'_Y] \circ [1_{Hom_A(P_1,A)}\otimes h]
        $.} \\\\
        Navíc protože dle \hyperref[varphi-izomorfismus]{Věty \ref*{varphi-izomorfismus}} 
        je $\varphi_{P_1, Y}$ přirozený v $Y$, je jejich složení $\phi_Y$ také 
        přirozené v $Y$.        
      \end{proof}
      
      Připomeňme, že jsme si jako na začátku této kapitoly zvolili pevně exaktní 
      posloupnost $A$-modulů $\delta$\\\\
      \centerline{\xymatrix{
        0 \ar@{->}[r] 
          & M \ar@{->}[r]^f
          & N \ar@{->}[r]^g 
          & L \ar@{->}[r] 
          & 0
      }.}\\
      
      \begin{thm}\label{mega-diagram}
        Následující diagram je komutativní a jeho řádky i sloupce jsou exaktní 
        posloupnosti $R$-modulů:\\\\
      \centerline{\xymatrix{
        & 
          & 
          & 
          & 0 \ar@{->}[d] 
          & \\
        & 0 \ar@{->}[d]
          & 0 \ar@{->}[d] 
          & 0 \ar@{->}[d] 
          & Ker(1_{Tr(X)}\otimes f) \ar@{_{(}->}[d]
          & \\
        0 \ar@{->}[r]
          & Hom_A(X,M) \ar@{->}[r]^{(-\circ t)_M} \ar@{->}[d]_{(f\circ-)_X} 
          & Hom_A(P_0,M) \ar@{->}[r]^{(-\circ s)_M} \ar@{->}[d]_{(f\circ-)_P_0} 
          & Hom_A(P_1,M) \ar@{->}[r]^{\phi_M} \ar@{->}[d]_{(f\circ-)_P_1} 
          & Tr(X)\otimes_A M \ar@{->}[r] \ar@{->}[d]_{1_{Tr(X)}\otimes f} 
          & 0 \\
        0 \ar@{->}[r]
          & Hom_A(X,N) \ar@{->}[r]^{(-\circ t)_N} \ar@{->}[d]_{(g\circ-)_X} 
          & Hom_A(P_0,N) \ar@{->}[r]^{(-\circ s)_N} \ar@{->}[d]_{(g\circ-)_P_0} 
          & Hom_A(P_1,N) \ar@{->}[r]^{\phi_N} \ar@{->}[d]_{(g\circ-)_P_1} 
          & Tr(X)\otimes_A N \ar@{->}[r] \ar@{->}[d]_{1_{Tr(X)}\otimes g} 
          & 0 \\
        0 \ar@{->}[r]
          & Hom_A(X,L) \ar@{->}[r]^{(-\circ t)_L} \ar@{->}[d] 
          & Hom_A(P_0,L) \ar@{->}[r]^{(-\circ s)_L} \ar@{->}[d] 
          & Hom_A(P_1,L) \ar@{->}[r]^{\phi_L} \ar@{->}[d] 
          & Tr(X)\otimes_A L \ar@{->}[r] \ar@{->}[d] 
          & 0 \\
        & \delta^*(X) \ar@{->}[d] 
          & 0
          & 0
          & 0
          & \\
        & 0
      }}\\
      \end{thm}
      \begin{proof}
        Exaktnost řádků a komutativita pravých čtverců plyne z 
        \hyperref[lem-exakt-radek-phi-Y]{Lemma \ref*{lem-exakt-radek-phi-Y}}. 
        Zbytek diagramu komutuje z asociativity skládání homomorfismů 
        $R$-modulů.
        Exaktnost levého sloupce plyne z exaktnosti zleva funktoru $Hom_A(X,-)$ a 
        definice $\delta^*(X)$. Prostřední dva sloupce jsou exaktní z 
        projektivity $A$-modulů $P_0$ a $P_1$. A konečně pravý sloupec je 
        exaktní, protože $Tr(X)\otimes_A-$ je zprava exaktní funktor.      
      \end{proof}
      
      \begin{dfn}
        Definujme zobrazení $\sigma_{\delta,X}:\delta^*(X)\rightarrow Ker(1_{Tr(X)}\otimes f)$ 
        následujícím algoritmem:     
        
        \begin{description}
          \item[\space\space\space Vstup:] $\bar{h}\in \delta^*(X)$
          
          \item[\space\space\space Výstup:] $\sigma_{\delta,X}(\bar{h})$
          
         \item[\space\space\space Průběh:] 
            \begin{description} 
               \item[]
               \item[(a)] Nejprve zvolme vzor $h\in Hom_A(X,L)$ prvku $\bar{h}$.
               \item[(b)] Zvolme libovolné $u\in Hom_A(P_0,N)$ takové, že $gu=ht$.
               \item[(c)] Nalezněme $v\in Hom_A(P_1,M)$ takové, že $fv=us$.
               \item[(d)] Položme $\sigma_{\delta,X}(\bar{h}):=\phi_M(v)$.
            \end{description}     
        \end{description}\,\,     
      \end{dfn}
      
      \begin{thm}\label{thm-sigma-delta-x}
        Zobrazení $\sigma_{\delta,X}$ je 
        izomorfismus $\underline{End}_A(X)^{op}$-modulů \\\\
        \centerline{$\sigma_{\delta,X}:\delta^*(X)\rightarrow Ker(1_{Tr(X)}\otimes f)$,} \\\\
        který je přirozený v $\delta$ a $X$.
      \end{thm}
      \begin{proof}
        Nebudeme zde podrobně dokazovat, že $\sigma_{\delta,X}$ je dobře 
        definovaný (nezávislý na volbě $h$, $u$ a $v$),
         zobrazuje do $Ker(1_{Tr(X)}\otimes f)$ a že je přirozený v $\delta$ i $X$. 
         Podrobný důkaz je možné nalézt v \cite{3} Proposition 76.
        
        Naznačíme zde alespoň jakým způsobem jsou obě strany 
        $\underline{End}_A(X)^{op}$-moduly, čímž lépe porozumíme jejich 
        struktuře pro naši další práci. Struktura $\underline{End}_A(X)^{op}$-modulu na $\delta^*(X)$ 
         je dána následujícím způsobem
         \begin{eqnarray}
           \delta^*(X)\times \underline{End}_A(X) &\to& \delta^*(X) \nonumber \\
           (\bar h,\bar e) &\mapsto& \overline{h\circ e}, \nonumber
         \end{eqnarray}
         kde $h\in Hom_A(X,L)$  a $e\in End_A(X)$ jsou libovolní reprezentanti 
         prvků $\bar h$ a $\bar e$.
         A v druhém případě je
         $\underline{End}_A(X)^{op}$-modulová struktura na $Ker(1_{Tr(X)}\otimes f)$ 
         dána zobrazením
         \begin{eqnarray}
           Ker(1_{Tr(X)}\otimes f) \times \underline{End}_A(X)^{op} &\to& Ker(1_{Tr(X)}\otimes f) 
           \nonumber \\
           (q\otimes a, \bar e) &\mapsto& 
           (Tr(e)\otimes 1_M)_{Ker}\underbrace{(q\otimes a)}_{\in Ker(1_{Tr(X)}\otimes 
           f)}= \nonumber \\
            && =(Tr(e)\otimes 1_M)\underbrace{(q\otimes a)}_{\in Tr(X)\otimes_A M}
           \nonumber
         \end{eqnarray}
         jak je ilustrováno na následujícím diagramu: \\\\
         \centerline{\xymatrix{
           0 \ar[r] 
             & Ker(1_{Tr(X)}\otimes f) \ar[r] \ar[d]^{(Tr(e)\otimes 1_M)_{Ker}}
             & Tr(X)\otimes_A M \ar[r] \ar[d]^{Tr(e)\otimes 1_M}
             & Tr(X)\otimes_A N \ar[r] \ar[d]^{Tr(e)\otimes 1_N}
             & Tr(X)\otimes_A L \ar[r] \ar[d]^{Tr(e)\otimes 1_L}
             & 0 \\
           0 \ar[r] 
             & Ker(1_{Tr(X)}\otimes f) \ar[r]
             & Tr(X)\otimes_A M \ar[r]
             & Tr(X)\otimes_A N \ar[r]
             & Tr(X)\otimes_A L \ar[r]
             & 0
         }.}\\\\\\
        Zbytek důkazu  tedy vynecháme.        
      \end{proof}
      
    \subsection{Izomorfismus $\gamma_{\delta,X}$}
        
        Nejprve připomeňme izomorfismus abelovských group z \hyperref[thm-adjunkce]{Věty \ref*{thm-adjunkce}} 
        \\\\
        \centerline{$\theta_{M,N,L}:Hom_R(M\otimes_A N,L)\rightarrow Hom_A(N,Hom_R(M,L))$,} 
        \\\\
        kde $M\in Mod(A^{op})$, $N\in Mod(A)$ a $L\in Mod(R)$, přirozený ve všech složkách. Ten využijeme ke 
        konstrukci izomorfismu 
        \\\\
        \centerline{$\gamma_{\delta, X}:D(Ker(1_{Tr(X)}\otimes f)) \rightarrow \delta_*(DTr(X))$,}
        \\\\
        kterému poté dodefinujeme strukturu $\underline{End}_A(X)$-modulového 
        homomorfismu.
        
        Budeme opět pracovat s diagramem z \hyperref[mega-diagram]{Věty 
        \ref*{mega-diagram}} a naší posloupností $\delta$. 
        Připomeňme si funktor duálu $D=Hom_R(-,I)$ 
        a uvažujme endomorfismus $h\in End_A(X)$. Potom je $Tr(h)\in End_R(Tr(X))$. 
        Máme následující komutativní diagram:\\
        \centerline{\xymatrix{
          Ker(1_{Tr(X)}\otimes f) \ar@{->}[r] \ar@{->}[d]_{(Tr(h)\otimes 1_M)_{ker}} 
            & Tr(X) \otimes_A M  \ar@{->}[rr]^{1_{Tr(X)}\otimes f} \ar@{->}[d]_{Tr(h)\otimes 1_M}
            & & Tr(X) \otimes_A N \ar@{->}[d]_{Tr(h)\otimes 1_N} \\
          Ker(1_{Tr(X)}\otimes f) \ar@{->}[r]   
            & Tr(X) \otimes_A M  \ar@{->}[rr]^{1_{Tr(X)}\otimes f}
            & & Tr(X) \otimes_A N
        }}\\\\\\
        Aplikujeme-li na jeho levý čverec funktor $D$, dostaneme následující komutativní 
        diagram:\\\\
        \centerline{\xymatrix{
          DKer(1_{Tr(X)}\otimes f) = Hom_R(Ker(1_{Tr(X)}\otimes f),I)
              \ar@{->}[r] 
              \ar@{->}[d]_{(-\circ (Tr(h)\otimes 1_M)_{ker})_I} 
            & Hom_R(Tr(X) \otimes_A M,I)  \ar@{->}[d]_{(-\circ Tr(h)\otimes 1_M)_I} \\
          DKer(1_{Tr(X)}\otimes f) = Hom_R(Ker(1_{Tr(X)}\otimes f),I)
            \ar@{->}[r]   
            & Hom_R(Tr(X) \otimes_A M,I)  
        }}\\\\\\
        Na základě tohoto diagramu zformulujeme následující lemma.
                 
        \begin{lem}
          $R$-modul $DKer(1_{Tr(X)}\otimes f)$ je spolu s násobením \\\\
          \centerline{$\underline{End}_A(X)\times DKer(1_{Tr(X)}\otimes f) \rightarrow DKer(1_{Tr(X)}\otimes f)$} 
          \\\\
          definovaným  \\\\
          \centerline{$\bar{h}\cdot\bar{z}:=(-\circ(Tr(h)\otimes1_M)_{Ker})_I(\bar{z})=\bar{z}\circ(Tr(h)\otimes1_M)_{Ker}$} 
          \\\\
          $\underline{End}_A(X)$-modulem.
        \end{lem}
        \begin{proof}
           Nejprve ukážeme, že \\\\           
           \centerline{$\bar{h}\cdot\bar{z}:=\bar{z}\circ(Tr(h)\otimes1_M)_{Ker}=0$}\\\\
           pro všechny $h\in P_A(X,X)$ a $z\in DKer(1_{Tr(X)}\otimes f)$. V  
           důkazu \hyperref[thm-sigma-delta-x]{Věty \ref*{thm-sigma-delta-x}} 
           jsme viděli, že $(Tr(h)\otimes 1_M)_{Ker}=0$ pro všechny $h\in P_A(X,X)$. 
           Tím jsme hotovi.
           
           Dále, protože $I$ je injektivní modul, pro každé $\bar z\in DKer(1_{Tr(X)}\otimes f)$ 
           existuje $z\in D(Tr(X)\otimes_A M)$ takové, že \\
           \centerline{$\bar z = zi$.}\\\\
           Uvažujme následující diagram v $mod(R)$: \\\\
        \centerline{\xymatrix{
          Ker(1_{Tr(X)}\otimes f) \ar@{->}[r]^i \ar@{->}[d]_{(Tr(h)\otimes 1_M)_{Ker}}
            & Tr(X) \otimes_A M \ar@{->}[d]^{Tr(h)\otimes 1_M}  \ar@{->}[rr]^{1_{Tr(X)\otimes f}}
            & & Tr(X) \otimes_A N \\
          Ker(1_{Tr(X)}\otimes f) \ar@{->}[r]^i \ar@{->}[d]_{\bar z}
            & Tr(X) \otimes_A M \ar@{.>}[ld]_z \ar@{->}[rr]^{1_{Tr(X)\otimes f}}
            & & Tr(X) \otimes_A N \\
          I
         }}\\\\\\
         Vidíme, že \\\\
         \centerline{$\bar h \bar z=\bar z \circ(Tr(h)\otimes 1_M)=z\circ (Tr(h)\otimes 1_M)\circ 
         i$}*\\\\\
         pro každé $z$ splňující $\bar z = zi$.
         
           Pro $q\otimes a \in Ker(1_{Tr(X)}\otimes f)$ platí
            \begin{eqnarray}
            (\bar h\bar z)(q\otimes a)
             &=& \bar z((Tr(h)\otimes 1_M)_{Ker})(q\otimes a)  \nonumber \\
             &=& z(Tr(h)\otimes 1_M)i(q\otimes a)   \nonumber \\
             &=& z(Tr(h)(q)\otimes a).  \nonumber
          \end{eqnarray}
        
          Nyní ověříme, že násobení splňuje axiom asociativity. Nechť $\bar z\in DKer(1_{Tr(X)}\otimes 
          f)$, $\bar h_1,\bar h_\in\bar{End}_A(X)$, předpokládejme, že $z\in D(Tr(X)\otimes_A M)$ 
          splňuje podmínku $\bar z = zi$ a $Tr(h_1)$ a $Tr(h_2)$ jsou 
          reprezentanti $Tr(\bar h_1)$ a  $Tr(\bar h_2)$. Pak
          \begin{eqnarray}
            \bar h_1(\bar h_2 \bar z)
              &\overset{*}{=}& \bar h_1 (\underbrace{z\circ (Tr(h_2)\otimes 1_M)\circ i}_{\in DKer(1_{Tr(X)}\otimes f)})  \nonumber \\
              &\overset{*}{=}& z \circ (Tr(h_2)\otimes 1_M) \circ (Tr(h_1)\otimes 1_M) \circ i   \nonumber \\
              &=& z (Tr(h_2)Tr(h_1)\otimes 1_M) \circ i \nonumber \\
              &=& z (Tr(h_2h_1)\otimes 1_M) \circ i \nonumber \\
              &\overset{*}{=}& (\bar h_1 \bar h_2)\bar z.  \nonumber
          \end{eqnarray}
        
          Zbytek axiomů ověřovat nebudeme, jejich ověření je přímočaré a ponecháme ho čtenáři.
        \end{proof}       
          
        \begin{lem}\label{upsilon-je-modul}
          Definujme na množině $\Upsilon_{DTr(X),L}/\sim$ tříd ekvivalence krátkých exaktních posloupností 
          vedoucích z $DTr(X)$ do $L$ násobení \\\\
          \centerline{$\underline{End}_A(X)\times (\Upsilon_{DTr(X),L}/\sim) \rightarrow (\Upsilon_{DTr(X),L}/\sim)$} 
          \\\\
          předpisem  \\\\
          \centerline{$
            \bar{h} 
            \cdot 
            (0\rightarrow DTr(X)\rightarrow E\rightarrow L\rightarrow 0)
            \mapsto
            (0\rightarrow DTr(X)\rightarrow E'\rightarrow L\rightarrow 0)
          $,} 
          \\\\
          kde modul $E'$ je pushout diagramu: \\\\
          \centerline{\xymatrix{
          DTr(X) \ar@{->}[r] \ar@{->}[d]^{DTr(h)}        
            &E 
            \\
          DTr(X)  
          }}\\\\\\
          Pak je $\Upsilon_{DTr(X),L}/\sim$ spolu s výše definovaným násobením  
          $\underline{End}_A(X)$-modulem.
        \end{lem}
        \begin{proof}
           Nejprve nechť $h_1,h_2\in \underline{End}_A(X)$. Pak \\\\
           \centerline{$\bar h_1(\bar h_2(0\to DTr(X)\to E\to L\to 0))$} \\\\
           nám dává následující diagram:  \\\\
          \centerline{\xymatrix{
          0 \ar@{->}[r]        
            & DTr(X) \ar@{->}[r]  \ar@{->}[d]^{DTr(h_2)}
            & E \ar@{->}[r] \ar@{->}[d]
            & L \ar@{->}[r] \ar@{=}[d]        
            & 0  \\
          0 \ar@{->}[r]  
            & DTr(X) \ar@{->}[r]  \ar@{->}[d]^{DTr(h_1)}
            & E' \ar@{->}[r] \ar@{->}[d]
            & L \ar@{->}[r]  \ar@{=}[d]     
            & 0\\
          0 \ar@{->}[r]  
            & DTr(X) \ar@{->}[r] 
            & E'' \ar@{->}[r] 
            & L \ar@{->}[r]       
            & 0
          }}\\\\\\
          Potřebujeme dokázat, že spodní řádek diagramu obdržíme i jako:\\\\
           \centerline{$(\bar h_1\bar h_2)(0\to DTr(X)\to E\to L\to 0)=(\overline{h_1 h_2})(0\to DTr(X)\to E\to L\to 0)$} \\\\
           Neboli, že $E''$ je zároveň pushoutem diagramu i následujícího diagramu: \\\\
          \centerline{\xymatrix{
            DTr(X) \ar@{->}[r]  \ar@{->}[d]^{DTr(h_1h_2)}
              & E  \\
            DTr(X)
          }}\\\\\\
          Protože jsou $D$ i $Tr$ kontravariantní funktory, pak \\\\
          \centerline{$DTr(h_1h_2)=DTr(h_1)DTr(h_2)$.}\\\\
          Z vlastností pushoutu je zřejmé, že $E''$ společně s korespondujícím 
          homomorfismem z $Hom_A(DTr(X),E'')$ a složením pushoutových morfismů z 
          $Hom_A(E,E')$ a $Hom_A(E,E'')$ z původního diagramu splní první 
          vlastnost pushoutu, neboli udělá následující diagram komutativní:\\\\
          \centerline{\xymatrix{
            DTr(X) \ar@{->}[r]  \ar@{->}[d]^{DTr(h_1h_2)} 
              & E \ar@{->}[d] \\
            DTr(X) \ar@{->}[r]
              & E''
          }}\\\\\\
          Ještě je třeba dokázat univerzální vlastnost pushoutu. Využijeme toho, 
          že $E'$  a $E''$ jsou pushouty následujících diagramů:\\\\
          \centerline{\xymatrix{
            DTr(X) \ar@{->}[r]  \ar@{->}[d]^{DTr(h_2)}
              & E  \\
            DTr(X)
          }\\\\\xymatrix{
            DTr(X) \ar@{->}[r]  \ar@{->}[d]^{DTr(h_1)}
              & E'  \\
            DTr(X)
          }}\\\\\\
          Mějme nějaké $E'''\in mod(A)$ spolu s morfismy z $Hom_A(E,E'')$ a $Hom_A(DTr(X),E''')$ 
          takovými, že následující diagram komutuje\\\\
          \centerline{\xymatrix{
            DTr(X) \ar@{->}[r]  \ar@{->}[d]^{DTr(h_1h_2)} 
              & E \ar@{->}[d] \\
            DTr(X) \ar@{->}[r]
              & E'''
          }}\\\\\\
          Protože $DTr(h_1h_2)=DTr(h_1)DTr(h_2)$, 
          existuje z univerzální vlastnosti pushoutu 
          $E'$ jednoznačný homorfismus z $Hom_A(E',E''')$ s odpovídajícími 
          vlastnostmi. A z univerzální vlastnosti $E''$ dostaneme stejným způsobem homomorfismus
          $Hom_A(E'',E''')$. Je zřejmé, že následující diagram komutuje \\\\
          \centerline{\xymatrix{
          DTr(X) \ar@{->}[r]  \ar@{->}[d]^{DTr(h_2)} \ar@/_3pc/[dd]_{DTr(h_1h_2)}
            & E  \ar@{->}[d]  \\
          DTr(X) \ar@{->}[r]  \ar@{->}[d]^{DTr(h_1)}
            & E' \ar@{->}[d] \ar@{.>}[ddr] \\
          DTr(X) \ar@{->}[r] \ar@{->}[drr]
            & E'' \ar@{.>}[dr] \\
          & & E'''
          }}\\\\\\
          a $E''$ je tedy pushoutem diagramu\\\\
          \centerline{\xymatrix{
            DTr(X) \ar@{->}[r]  \ar@{->}[d]^{DTr(h_1h_2)}
              & E  \\
            DTr(X)
          }}\\\\\\
          a asociativita našeho násobení je dokázána.
        \end{proof}
        
        \begin{lem}\label{lem-ext-delta-jsou-moduly}
          R-moduly $Ext^1(L,DTr(X))$ a $\delta_*(DTr(X))$ jsou zároveň $\underline{End}_A(X)$-moduly. 
        \end{lem}
        \begin{proof}
          Dle \hyperref[ekvivalence-upsilon-ext]{Věty \ref*{ekvivalence-upsilon-ext}} máme izomorfismus 
          abelovských grup \\\\ 
          \centerline{$Ext^1(L,DTr(X))\simeq \Upsilon_{DTr(X),L}/\sim $,} \\\\
           který přenáší strukturu 
          $\underline{End}_A(X)$-modulu i na $Ext^1(L,DTr(X))$.          
          Dále je \\\\
          \centerline{$\delta_*(DTr(X)) \subseteq Ext^1(L,DTr(X))$}\\\\
           jakožto $R$-modul. Potřebujeme dokázat, že 
          $\delta_*(DTr(X))$ je $\underline{End}_A(X)$-podmodul 
          $Ext^1(L,DTr(X))$. Postupovat budeme tak, že identifikujeme každý 
          prvek modulu $\delta_*(DTr(X))$ s prvkem $\Upsilon_{DTr(X),L}/\sim$ a poté 
          dokážeme, že násobením výsledného prvku prvkem $\bar{h}\in\underline{End}_A(X)$ 
          dostaneme opět prvek $\Upsilon_{DTr(X),L}/\sim$ korespondující s nějakým prvkem 
          $\delta_*(DTr(X))$. Tím bude dána struktura $\underline{End}_A(X)$-modulu 
          i na $\delta_*(DTr(X))$.
            
          Mějme tedy libovolný prvek $\bar{y}\in\delta_*(DTr(X))$. Protože z definice 
          máme \\\\
          \centerline{$\delta_*(DTr(X))=Hom_A(M,DTr(X))/Im((-\circ f )_I)$,} \\\\
          můžeme zvolit $y\in Hom_A(M,DTr(X))$ takové, že \\\\
          \centerline{$\bar{y}=y+Im((-\circ f )_I)$.} \\\\
          Dle \cite{5} Proposition 5.13 pushout v kategorii modulů vždy existuje. 
          Pak nám pushout $E$ homomorfismů $f$ a $y$ dává následující komutativní diagram v 
          $mod(A)$:  \\\\
          \centerline{\xymatrix{
          & 0 \ar@{->}[r]        
            & M \ar@{->}[r]^f \ar@{->}[d]^y
            & N \ar@{->}[r]^g \ar@{->}[d]
            & L \ar@{->}[r] \ar@{=}[d]        
            & 0  \\
          & 0 \ar@{->}[r]  
            & DTr(X) \ar@{->}[r] 
            & E \ar@{->}[r] 
            & L \ar@{->}[r]       
            & 0
          }}\\\\
          Spodní řádek diagramu je krátká exaktní posloupnost, kterou 
          identifikujeme s prvkem $\bar{y}\in\delta_*(DTr(X))$. Je možné dokázat, 
          že je určena jednoznačně až na ekvivalenci $\sim$ (viz. [5, Ch.7]).
         
         Vynásobíme-li tuto posloupnost prvkem $\bar{h}\in \underline{End}_A(X)$, dostaneme exaktní 
         posloupnost   \\\\
          \centerline{\xymatrix{
          0 \ar@{->}[r]  
            & DTr(X) \ar@{->}[r] 
            & E' \ar@{->}[r] 
            & L \ar@{->}[r]       
            & 0
          },}\\\\
          která při našem ztotožnění odpovídá prvku $DTr(h)y\in Hom_A(M,DTr(X))$, který je reprezentant 
          třídy $\bar{DTr(h)y}\in\delta_*(DTr(X))$. Vše ilustruje následující 
          diagram:  \\
          \centerline{\xymatrix{
          & 0 \ar@{->}[r]        
            & M \ar@{->}[r]^f \ar@{->}[d]^y
            & N \ar@{->}[r]^g \ar@{->}[d]
            & L \ar@{->}[r] \ar@{=}[d]        
            & 0  \\
          & 0 \ar@{->}[r]  
            & DTr(X) \ar@{->}[r] \ar@{->}[d]^{DTr(h)}
            & E \ar@{->}[r] \ar@{->}[d]
            & L \ar@{->}[r] \ar@{->}[d]      
            & 0  \\
          & 0 \ar@{->}[r]  
            & DTr(X) \ar@{->}[r] 
            & E' \ar@{->}[r] 
            & L \ar@{->}[r] 
            & 0
          }}\\\\\\
          A tedy $\delta_*(DTr(X))$ je $\underline{End}_A(X)$-podmodul 
          $Ext^1(L,DTr(X))\simeq \Upsilon_{DTr(X),L}/\sim$.
        \end{proof}
        
        \begin{thm}\label{izo-gamma-existuje}
           Nechť $(-\circ \iota )_I$ je kanonická projekce \\\\
           \centerline{$D(Tr(X)\otimes_A M)\rightarrow DKer(1_{Tr(X)}\otimes  f)$.} \\\\
           Izomorfismus $\gamma_{\delta, X}:D(Ker(1_{Tr(X)}\otimes f)) \rightarrow \delta_*(DTr(X))$ 
           existuje a je dán předpisem 
           \begin{eqnarray}
             \gamma_{\delta, X}(\bar{z}) &=& \theta_{Tr(X),M,I}(z)+Im((-\circ f)_{DTr(X)}) \nonumber \\
             &=& [m\mapsto z(-\otimes m)]+Im((-\circ f)_{DTr(X)}), \nonumber
           \end{eqnarray}
           kde $z\in D(Tr(X)\otimes_A M)$ je takové, že $\bar{z}=(-\circ \iota)_I(z)=zi$.
        \end{thm}
        \begin{proof} 
          Nejprve dokážeme existenci tohoto izomorfismu.
          Aplikujeme-li funktor $D$ na exaktní posloupnost z \hyperref[mega-diagram]{Věty \ref*{mega-diagram}} \\\\
          \resizebox{14cm}{!}{\xymatrix{
          0 \ar@{->}[r] 
            & Ker(1_{Tr(X)}\otimes f) \ar@{->}[r]           
            & Tr(X)\otimes_A M \ar@{->}[r]
            & Tr(X)\otimes_A N \ar@{->}[r]          
            & Tr(X)\otimes_A L \ar@{->}[r] 
            & 0      
          }}\\\\
          a $Hom_A(-,DTr(X))$ na posloupnost $\delta$, dostaneme následující diagram \\\\
          \resizebox{14cm}{!}{\xymatrix{
          0 \ar@{->}[r]        
            & D(Tr(X)\otimes_A L) \ar@{->}[r] \ar@{->}[d]_{\simeq}^{\theta_{Tr(X),L,I}} 
            & D(Tr(X)\otimes_A N) \ar@{->}[r] \ar@{->}[d]_{\simeq}^{\theta_{Tr(X),N,I}} 
            & D(Tr(X)\otimes_A M) \ar@{->}[r] \ar@{->}[d]_{\simeq}^{\theta_{Tr(X),M,I}}   
            & D(Ker(1_{Tr(X)}\otimes f)) \ar@{->}[r] \ar@{-->}[d]_{\simeq}^{\gamma_{\delta, X}}            
            & 0  \\
          0 \ar@{->}[r]  
            & Hom_A(L, DTr(X)) \ar@{->}[r]   
            & Hom_A(N, DTr(X)) \ar@{->}[r]   
            & Hom_A(M, DTr(X)) \ar@{->}[r]   
            & \delta_*(DTr(X)) \ar@{->}[r]            
            & 0
          }}\\\\\\
          Pak protože $\theta_{Tr(X),M,I}$, $\theta_{Tr(X),N,I}$ i $\theta_{Tr(X),L,I}$ 
          jsou izomorfismy abelovských grup, existuje dle \hyperref[lemma-five]{Lemma \ref*{lemma-five}} 
          takový   izomorfismus abelovských grup \\\\
          \centerline{$\gamma_{\delta, X}:D(Ker(1_{Tr(X)}\otimes f)) \rightarrow \delta_*(DTr(X))$,} 
          \\\\
          že diagram komutuje.
          
          
          Pro $\bar h \in \underline{End}_A(X)$ máme
          \begin{eqnarray}
            \gamma_{\delta, X}(\bar h \bar z)&=& \gamma_{\delta, X}(z\circ (Tr(h)\otimes 1_M)\circ i) \nonumber \\
            &=& [a\mapsto z(Tr(h)(-)\otimes a)]+Im((-\circ f)_{DTr(X)}) \nonumber
          \end{eqnarray}      
         Ukážeme, že stejný prvek $\Upsilon_{DTr(X),L}/\sim$ obdržíme
         \begin{description}
           \item[(1)] identifikováním $[a\mapsto z(Tr(h)(-)\otimes a)]+Im((-\circ f)_{DTr(X)}) 
           $
           s třídou ekvivalence v $\Upsilon_{DTr(X),L}/\sim$.
           \item[(2)] identifikováním $[a\mapsto z(-\otimes a)]+Im((-\circ f)_{DTr(X)})$ 
            s třídou ekvivalence v $\Upsilon_{DTr(X),L}/\sim$ a poté 
            vynásobením $\bar h$, jako v  \hyperref[lem-ext-delta-jsou-moduly]{Lemma \ref*{lem-ext-delta-jsou-moduly}}.
         \end{description}
         
         Poznamenejme, že první případ vede ke krátké exaktní posloupnosti 
         vzniklé jako pushout $f$ a $[a\mapsto z(Tr(h)(-)\otimes a)]$, zatímco 
         druhý případ k pushoutu $f$ a složeného zobrazení \\\\
         \centerline{$DTr(h)\circ[a\mapsto z(-\otimes a)]=[a\mapsto DTr(h)(z(-\otimes a))]$.}\\\\
         Protože
         \begin{eqnarray}
           DTr(h)(z(-\otimes a)) &=& (-\circ Tr(h))_I(z(-\otimes a)) \nonumber 
           \\
           &=& z(-\otimes a)\circ Tr(h) \nonumber \\
           &=& z(Tr(h)(-)\otimes a), \nonumber 
         \end{eqnarray}
         vidíme, že \\\\
         \centerline{$
         DTr(h)\circ [a\mapsto z(-\otimes a)]=[a\mapsto z(Tr(h)(-)\otimes a)].
         $}\\\\
         Právě jsme ukázali, že $\gamma_{\delta, X}(\bar h\bar z)=\bar h \gamma_{\delta, X}(\bar z)$ 
         pro všechny $\bar z\in D(Ker(1_{Tr(X)}\otimes f))$ a $\bar h\in 
         \underline{End}_A(X)$. Tedy, že jde o izomorfismus 
         $\underline{End}_A(X)$-modulů.
         
         
          Přirozenost našeho izomorfismu dokazovat nebudeme. Důkaz je možné 
          nalézt v \cite{3} na stranách 109-113.
        \end{proof}
        
    \subsection{Izomorfismus $\omega_{\delta,X}$} 
    
      \begin{thm}\label{thm-omega}
        Položme \\\\
        \centerline{$\omega_{\delta,X}:=\gamma_{\delta,X}D\sigma^{-1}_{\delta,X}$.}  \\\\        
        Pak $\omega_{\delta,X}$ je izomorfismem $\underline{End}_A(X)$-modulů  \\\\
        \centerline{$\omega_{\delta,X}:D\delta^*(X)\rightarrow \delta_*(DTr(X))$,} \\\\
        který je přirozený v $\delta$ a $X$.        
      \end{thm}
      \begin{proof}
        Protože \\\\
        \centerline{$\sigma^{-1}_{\delta,X}:Ker(1_{Tr(X)}\otimes f)\rightarrow \delta^*(X)$} 
        \\\\
        je izomorfismem $\underline{End}_A(X)$-modulů, je \\\\
        \centerline{$D\sigma^{-1}_{\delta,X}:D\delta^*(X) \rightarrow DKer(1_{Tr(X)}\otimes f)$} 
        \\\\ 
        také izomorfismem $\underline{End}_A(X)$-modulů. Pak je 
        $\gamma_{\delta,X}D\sigma^{-1}_{\delta,X}$ jakožto homomorfismus vzniklý 
        složením dvou izomorfismů $\underline{End}_A(X)$-modulů také
        izomorfismem $\underline{End}_A(X)$-modulů. 
        
        Protože $D$ je funktor, plyne přirozenost $D\sigma^{-1}_{\delta,X}$ v $\delta$ a $X$ z 
        přirozenosti $\sigma^{-1}_{\delta,X}$ ve stejných proměnných.
        
        Navíc protože je složení dvou přirozených 
        transformací opět přirozená transformace, je i 
        $\omega_{\delta,X}:=\gamma_{\delta,X}D\sigma^{-1}_{\delta,X}$
        přirozená v proměnných  v $\delta$ a $X$.
      \end{proof}
      \clearpage
      
  \section{Algoritmus pro nalezení skoro štěpitelné posloupnosti} 
  
    \paragraph{ } Nyní popíšeme samotný algoritmus a dokážeme jeho správnost.
    Nadále budeme namísto  okruhu R pracovat s libovolným komutativním tělesem K. 
    Funktor $D$ tedy bude dle \hyperref[lem-dual-teleso]{Lemma \ref*{lem-dual-teleso}} 
    tvaru $D=Hom_K(-,K):mod(K)\rightarrow mod(K)$.      
    Navíc budeme pracovat s jednou pevně zvolenout exaktní 
    poslupností $\delta$ \\\\
      \centerline{\xymatrix{
        0 \ar@{->}[r] 
          & \Omega \ar@{->}[r]^i 
          & P_0 \ar@{->}[r]^t 
          & X \ar@{->}[r] 
          & 0
      },}\\\\
    kde $(P_0, t)$ je projektivní pokrytí modulu $X$, $\Omega:=Ker(t)$ a $i$ kanonické vnoření. Nechť navíc $(P_1, \omega)$ 
    je projektivní pokrytí $\Omega$ a $s:=i\omega$.  Pak máme projektivní 
    prezentaci modulu $X$:\\\\
      \centerline{\xymatrix{
           & P_1 \ar@{->}[r]^s
          & P_0 \ar@{->}[r]^t 
          & X \ar@{->}[r] 
          & 0
      },}\\\\
    Pevná volba posloupnosti $\delta$ je možná, protože moduly $P_0$, $P_1$ a $\Omega$ 
    jsou určeny modulem $X\in mod(A)$ jednoznačně až na izomorfismus.
    
    Navíc si vzledem k pevné volbě $\delta$ jednodušeji označme 
    homomorfismy z předchozí kapitoly. Položme:
    \begin{description}
      \item[(a)] $\sigma_X:=\sigma_{\delta,X}:\delta^*(X)\rightarrow Ker(1_{Tr(X)}\otimes i)$ 
      \item[(b)] $\gamma_X:=\gamma_{\delta,X}: DKer(1_{Tr(X)\otimes i})\rightarrow \delta_*(DTr(X))$ 
      \item[(c)]  $\omega_X:=\omega_{\delta,X}: D\delta^*(X) \rightarrow \delta_*(DTr(X))$ 
    \end{description}
    
    V následujících několika tvrzeních zjednodušíme výsledky z předchozích 
    částí, poté již zformulujeme algoritmus.
  
      \begin{lem}\label{lem-delta-as-hom}
        Nechť $Y\in mod(A)$, pak $\delta^*(Y)=\underline{Hom}_A(Y,X)$.
      \end{lem}
      \begin{proof}
        Připomeňme, že pro $Y\in mod(A)$ je $\delta^*(Y)$ definováno exaktností 
        následující posloupnosti: \\
        \centerline{\xymatrix{
          0 \ar@{->}[r] 
            & Hom_A(Y,\Omega) \ar@{->}[r]^{(i\circ-)_Y} 
            & Hom_A(Y,P_0) \ar@{->}[r]^{(t\circ-)_Y}
            & Hom_A(Y,X) \ar@{->}[r] 
            & \delta^*(Y) \ar@{->}[r] 
            & 0
        }}\\\\      
        Tedy $\delta^*(Y)$ je kojádro $(t\circ-)_Y$, neboli \\\\
        \centerline{$\delta^*(Y)=Hom_A(Y,X)/Im((t\circ-)_Y)$}* \\\\
        Zbývá dokázat, že $Im(t\circ-)_Y=P(Y,X)$.
        
        Pokud $u\in Im(t\circ-)_Y$, tak se $u$ faktorizuje skrze projektivní 
        A-modul $P_0$. Opačně, pokud se $u$ faktorizuje skrze nějaký projektivní 
        A-modul $P$, tak dle \hyperref[lem-faktorizuje-skrze-proj]{Lemma \ref*{lem-faktorizuje-skrze-proj}}
        se $u$ také faktorizuje skrze
        $P(X)$. Pak dle (*) dostáváme \\\\
        \centerline{$\delta^*(Y)=Hom_A(Y,X)/P(Y,X)=\underline{Hom}_A(Y,X)$.}
      \end{proof}
      
      \begin{thm}\label{thm-omega-x}
        $\omega_X$ je izomorfismem $\underline{End}_A(X)$-modulů: \\\\
        \centerline{$\omega_X: D\underline{End}_A(X) \rightarrow Ext_A^1(X,DTr(X))$}
      \end{thm}
      \begin{proof}
        Připomeňme z \hyperref[thm-omega]{Věty \ref*{thm-omega}}, že \\\\
        \centerline{$\omega_X: D\delta^*(X) \rightarrow\delta_*(DTr(X))$} \\\\
        \hyperref[lem-delta-as-hom]{Lemma \ref*{lem-delta-as-hom}} 
        imlikuje, že $\delta^*(X)=\underline{End}_A(X)$ a tedy \\\\
        \centerline{$D\delta^*(X)=D\underline{End}_A(X)$}. \\
        
        Navíc protože $X\in mod(A)$, máme $Tr(X)\in mod(A^{op})$ a $DTr(X)\in 
        mod(A)$.
        Aplikováním kontravariatního funktoru $Hom_A(-,DTr(X))$ na $\delta$ 
        dostaneme dle \cite{5} Theorem 7.3 následující exaktní 
        posloupnost: \\\\
        \centerline{$
          0\rightarrow 
          Hom_A(X,DTr(X))\rightarrow 
          Hom_A(P_0,DTr(X))\rightarrow
          Hom_A(\Omega(X),DTr(X))\rightarrow\ldots 
        $}
        \centerline{$
          \ldots\rightarrow 
          Ext^1_A(X,DTr(X))\rightarrow 
          Ext^1_A(P_0,DTr(X))
        $}\\
        
        Protože je $P_0$ projektivní, je $Ext_A^1(P_0,DTr(X))=0$ a my dostáváme 
        posloupnosti z \hyperref[def-delta-*]{Definice \ref*{def-delta-*}}, kde je $X$ 
        nahrazeno $DTr(X)$. A tedy: 
        \\\\
        \centerline{$\delta_*(DTr(X))=Ext^1_A(X,DTr(X))$.}
      \end{proof} 
    
      \paragraph{ }\label{phi-omega-nenul}
      Připomeňme \hyperref[lem-ext-delta-jsou-moduly]{Lemma \ref*{lem-ext-delta-jsou-moduly}}, 
      že $Ext^1_A(X,DTr(X)$ má strukturu konečně generovaného $\underline{End}_A(X)$-modulu 
      (tu jsme přenesli ztotožněním jeho prvků s třídami ekvivalence krátkých exaktních posloupností).
      Dále dle věty \hyperref[ekvivalence-upsilon-ext]{Věty \ref*{ekvivalence-upsilon-ext}} 
      máme izomorfismy:
      \begin{eqnarray}
         Soc_\Gamma(\Upsilon_{DTr(X),X}/\sim)&\simeq&(\hat{\Upsilon}_{DTr(X),X}/\sim) \nonumber \\
         Ext_A^1(V,U)&\simeq& (\Upsilon_{U,V}/\sim) \nonumber 
      \end{eqnarray}      
      Tedy $Soc_\Gamma(Ext^1_A(X,DTr(X))$ koresponduje s množinou $\tilde{\Upsilon}_{DTr(X),X}/\sim$ tříd 
      ekvivalence skoro štěpitelných posloupností $mod(A)$ končících v $X$.
      
      Navíc $Soc_\Gamma(Ext^1_A(X,DTr(X))$ je jako $\underline{End}_A(X)$-modul 
      jednoduchý a tedy dle \hyperref[lem-jednoduchy-modul-gen]{Lemma \ref*{lem-jednoduchy-modul-gen}}
      může být vygenerován každým svým prvkem. Díky izomorfismu $\omega_X$ ve tvaru z
      \hyperref[thm-omega-x]{Věty \ref*{thm-omega-x}} \\\\
        \centerline{$\omega_X: D\underline{End}_A(X) \rightarrow Ext_A^1(X,DTr(X))$,} 
        \\\\
      vidíme, že každý nenulový prvek $e\in Soc_\Gamma(D\underline{End}_A(X))$ 
      může být použit k vygenerování celého $\tilde{\Upsilon}_{DTr(X),X}/\sim$, jelikož
      \\\\
      \centerline{$\omega_X(e)\in Soc_\Gamma(Ext^1_A(X,DTr(X))\simeq (\tilde{\Upsilon}_{DTr(X),X}/\sim)$} \\\\
      bude nenulový.     
    
      \begin{lem}\label{lem-B-ker}
        Uvažujme identitu $\bar{1_X}\in\underline{End}_A(X)$. Nechť \\\\
        \centerline{$B_{Ker(1_{Tr(X)}\otimes i)}:=\{\sigma_X(\bar{1_X}),\omega_2\,\ldots,\omega_l\}$} \\\\
        je K-báze $Ker(1_{Tr(X)}\otimes i)$. Pak \\\\
        \centerline{$\gamma_X(d_{B_{Ker(1_{Tr(X)}\otimes i)}}(\sigma_X(\bar{1_X})))
        \in Soc_\Gamma(Ext^1_A(X,DTr(X)))$} \\\\
        je generátor.
      \end{lem}
      
      \begin{proof}
        Nechť \\\\
        \centerline{$\sigma^{-1}_X(B_{Ker(1_{Tr(X)}\otimes i)}):=\{\bar 1_X,\sigma^{-1}_X(\omega_2),\ldots,\sigma^{-1}_X(\omega_l)\}$} 
        \\\\
         je $K$-báze $\underline{End}_A(X)$ korespondující s $B_{Ker(1_{Tr(X)}\otimes 
         i)}$. Víme, že \\\\
         \centerline{$\bar 1_X\in Top_{\Gamma^{op}}(\underline{End}_A(X))$}\\\\
         je nenulový prvek, pak dle \hyperref[lem-soc-top]{Věty \ref*{lem-soc-top}}  je\\\\
         \centerline{$(d_{\sigma_X^{-1}(B_{Ker(1_{Tr(X)}\otimes i)})})(\bar 1_X)\in Soc_\Gamma(D\underline{End}_A(X))$}\\\\
         nenulový prvek a tedy protože $\omega_X$ je izomorfismem $\underline{End}_A(X)$-modulů, je 
         \begin{eqnarray}
           \omega_X ((d_{\sigma_X^{-1}(B_{Ker(1_{Tr(X)}\otimes i)})})(\bar 1_X)) 
           &=& 
           \underbrace{
             \gamma_X(D\sigma_X^{-1}) ((d_{\sigma_X^{-1}(B_{Ker(1_{Tr(X)}\otimes i)})})(\bar 1_X))
           }_{\in Soc_\Gamma(Ext_A^1(X,DTr(X)))}   
           \nonumber
         \end{eqnarray}
         nenulový prvek. Navíc dle \hyperref[lem-baze-dual-xi]{Lemma \ref*{lem-baze-dual-xi}} 
         víme, že \\\\
         \centerline{$
           (D\sigma_X^{-1}) (d_{\sigma_X^{-1}(B_{Ker(1_{Tr(X)}\otimes i)})}(\bar 1_X))
           = 
           d_{B_{Ker(1_{Tr(X)}\otimes i)}}(\sigma_X(\bar 1_X)) 
         $}\\\\
         a tedy \\\\
         \centerline{$
           \omega_X ((d_{\sigma_X^{-1}(B_{Ker(1_{Tr(X)}\otimes i)})})(\bar 1_X)) 
           = 
           \gamma_X( d_{B_{Ker(1_{Tr(X)}\otimes i)}}(\sigma_X(\bar 1_X))  )
         $.}\\\\
         Potom dle \hyperref[lem-jednoduchy-modul-gen]{Lemma \ref*{lem-jednoduchy-modul-gen}} 
         je tento nenulový prvek generátorem $Soc_\Gamma(Ext_A^1(X,DTr(X)))$.
      \end{proof}
      
      \begin{lem}\label{alg-vraci}
        Algoritmus výpočtu $\sigma_{\delta,X}$ nám při našem pevně zvoleném $\delta$ a 
        vstupu $\bar{1}_X$ vrátí \\
        \centerline{ $\sigma_{X}(\bar{1}_X)=\phi_\Omega(\omega)$,}\\\\
        kde $\omega$ je projektivní pokrytí $\Omega$. ($\phi_\Omega$
        jsme zavedli v \hyperref[def-phi]{Definici \ref*{def-phi}}) 
      \end{lem}
      \begin{proof}
        Podívejme se znovu na diagram z \hyperref[mega-diagram]{Věty \ref*{mega-diagram}} a 
        upravme ho dle naší posloupnosti $\delta$: \\\\
      \centerline{\xymatrix{
        & 
          & 
          & 
          & 0 \ar@{->}[d] 
          & \\
        & 0 \ar@{->}[d]
          & 0 \ar@{->}[d] 
          & 0 \ar@{->}[d] 
          & Ker(1_{Tr(X)}\otimes i) \ar@{_{(}->}[d]
          & \\
        0 \ar@{->}[r]
          & Hom_A(X,\Omega) \ar@{->}[r]^{(-\circ t)_\Omega} \ar@{->}[d]_{(i\circ-)_X} 
          & Hom_A(P_0,\Omega) \ar@{->}[r]^{(-\circ s)_\Omega} \ar@{->}[d]_{(i\circ-)_P_0} 
          & Hom_A(P_1,\Omega) \ar@{->}[r]^{\phi_\Omega} \ar@{->}[d]_{(i\circ-)_P_1} 
          & Tr(X)\otimes_A \Omega \ar@{->}[r] \ar@{->}[d]_{1_{Tr(X)}\otimes i} 
          & 0 \\
        0 \ar@{->}[r]
          & Hom_A(X,P_0) \ar@{->}[r]^{(-\circ t)_{P_0}} \ar@{->}[d]_{(t\circ-)_X} 
          & End_A(P_0) \ar@{->}[r]^{(-\circ s)_{P_0}} \ar@{->}[d]_{(t\circ-)_P_0} 
          & Hom_A(P_1,P_0) \ar@{->}[r]^{\phi_{P_0}} \ar@{->}[d]_{(t\circ-)_P_1} 
          & Tr(X)\otimes_A {P_0} \ar@{->}[r] \ar@{->}[d]_{1_{Tr(X)}\otimes t} 
          & 0 \\
        0 \ar@{->}[r]
          & End_A(X) \ar@{->}[r]^{(-\circ t)_X} \ar@{->}[d] 
          & Hom_A(P_0,X) \ar@{->}[r]^{(-\circ s)_X} \ar@{->}[d] 
          & Hom_A(P_1,X) \ar@{->}[r]^{\phi_X} \ar@{->}[d] 
          & Tr(X)\otimes_A X \ar@{->}[r] \ar@{->}[d] 
          & 0 \\
        & \underline{End}_A(X) \ar@{->}[d] 
          & 0
          & 0
          & 0
          & \\
        & 0
      }}\\\\
      Provedeme $\sigma_X$-algoritmus pro prvek $\bar 1_X\in \underline{End}_A(X)$. 
      Jeho vzorem je prvek $1_X\in {End}_A(X)$. Jako $u\in {End}_A(P_0)$ takový, 
      že $tu=t$, zvolíme jednoduše $1_{P_0}$. Dále hledáme prvek $v\in Hom_A(P_1,\Omega)$ 
      takový, že $iv=s$, což z definice splňuje projektivní pokrytí $\omega$ 
      modulu $\Omega$. To pak zobrazíme na $\phi_\Omega(\omega)$.
      \end{proof}
      
      \paragraph{Algoritmus pro výpočet generátoru $\tilde{\Upsilon}_{DTr(X),X}/\sim$}     
        \begin{description}
          \item[\space\space\space Vstup:] $X\in mod(A)$ nerozložitelný a 
          neprojektivní.
          \item[\space\space\space Výstup:] Generátor $0 \rightarrow DTr(X) \rightarrow E \rightarrow X \rightarrow 0$} 
            množiny $\tilde{\Upsilon}_{DTr(X),X}/\sim$.
         \item[\space\space\space Průběh:] 
            \begin{description} 
               \item[]
               \item[(a)] Spočtěme projektivní pokrytí ($P_0$, $t$) modulu $X$  \\\\
                   \centerline{\xymatrix{
                      Ker(t) \ar@{^{(}->}[r]^i & P_0 \ar@{->}[r]^t & X  
                    },}\\\\  a položme $\Omega:=Ker(t)$.
               \item[(b)] Vezměme $\phi_\Omega(\omega)\in Ker(1_{Tr(X)}\otimes i)$, kde $(P_1,\omega)$ 
                    je projektivní pokrytí $\Omega$ a rozšiřme ho na K-bázi: \\\\
                    \centerline{$B_{Ker(1_{Tr(X)}\otimes i)}:=
                    \{\phi_\Omega(\omega),\omega_2,\ldots,\omega_l\}$.}  \\\\
                   \centerline{\xymatrix{
                      P_1 \ar@{->}[r]^\omega & \Omega \ar@{^{(}->}[r]^i & P_0 \ar@{->}[r]^t & X  
                    },}\\
               \item[(c)] Rozšiřme $B_{Ker(1_{Tr(X)}\otimes i)}$ na K-bázi \\\\
                    \centerline{$B_{Tr(X)\otimes\Omega}:=
                    \{\phi_\Omega(\omega),\omega_2,\ldots,\omega_l,\omega_{l+1},\ldots,\omega_{l+m}\}$.} 
                    \\
               \item[(d)] Definujme homomorfismus A-modulů $\xi:\Omega\rightarrow DTr(X)$  
                    následovně. Pro $a\in \Omega$ definujme $\xi(a):Tr(X)\rightarrow K$ 
                    předpisem \\\\
                    \centerline{$q\mapsto$[první K-koeficient $q\otimes a$ vzhledem k bázi 
                    $B_{Tr(X)\otimes_A\Omega}]$.}\\
               \item[(e)] Položme $E$ rovno pushoutu $i$ a $\xi$. 
            \end{description}     
        \end{description}          
        \centerline{ \xymatrix{
        \Omega \ar[r]^\xi \ar[d]_i & DTr(X) \ar@{..}[d] \\ 
        P_0 \ar@{..}[r] & E
      } }\,
      
      \begin{thm}
        Algoritmus vrací generátor \\\\
        \centerline{$0 \rightarrow DTr(X) \rightarrow E \rightarrow X \rightarrow 0$} 
        \\\\
        všech skoro štěpitelných posloupností v $mod(A)$ končících v $X$.
      \end{thm}
      \begin{proof}
        Dle \hyperref[alg-vraci]{Lemma \ref*{alg-vraci}} máme
        \\\\\centerline{$\sigma_X(\bar 1_X)=\phi_\Omega(w)$}\\\\
        a $B_{Ker(1_{Tr(X)}\otimes i)}$ je jako v  
        \hyperref[lem-B-ker]{Lemma \ref*{lem-B-ker}} a tedy 
        \\\\\centerline{$\gamma_X \underbrace{( d_{B_{Ker(1_{Tr(X)}\otimes i)}}( \phi_\Omega(w) ) )}_{a}=: \bar y$}\\\\
        generuje $Soc_\Gamma(Ext_A^1(X,DTr(X)))$.
        
        Stejně jako v důkazu \hyperref[lem-ext-delta-jsou-moduly]{Lemma \ref*{lem-ext-delta-jsou-moduly}} 
        je prvek $\hat\Upsilon/\sim$ obdržen jako pushout $E$ morfismů $i$ a 
        libovolného reprezentantu $y\in Hom_A(\Omega, DTr(X))$ prvku $\bar y$.
        Protože $\gamma_X$ je kojádro zobrazení $\theta_{Tr(X),\Omega,K}$, tak 
        pokud je 
        \\\\\centerline{$y:=\theta_{Tr(X),\Omega,K}(z)$,}\\\\
        pro libovolný reprezentant $z\in (Tr(X)\otimes_A\Omega)$ prvku $\bar z$, 
        pak je $y$ reprezentant prvku $\gamma_X(\bar z)=\bar y$. Nechť $\mu$ 
        značí inkluzi 
        \\\\\centerline{$\mu:Ker(1_{Tr(X)}\otimes i)\to Tr(X)\otimes_A\Omega$.}\\\\
        Pak 
        \\\\\centerline{$D\mu=(-\circ\mu)_K:D(Tr(X)\otimes_A\Omega)\to DKer(1_{Tr(X)}\otimes i)$}\\\\
        je kanonická projekce a pro každé $z\in D(Tr(X)\otimes_A\Omega)$ takové, že
        \\\\\centerline{$z\mu=\bar z$,}\\\\
        je reprezentantem $\bar z$. Ukážeme, že volba
        \\\\\centerline{$z:=d_{B_{Ker(1_{Tr(X)}\otimes i)}}( \phi_\Omega(w) )$}\\\\
        tuto podmínku splňuje.
        
        Nechť $q\otimes a\in Ker(1_{Tr(X)}\otimes i)$. Aplikace našeho zvoleného $z$ 
        na $q\otimes a$ koresponduje s vyjádřením $q\otimes a$ vzhledem k bázi  
        $B_{Ker(1_{Tr(X)}\otimes i)}$
        a extrakcí prvního $K$-koeficientu. Protože $B_{Tr(X)\otimes_A\Omega}$ je 
        pouze rozšířením $B_{Ker(1_{Tr(X)}\otimes i)}$, tak se výsledek nezmění, 
        pokud $q\otimes a$ vnoříme do $Tr(X)\otimes_A \Omega$ a vyjádříme ho 
        vzhledem k bázi $B_{Tr(X)\otimes_A\Omega}$ před extrakcí prvního 
        $K$-koeficientu, což odpovídá aplikaci $d_{B_{Tr(X)\otimes_A \Omega)}}( \phi_\Omega(w) ) \mu$ 
        na $q\otimes a$.
        
        Pak musíme spočíst pushout $i$ a 
        $\theta_{Tr(X),\Omega,K}(d_{B_{Tr(X)\otimes_A \Omega)}}( \phi_\Omega(w) 
        ))$, abychom obdrželi hledanou skoro štěpitelnou posloupnost. Připomeňme 
        \hyperref[thm-adjunkce]{Větu \ref*{thm-adjunkce}}, že
        \begin{eqnarray}
          \theta_{Tr(X),\Omega,K}(d_{B_{Tr(X)\otimes_A \Omega)}}( \phi_\Omega(w) ))(a)
            &=& [q\mapsto d_{B_{Tr(X)\otimes_A \Omega)}}( \phi_\Omega(w) )(q\otimes a)]\nonumber \\
            &=& [q\mapsto 1.\, koef.\, q\otimes a\,vzhledem\,k\,B_{Tr(X)\otimes_A \Omega)}]\nonumber \\
            &=& \xi(a) \nonumber 
        \end{eqnarray}
        pro každé $a\in\Omega$. A tedy 
        \\\\\centerline{$\theta_{Tr(X),\Omega,K}(d_{B_{Tr(X)\otimes_A \Omega)}}( \phi_\Omega(w) ))=\xi$,}\\\\
        tím je důkaz hotov a my se můžeme pustit do jeho implementace.
      \end{proof}











